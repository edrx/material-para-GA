% (find-angg "LATEX/material-para-GA.tex")
% (defun c () (interactive) (find-LATEXsh "lualatex -record material-para-GA.tex"))
% (defun d () (interactive) (find-xpdfpage "~/LATEX/material-para-GA.pdf"))
% (defun b () (interactive) (find-zsh "bibtex material-para-GA; makeindex material-para-GA"))
% (defun e () (interactive) (find-LATEX "material-para-GA.tex"))
% (defun u () (interactive) (find-latex-upload-links "material-para-GA"))
% (find-xpdfpage "~/LATEX/material-para-GA.pdf")
% (find-sh0 "cp -v  ~/LATEX/material-para-GA.pdf /tmp/")
% (find-sh0 "cp -v  ~/LATEX/material-para-GA.pdf /tmp/pen/")
%   file:///home/edrx/LATEX/material-para-GA.pdf
%               file:///tmp/material-para-GA.pdf
%           file:///tmp/pen/material-para-GA.pdf
% http://angg.twu.net/LATEX/material-para-GA.pdf

% «.Psection»			(to "Psection")
% «.mypsection»			(to "mypsection")
% «.picturedots»		(to "picturedots")
% «.pictOuv»			(to "pictOuv")
% «.pictABCDE»			(to "pictABCDE")
% «.cells»			(to "cells")
% «.tikz-defs»			(to "tikz-defs")
% «.pictureFxy»			(to "pictureFxy")
% «.calcpoints»			(to "calcpoints")
%
% «.title-page»			(to "title-page")
% «.introducao»			(to "introducao")
%
% «.coisas-muito»		(to "coisas-muito")
% «.dicas»			(to "dicas")
% «.substituicao»		(to "substituicao")
% «.matrizes»			(to "matrizes")
% «.comprehension»		(to "comprehension")
% «.comprehension-tables»	(to "comprehension-tables")
% «.comprehension-ex123»	(to "comprehension-ex123")
% «.comprehension-prod»		(to "comprehension-prod")
% «.comprehension-gab»		(to "comprehension-gab")
% «.retas»			(to "retas")
% «.pontos-e-vetores»		(to "pontos-e-vetores")
% «.pontos-e-vetores-graficamente»  (to "pontos-e-vetores-graficamente")
% «.retas-de-novo»		(to "retas-de-novo")
% «.intersecoes-de-retas»	(to "intersecoes-de-retas")
% «.sistemas-de-coordenadas»	(to "sistemas-de-coordenadas")
% «.sistemas-de-coordenadas-2»	(to "sistemas-de-coordenadas-2")
% «.sistemas»			(to "sistemas")
% «.sistemas-2»			(to "sistemas-2")
% «.varias-coords»		(to "varias-coords")
% «.varias-coords-2»		(to "varias-coords-2")
% «.Fxy»			(to "Fxy")
% «.pitagoras»			(to "pitagoras")
% «.normas»			(to "normas")
% «.uma-demonstracao-errada»	(to "uma-demonstracao-errada")
% «.propriedades-basicas»	(to "propriedades-basicas")
% «.dicas-V-F-justifique»	(to "dicas-V-F-justifique")
% «.propriedades-basicas-2»	(to "propriedades-basicas-2")
% «.propriedades-de-normas»	(to "propriedades-de-normas")
% «.demonstracao-comentada»	(to "demonstracao-comentada")
% «.projecao-ortogonal»		(to "projecao-ortogonal")
% «.projecoes-no-olhometro»	(to "projecoes-no-olhometro")
% «.propriedades-da-projecao»	(to "propriedades-da-projecao")
% «.senos-e-cossenos»		(to "senos-e-cossenos")
% «.areas-e-determinantes»	(to "areas-e-determinantes")
% «.areas-e-determinantes-2»	(to "areas-e-determinantes-2")
% «.pontos-mais-proximos»	(to "pontos-mais-proximos")
% «.circulos»			(to "circulos")
% «.circulos-2»			(to "circulos-2")
% «.decomp-ang»			(to "decomp-ang")
% «.distancia-ponto-reta»	(to "distancia-ponto-reta")
%
% «.conicas-e-R3»		(to "conicas-e-R3")
%
% «.areas-em-R3»		(to "areas-em-R3")
% «.R3-retas-e-planos»		(to "R3-retas-e-planos")
% «.R3-retas-e-planos-2»	(to "R3-retas-e-planos-2")
% «.determinantes-em-R3»	(to "determinantes-em-R3")
% «.determinantes-em-R3-2»	(to "determinantes-em-R3-2")
% «.cross-prod»			(to "cross-prod")
% «.alguns-usos-do-x»		(to "alguns-usos-do-x")
%
% «.indice»			(to "indice")
% «.git»			(to "git")

\documentclass[oneside]{book}
\usepackage[colorlinks,urlcolor=DarkRed]{hyperref} % (find-es "tex" "hyperref")
\usepackage[x11names,svgnames]{xcolor} % (find-es "tex" "xcolor")
\usepackage{amsmath}
\usepackage{amsfonts}
\usepackage{amssymb}
\usepackage{pict2e}
\usepackage{color}                % (find-LATEX "edrx15.sty" "colors")
\usepackage{colorweb}             % (find-es "tex" "colorweb")
\usepackage{tikz}
\usepackage{boxedminipage}
%
% (find-dn6 "preamble6.lua" "preamble0")
\usepackage{proof}   % For derivation trees ("%:" lines)
\input diagxy        % For 2D diagrams ("%D" lines)
%\xyoption{curve}     % For the ".curve=" feature in 2D diagrams
%
\usepackage{edrx15}               % (find-angg "LATEX/edrx15.sty")
\input edrxaccents.tex            % (find-angg "LATEX/edrxaccents.tex")
\input edrxchars.tex              % (find-LATEX "edrxchars.tex")
\input edrxheadfoot.tex           % (find-dn4ex "edrxheadfoot.tex")
\input edrxgac2.tex               % (find-LATEX "edrxgac2.tex")
%
\begin{document}


\catcode`\^^J=10
\directlua{dofile "dednat6load.lua"}  % (find-LATEX "dednat6load.lua")

% \catcode`\^^J=10
% \directlua{dednat6dir = "dednat6/"}
% \directlua{dofile(dednat6dir.."dednat6.lua")}
% \directlua{texfile(tex.jobname)}
% \directlua{verbose()}
% \directlua{output(preamble1)}
% \def\expr#1{\directlua{output(tostring(#1))}}
% \def\eval#1{\directlua{#1}}
% \def\pu{\directlua{pu()}}

\directlua{dofile "edrxtikz.lua"} % (find-LATEX "edrxtikz.lua")
\directlua{dofile "edrxpict.lua"} % (find-LATEX "edrxpict.lua")
%L V.__tostring = function (v) return format("(%.3f,%.3f)", v[1], v[2]) end
%L V.__tostring = function (v) return pformat("(%s,%s)",    v[1], v[2]) end






\def\erro{\operatorname{erro}}
\def\setofpt  #1 #2 #3 #4 {\setofet{(#1,#2)+t\VEC{#3,#4}}}
\def\setofpu  #1 #2 #3 #4 {\setofeu{(#1,#2)+u\VEC{#3,#4}}}

% (find-LATEXfile "2016-1-GA-material.tex" "setofet")
\def\setofet  #1{\setofst{#1}{t∈\R}}
\def\setofeu  #1{\setofst{#1}{u∈\R}}
\def\setofpt  #1 #2 #3 #4 {\setofet{(#1,#2)+t\VEC{#3,#4}}}
\def\setofpu  #1 #2 #3 #4 {\setofeu{(#1,#2)+u\VEC{#3,#4}}}

\unitlength=5pt


% «Psection» (to ".Psection")
% (find-es "tex" "section")
\makeatletter
\newcommand\Psection{\@startsection{Psection}{1}{\z@}%
                                    {0ex}%
                                    {\ssk}%
                                    {\normalfont\normalsize\bfseries}}
\makeatother
\newcounter{Psection}
% \Psection{Foo}
% \Psection{Bar}

% «mypsection» (to ".mypsection")
% (find-es "tex" "protect")
% (find-angg ".emacs" "eewrap-mypsection")
% \def\mypsection#1#2{\label{#1}{\bf #2}\ssk}

% (find-es "tex" "page-numbers")
%L psections = {}
%L psectionstex = function ()
%L     local f = function(A)
%L         return format("\\mypsectiontex{%s}{%s}", A[1], A[2])
%L       end
%L     return mapconcat(f, psections, "\n")
%L   end
\def\mypsectiontex#1#2{\par\pageref{#1} #2}
\def\mypsectionstex{\expr{psectionstex()}}
\pu

\def\mypsectionadd#1#2{\directlua{table.insert(psections, {"#1", [[#2]]})}}
\def\mypsection   #1#2{\label{#1}{\bf #2}\mypsectionadd{#1}{#2}\ssk}
%\def\mypsection   #1#2{\label{#1}{\bf #2}\mypsectionadd{#1}{\protect{#2}}\ssk}

% (find-es "tex" "protect")



% «picturedots» (to ".picturedots")
% (find-LATEX "edrxpict.lua" "pictdots")
% (find-LATEX "edrxgac2.tex" "pict2e")
% (to "comprehension-gab")
%
\def\beginpicture(#1,#2)(#3,#4){\expr{beginpicture(v(#1,#2),v(#3,#4))}}
\def\pictaxes{\expr{pictaxes()}}
\def\pictdots#1{\expr{pictdots("#1")}}
\def\picturedots(#1,#2)(#3,#4)#5{%
  \vcenter{\hbox{%
  \beginpicture(#1,#2)(#3,#4)%
  \pictaxes%
  \pictdots{#5}%
  \end{picture}%
  }}%
}

\unitlength=5pt

%L p = function (a, b) return O + a*uu + b*vv end

%L --  «pictOuv» (to ".pictOuv")
%L pictOOuuvv = function (OO, xx, yy, OOtext, xxtext, yytext, vtextdist, Otextdist)
%L     local bprint, out = makebprint()
%L     local xxpos = OO + xx/2 + xx:rotright():unit(vtextdist)
%L     local yypos = OO + yy/2 + yy:rotleft() :unit(vtextdist)
%L     local OOpos = OO + (-xx-yy):unit(Otextdist or vtextdist)
%L     local f = function (str) return (str:gsub("!", "\\")) end
%L     bprint("\\Vector%s%s", OO, OO+xx)
%L     bprint("\\Vector%s%s", OO, OO+yy)
%L     bprint("\\put%s{\\cell{%s}}", OOpos, f(OOtext))
%L     bprint("\\put%s{\\cell{%s}}", xxpos, f(xxtext))
%L     bprint("\\put%s{\\cell{%s}}", yypos, f(yytext))
%L     return out()
%L   end
%L -- sysco = pictOOuuvv
\def\pictOuv(#1,#2){
  {\color{GrayPale}\expr{pictpgrid(0,0,4,4)}}
  \pictaxes
  {\linethickness{1.0pt}
   \expr{pictOOuuvv(O, uu, vv,  "O", "!uu", "!vv", #1, #2)}
  }
}

%L -- «pictABCDE» (to ".pictABCDE")
%L -- Used to draw "F"s in:
%L -- (find-LATEX "2018-1-GA-material.tex" "sistemas-de-coordenadas")
%L -- (find-LATEX "2018-1-GA-material.tex" "sistemas-de-coordenadas" "pictABCDE")
%L -- (gam181p 15 "sistemas-de-coordenadas")
%L tt = v(1, 0)
%L pictABCDE = function (aang, bang, cang, dang, eang)
%L     local bprint, out = makebprint()
%L     local AA, BB, CC, DD, EE = p(1,1), p(1,3), p(3,3), p(1,2), p(2,2)
%L     local f = function (str) return (str:gsub("!", "\\")) end
%L     bprint("\\Line%s%s", AA, BB)
%L     bprint("\\Line%s%s", BB, CC)
%L     bprint("\\Line%s%s", DD, EE)
%L     bprint("\\put%s{\\closeddot}", AA)
%L     bprint("\\put%s{\\closeddot}", BB)
%L     bprint("\\put%s{\\closeddot}", CC)
%L     bprint("\\put%s{\\closeddot}", DD)
%L     bprint("\\put%s{\\closeddot}", EE)
%L     bprint("\\put%s{\\cell{%s}}", AA + tt:rot(aang), "A")
%L     bprint("\\put%s{\\cell{%s}}", BB + tt:rot(bang), "B")
%L     bprint("\\put%s{\\cell{%s}}", CC + tt:rot(cang), "C")
%L     bprint("\\put%s{\\cell{%s}}", DD + tt:rot(dang), "D")
%L     bprint("\\put%s{\\cell{%s}}", EE + tt:rot(eang), "E")
%L     return out()
%L   end
\def\pictABCDE(#1,#2,#3,#4,#5){
  {\linethickness{1.0pt}
   \expr{pictABCDE(#1,#2,#3,#4,#5)}
  }
}

\pu


% «cells» (to ".cells")
% (find-es "tex" "fbox")

\def\cellhr#1{\hbox to 0pt    {\cellfont${#1}$\hss}}
\def\cellhc#1{\hbox to 0pt{\hss\cellfont${#1}$\hss}}
\def\cellhl#1{\hbox to 0pt{\hss\cellfont${#1}$}}
\def\cellva#1{\setbox0#1\raise \dp0       \box0}
\def\cellvm#1{\setbox0#1\lower \celllower \box0}
\def\cellvb#1{\setbox0#1\lower \ht0       \box0}

\def\cellnw  #1{\cellva{\cellhl{#1}}}
 \def\celln  #1{\cellva{\cellhc{#1}}}
  \def\cellne#1{\cellva{\cellhr{#1}}}
\def\cellw   #1{\cellvm{\cellhl{#1}}}
 \def\celle  #1{\cellvm{\cellhr{#1}}}
\def\cellsw  #1{\cellvb{\cellhl{#1}}}
 \def\cells  #1{\cellvb{\cellhc{#1}}}
  \def\cellse#1{\cellvb{\cellhr{#1}}}

\newdimen\cellsep
\cellsep=4pt
\def\addcellsep{%
  \setbox0=\hbox{\kern\cellsep\box0\kern\cellsep}%
  \ht0=\ht0 plus \cellsep%
  \dp0=\dp0 plus \cellsep%
  \box0%
}
\def\cellsp#1{%
  \setbox0=\hbox{#1}%
  \addcellsep%
  \box0%
}




% «tikz-defs» (to ".tikz-defs")
%
% \mygrid and \myaxes
% (find-es "tikz" "mygrid")
\tikzset{mycurve/.style=very thick}
\tikzset{axis/.style=semithick}
\tikzset{tick/.style=semithick}
\tikzset{grid/.style=gray!20,very thin}
\tikzset{anydot/.style={circle,inner sep=0pt,minimum size=1.2mm}}
\tikzset{opdot/.style={anydot, draw=black,fill=white}}
\tikzset{cldot/.style={anydot, draw=black,fill=black}}
%
\def\mygrid(#1,#2) (#3,#4){
  \clip              (#1-0.4, #2-0.4) rectangle (#3+0.4, #4+0.4);
  \draw[step=1,grid] (#1-0.2, #2-0.2) grid      (#3+0.2, #4+0.2);
  \draw[axis] (-10,0) -- (10,0);
  \draw[axis] (0,-10) -- (0,10);
  \foreach \x in {-10,...,10} \draw[tick] (\x,-0.2) -- (\x,0.2);
  \foreach \y in {-10,...,10} \draw[tick] (-0.2,\y) -- (0.2,\y);
}
\def\myaxes(#1,#2) (#3,#4){
  \clip              (#1-0.4, #2-0.4) rectangle (#3+0.4, #4+0.4);
 %\draw[step=1,grid] (#1-0.2, #2-0.2) grid      (#3+0.2, #4+0.2);
  \draw[axis] (-20,0) -- (20,0);
  \draw[axis] (0,-20) -- (0,20);
  \foreach \x in {-20,...,20} \draw[tick] (\x,-0.2) -- (\x,0.2);
  \foreach \y in {-20,...,20} \draw[tick] (-0.2,\y) -- (0.2,\y);
}

% Grid color
\tikzset{grid/.style=gray!50,very thin}

\def\tikzp#1{\mat{\begin{tikzpicture}#1\end{tikzpicture}}}

\def\mydraw       #1;{\draw [mycurve]  \expr{#1};}
\def\mydot        #1;{\node [cldot] at \expr{#1} [] {};}
\def\myldot #1 #2 #3;{\node [cldot] at \expr{#1} [label=#2:${#3}$] {};}
\def\myseg     #1 #2;{\draw [mycurve]  \expr{#1} -- \expr{#2};}
\def\mylabel #1 #2 #3;{\node []     at \expr{#1} [label=#2:${#3}$] {};}
\def\myseggrid  #1 #2;{\draw [grid]    \expr{#1} -- \expr{#2};}

% \myvgrid, for things like this:
% (find-xpdfpage "~/LATEX/2016-1-GA-material.pdf" 6)
\def\myvgrid{
  \myseggrid p(0,0) p(0,4);
  \myseggrid p(1,0) p(1,4);
  \myseggrid p(2,0) p(2,4);
  \myseggrid p(3,0) p(3,4);
  \myseggrid p(4,0) p(4,4);
  \myseggrid p(0,0) p(4,0);
  \myseggrid p(0,1) p(4,1);
  \myseggrid p(0,2) p(4,2);
  \myseggrid p(0,3) p(4,3);
  \myseggrid p(0,4) p(4,4);
  \draw [->] \expr{p(0,0)} -- \expr{p(0,1)};
  \draw [->] \expr{p(0,0)} -- \expr{p(1,0)};
}


% «pictureFxy» (to ".pictureFxy")
\def\tcell#1{\lower\celllower\hbox to 0pt{\hss\cellfont#1\hss}}
\def\pictureFxy(#1,#2)(#3,#4)#5{%
  \vcenter{\hbox{%
  \beginpictureb(#1,#2)(#3,#4){.7}%
  {\color{GrayPale}%
   \Line(#1,0)(#3,0)%
   \Line(0,#2)(0,#4)%
  }
  \expr{pictFxy("#5")}
  \end{picture}%
  }}%
}


% «calcpoints» (to ".calcpoints")
%L calcpoints = function (str)
%L     local f = function (e1, e2) return format("(%s,%s)", expr(e1), expr(e2)) end
%L     return (str:gsub("<(.-),(.-)>", f))
%L   end
%L calcpoints1 = function (str) return (calcpoints(str):gsub("!", "\\")) end
%L
%L calcpoints2 = function (str)
%L     local result = str:gsub("<(.-)>", pformatexpr):gsub("!", "\\")
%L     print(result)
%L     return result
%L   end
\pu
%
\def\Calcpoints#1{\expr{calcpoints1("#1")}}
\def\CalcPoints#1{\expr{calcpoints2("#1")}}

\def\ang{\operatorname{ang}}
\def\det{\operatorname{det}}





%  _____ _ _         _       
% |_   _(_) |_ _   _| | ___  
%   | | | | __| | | | |/ _ \ 
%   | | | | |_| |_| | | (_) |
%   |_| |_|\__|\__,_|_|\___/ 
%                            
% «title-page» (to ".title-page")
% (find-LATEX "2018tug-dednat6.tex" "title-page")

\thispagestyle{empty}

\begin{tabular}[b]{c}
{\huge {\bf Material complementar}} \\
{\huge {\bf para Geometria Analítica}} \\
% {\Large Lua\LaTeX{} that understands} \\
\\
%\includegraphics[width=2cm]{2018tug-edrx-hoop.png}\\
Eduardo Ochs -
$〈${\tt eduardoochs@gmail.com}$〉$ \\
\url{http://angg.twu.net/material-para-GA.html} \\
RCN/CURO/UFF, 21/fev/2020 \\
\end{tabular}


\bsk
\bsk

$$\unitlength=15pt
    \vcenter{\hbox{%
     \beginpicture(0,0)(7,5)%
     \pictgrid%
     %\pictaxes
     {\linethickness{1.0pt}
      \Vector(1,1)(6,4)     \put(3,3){\cell{\ww}}
      \Vector(1,1)(5,1)     \put(4,0.2){\cell{λ\uu}}
      \Vector(5,1)(6,4)     \put(6,2.2){\cell{\vv}}
      \Vector(1,0.7)(3,0.7) \put(2,0){\cell{\uu}}   
     }
     \end{picture}%
   }}
$$



\newpage


%  ___       _                 _                       
% |_ _|_ __ | |_ _ __ ___   __| |_   _  ___ __ _  ___  
%  | || '_ \| __| '__/ _ \ / _` | | | |/ __/ _` |/ _ \ 
%  | || | | | |_| | | (_) | (_| | |_| | (_| (_| | (_) |
% |___|_| |_|\__|_|  \___/ \__,_|\__,_|\___\__,_|\___/ 
%                                                      
% «introducao» (to ".introducao")
% (mpgp 2 "introducao")
% (mpg    "introducao")
% (gam181p 2 "dicas")

% \mypsection {introducao} {Introdução (2019sep16)}

\mypsection {introducao} {Introdução}

% (find-TH "material-para-GA")
%    http://angg.twu.net/material-para-GA.html
% file:///home/edrx/TH/L/material-para-GA.html

% Introdução nova:



Quando a gente leciona uma disciplina a gente acaba preparando
material que complemente os livros nos pontos em que os alunos
costumam ter mais dúvidas... e nos últimos anos em que eu lecionei
Geometria Analítica a grande maioria das dúvidas --- pelo menos dos
alunos que se manifestavam --- eram sobre como passar entre casos
gerais e casos particulares: os alunos não sabiam como transformar os
teoremas dos livros (``muito abstratos''!) em casos concretos, não
sabiam nem como começar a tentar generalizar algo que eles vissem num
caso particular, não sabiam escrever uma hipótese, não sabiam {\sl
  testar} uma hipótese, não sabiam nem mesmo testar se uma determinada
reta desenhada num plano correspondia a uma determinada equação de
reta...

Esses alunos não sabiam estudar pelos livros. Aliás, a maioria deles
nem sabia {\sl como} estudar --- eles achavam que tinham que decorar
fórmulas e procedimentos, mas tinham muito pouca noção do que cada
trecho das fórmulas e procedimentos queria dizer, e muitos deles até
achavam que {\sl não dava tempo} de aprender os detalhes direito, e
isso acabava deixando eles paralisados.

\msk

Este PDF é um ``work in progress''. Ele é essencialmente a versão de
2018.1 do material que eu usava com as minhas turmas de GA pra tentar
complementar os livros-texto que tínhamos disponíveis online ou na
biblioteca, que eram estes aqui:

\msk

Delgado, J., Frensel, K. e Santo, N.E., {\sl Geometria Analítica 1}, CEDERJ.

Disponível em:
\url{https://canalcederj.cecierj.edu.br/recurso/4690}

\ssk

Venturi, J.J., {\sl Álgebra Vetorial e Geometria Analítica} e

{\sl Cônicas e Quádricas}. Ed.\ Unificado. Disponível em:

\url{https://www.geometriaanalitica.com.br/}

\ssk

Reis, G.L., e Silva, V.V., {\sl Geometria Analítica}. LTC.

\ssk

Steinbruch, A., e Winterle, P., {\sl Geometria Analítica}. Ed Makron.

\ssk

Boulos, P., e Camargo, I. de., {\sl Geometria Analítica - um
  tratamento vetorial}.

Ed.\ Prentice Hall Brasil.

% Baldin, Yuriko Yamamoto. Geometria Analítica para Todos. EdUFSCAR.

\msk

Tem várias coisas que eu gostaria de modificar aqui, mas que não tenho
tempo agora... as principais são: 1) a abordagem de vetores (que ficou
axiomática demais), 2) a parte sobre senos e cossenos (em 2018.1 eu
descobri, tarde demais, que os alunos aprendiam {\sl bem} melhor de
outro jeito), 3) o PDF deveria ter muitas referências explícitas aos
livros pra forçar os alunos a compararem a nossa abordagem com as dos
livros, mas por enquanto não tem quase nenhuma, 4) eu perdi o
código-fonte em \LaTeX{} dos exercícios e figuras que eu preparei
sobre cônicas, e ainda não tive tempo de redigitá-los e inclui-los
aqui... mas aqui tem um link pro PDF dele:

\msk

\url{http://angg.twu.net/LATEX/2018-1-GA-conicas.pdf}

\bsk

Praticamente todo o material que eu produzi para as minhas aulas de
Geometria Analítica no PURO/UFF está disponível online, incluindo
fotos de praticamente todos os quadros a partir de 2012.2 --- mesmo os
das aulas mal preparadas --- e em 2013.2 eu aprendi a fazer ``versões
imprimíveis'' dos quadros aumentando o contraste das fotos e
produzindo um arquivo PDF com todos os quadros do semestre. Veja os
links abaixo:

\msk

% (find-TH "2018.1-GA")
\url{http://angg.twu.net/2018.1-GA.html}

\url{http://angg.twu.net/2018.1-GA/2018.1-GA.pdf}

\url{http://angg.twu.net/2018.1-GA/Makefile.html}


\bsk
\bsk
\bsk



{\bf As mesmas idéias num nível mais alto}

A partir do meio de 2018 eu comecei a aplicar algumas das idéias que
estão por trás deste material --- {\sl que o melhor modo ensinar GA
  ``para crianças'' é trabalhar em casos particulares e casos gerais
  ao mesmo tempo, em paralelo} --- em áreas mais avançadas, como
Teoria de Categorias, e a apresentá-las em congressos, seminários,
workshops e artigos. Links:

\msk

\url{http://angg.twu.net/math-b.html\#zhas-for-children-2}

\url{http://angg.twu.net/logic-for-children-2018.html}

\url{http://angg.twu.net/math-b.html\#2020-tallinn}




% Eu ensinei Geometria Analítica no PURO/UFF de 2011.1 e 2018.1, e nos
% últimos semestres comecei a produzir um material \LaTeX ado que {\sl
%   complementava} os livros-texto - obs: os livros que eu mais usei
% foram o seu do CEDERJ e o do Reis e Silva - porque muitos dos nossos
% calouros tinham uma base matemática tão ruim que não conseguiam
% entender nada dos livros e precisavam de muito trabalho com exemplos
% mais concretos...
% 
% Durante esse tempo eu tentei várias vezes fazer contato com outras
% pessoas que também estivessem lidando com alunos com as mesmas
% dificuldades de abstração que os meus, mas muito pouca gente me
% respondia, e as pessoas eram muito reticentes em mostrar o material
% delas...
% 
% Aí eu fui pondo na rede tudo que eu fazia, inclusive fotos de
% quadros - mesmo as das aulas que eu não tinha preparado bem o
% suficiente...
% 
% Bom, a coisa principal é esse PDF aqui. Só há uma semana atrás eu
% consegui fazer uma introdução que deixa claro o que ele é... na
% verdade ela ainda não está pronta, falta eu digitar um monte de
% anotações em papel, incluir a bibliografia e um monte de links,
% explicar que isso é um "work in progress", e dizer quais são as seções
% que eu mudaria totalmente se tivesse tempo...
% 
% E agora que essa introdução tá ficando pronta é que me ocorreu que eu
% devia 1) mostrar isso pra você, 2) perguntar se você conhece alguma
% lista de discussão ou algo assim de professores de Geometria
% Analítica...
% 
% Link pro PDF (quero terminar a introdução dele e uma página web sobre
% ele até de noite):


% http://angg.twu.net/2018.1-GA.html
% https://geometrianaliticauff.wordpress.com/2015/12/02/bibliografia/

% Reis, G. L. e Silva, V. V. Geometria Analítica. LTC.
% Baldin, Yuriko Yamamoto. Geometria Analítica para Todos. EdUFSCAR.
% Venturi, J.J. Álgebra Vetorial e Geometria Analítica. Ed. Unificado. Disponibilizado em www.geometriaanalitica.com.br
% Venturi, J.J. Cônicas e Quádricas. Ed. Unificado. Disponibilizado em www.geometriaanalitica.com.br
% Steinbruch, A. e Winterle, P. Geometria Analítica. Ed Makron.
% Boulos, P. e Camargo, I. de. Geometria Analítica – um tratamento vetorial. Ed. Prentice Hall Brasil.
% Delgado, J., Frensel, K. e Santo, N.E., Geometria Analítica 1 e 2, CEDERJ.


% Introdução velha:
%
% Boa parte dos nossos alunos de Engenharia de Produção e Ciência da
% Computação --- os cursos de exatas do campus de Rio das Ostras ---
% entram na universidade vindos de um ensino médio fraquíssimo, com
% pouquíssima base matemática; por exemplo, muitos deles nunca viram um
% teorema na vida...
% 
% Este material é uma tentativa de {\sl complementar} os livros-texto
% que usamos nos nossos cursos de Geometria Analítica --- listados no
% fim desta introdução --- de uma forma que torne os livros e as aulas
% mais acessíveis para estes ``alunos sem base'' (``ASBs''). A
% característica mais deseperadora dos ASBs é que eles querem ``aprender
% as fórmulas'' mas não sabem testar seus resultados, e aliás não sabem
% nem verificar se o que eles escrevem está sintaticamente correto...
% 
% \def\twoninenyninelemma#1#2{
%   \fbox{
%   $\begin{array}{rcl}
%    2^{#2}-2^{#1} &=& 2^{1+#1}-2^{#1} \\
% %                &=& 2^{1}·2^{#1}-2^{#1} \\
%                  &=& 2·2^{#1}-1·2^{#1} \\
%                  &=& (2-1)·2^{#1} \\
% %                &=& 1·2^{#1} \\
%                  &=& 2^{#1} \\
%    \end{array}
%   $}
% }
% 
% $$
%   \twoninenyninelemma{n}{n+1}
%   \twoninenyninelemma{99}{100}
% $$


% https://mail.google.com/mail/ca/u/0/#search/reginaldo/FMfcgxvwzcBWHctfnghRHbxwrQXZqplX

% Oi Reginaldo,
% 
% não, eu não espero que vocês leiam o meu material de GA. Na internet
% em que eu cresci quando a gente dá um link pra alguma coisa a gente
% não espera que as pessoas leiam aquilo - a gente só espera, no máximo,
% que ALGUMAS PESSOAS abram ele, scrollem ele durante dez segundos e
% tenham uma noção de que aquilo existe, de quão disponível está, e de
% onde encontrá-lo se quiserem olhar de novo algum dia.
% 
% Que bom que você citou não ter tempo - isso foi um dos motivos pelos
% quais eu comecei a preparar esse material alguns semestres atrás.
% Quero explicar tudo direito num texto que eu pretendo escrever depois
% de terminar os gabaritos das provas de GA e de Cálculo 2 que eu dei
% ontem, mas deixa eu adiantar algumas coisas.
% 
% A minha "apostila", pra usar o termo da reclamação dos discentes
% incluída no memorando, não é uma apostila "tradicional" no sentido de
% uma versão escrita do que a gente escreve no quadro numa aula
% expositiva; ela é um material feito pra evitar que os alunos "PERCAM
% TEMPO" com coisas tipo ficar copiando o quadro perdidos pensando "PQP,
% vou tentar entender isso em casa depois", e ainda por cima muitas
% vezes copiando errado. Ela começa com dicas sobre como estudar e
% explicando que GA é um curso de escrita matemática, e ela está cheia
% de exercícios pros alunos - mesmo os sem base matemática quase nenhuma
% - poderem começar imediatamente, assim que eu distribuo cópias das
% folhas do dia no início de cada aula, a lerem o texto e os exemplos e
% começaram a fazer os exercícios DISCUTINDO ELES EM GRUPO - e tirando
% dúvidas primeiro entre si e depois comigo sempre que precisarem; ou
% seja, tem truques ali pra fazer eles APRENDEREM A ESTUDAR. Ah, e as
% minhas "apostilas" COMPLEMENTAM os livros-texto que a gente usa no
% curso e elas têm alguns "exercícios" que dizem coisas como "compare a
% nossa abordagem com a das páginas tais e tais do livro tal" (que os
% alunos têm em PDF)...
% 
% As minhas "apostilas" têm um bocado de material pra lidar com coisas
% cuja falta nos livros eu acho desesperadora... por exemplo - deixa eu
% descrever usando personagens - como lidar com alunos que desenham tudo
% torto? Que deixam pra estudar na última hora? Que não têm nenhuma
% referência de como fazer uma demonstração passo a passo? Que chegam
% até o meio do semestre achando que podem mostrar que uma proposição é
% verdadeira mostrando um caso particular? Que não fazem idéia de como
% fazer o gráfico da reta x=2 ou da "reta degenerada" 0x+0y=4? Que acham
% que basta mostrar o resultado de uma questão e não sabem escrever o
% desenvolvimento? O ideal seria que eles soubessem estimar o nível de
% cada questão e escrever tanto um desenvolvimento detalhado pra um
% "leitor burro" quanto um desenvolvimento pra um leitor mais avançado,
% né?
% 
% Eu tou produzindo esse material e mandando o link pra ele pra um monte
% de lugares e brigando por ele exatamente porque eu acredito que estou
% fazendo algo importante, e porque POR ENQUANTO as pessoas não têm
% tempo pra conversar sobre os seus métodos de, por exemplo, ensinarem
% alunos egressos do ensino médio de hoje em dia a representarem coisas
% graficamente, a traduzirem entre o geométrico e o algébrico, e a
% fazerem demonstrações...
% 
% Eu tou tentando deixar esse material cada vez mais visível porque POR
% ENQUANTO eu não tenho social skills pra conseguir mais do que poucas
% conversas de 10 minutos a cada semestre com as outras pessoas que dão
% GA e Cálculo 1 pra saber como elas estão ensinando isso.
% 
% Eu tou deixando esse material visível pra que daqui a 3 semanas, 3
% meses, ou 3 anos fique MUITO CLARO que a minha parte eu fiz e deixei
% do modo mais fácil de consultar possível, e pra que você - deixa eu te
% pegar como caso extremo =) - resolva que eu mereço 5 minutos da porra
% do seu tempo pra gente trocar figurinhas sobre nossos métodos
% didáticos, que certamente se complementam.


\newpage


%   ____      _                                 _ _
%  / ___|___ (_)___  __ _ ___   _ __ ___  _   _(_) |_ ___
% | |   / _ \| / __|/ _` / __| | '_ ` _ \| | | | | __/ _ \
% | |__| (_) | \__ \ (_| \__ \ | | | | | | |_| | | || (_) |
%  \____\___/|_|___/\__,_|___/ |_| |_| |_|\__,_|_|\__\___/
%
% «coisas-muito» (to ".coisas-muito")
% (gam181p 1 "coisas-muito")

% {\setlength{\parindent}{0em}
% \footnotesize
% \par Geometria Analítica - material para exercícios
% \par PURO-UFF - 2018.1 - Eduardo Ochs
% \par Links importantes:
% \par \url{http://angg.twu.net/2018.1-GA.html} (página do curso)
% \par \url{http://angg.twu.net/2018.1-GA/2018.1-GA.pdf} (quadros)
% \par \url{http://angg.twu.net/LATEX/2018-1-GA-material.pdf} (isto aqui)
% \par {\tt eduardoochs@gmail.com} (meu e-mail)
% \par Dá pra chegar na página do curso googlando por ``Eduardo Ochs'',
% \par indo pra qualquer subpágina do angg.twu.net, e clicando em ``GA''
% \par na barra de navegação à esquerda.
% 
% }
% 
% \bsk
% \bsk




{
\setlength{\parindent}{0em}

\mypsection {coisas-muito} {Coisas MUITO importantes sobre Geometria Analítica}
}

\ssk

A matéria é sobre duas linguagens diferentes: a
%
\begin{itemize}
\item ``Geometria'', que é sobre coisas gráficas como pontos, retas e
  círculos, e a

\item ``Analítica'', que é sobre ``álgebra'', sobre coisas matemáticas
  ``formais'' como contas, conjuntos e equações;

\end{itemize}
%
além disso Geometria Analítica é também sobre a TRADUÇÃO entre essas
duas linguagens.

\msk

Lembre que boa parte do que você aprendeu sobre álgebra no ensino
médio era sobre {\sl resolver equações}.

{\sl Encontrar soluções} de equações é difícil --- são muitos métodos,
e dá pra errar bastante no caminho --- mas {\sl testar} as soluções é
fácil.

\msk

Boa parte do que você aprendeu (ou deveria ter aprendido) sobre
geometria no ensino médio envolvia construções gráficas; por exemplo,
a partir de pontos $A$, $B$, $C$,

Seja $A'$ o ponto médio entre $B$ e $C$,

Seja $B'$ o ponto médio entre $A$ e $C$,

Seja $C'$ o ponto médio entre $A$ e $B$,

Seja $r_a$ a reta que passa por $A'$ e é ortogonal a $BC$,

Seja $r_b$ a reta que passa por $B'$ e é ortogonal a $AC$,

Seja $r_c$ a reta que passa por $C'$ e é ortogonal a $AB$,

Seja $D$ o ponto de interseção das retas $r_a$, $r_b$ e $r_c$,

então $D$ é o centro do círculo que passa por $A$, $B$ e $C$.

\msk

Você {\bf VAI TER QUE} aprender a definir seus objetos --- pontos, retas,
conjuntos, círculos, etc... isso provavelmente vai ser algo novo pra
você e é algo que precisa de MUITO treino. Dá pra passar em Cálculo 1
e em Prog 1 só aprendendo a ``ler'' as definições que o professor e os
livros mostram, mas em Geometria Analítica NÃO DÁ, em GA você vai ter
que aprender a ler {\bf E A ESCREVER} definições.




\newpage

%  ____  _
% |  _ \(_) ___ __ _ ___
% | | | | |/ __/ _` / __|
% | |_| | | (_| (_| \__ \
% |____/|_|\___\__,_|___/
%
% «dicas» (to ".dicas")
% (gam181p 2 "dicas")
\mypsection {dicas} {Dicas MUITO IMPORTANTES e pouco óbvias}

1) Aprenda a testar tudo: contas, possíveis soluções de equações,
representações gráficas de conjuntos...

2) Cada ``seja'' ou ``sejam'' que aparece nestas folhas é uma
definição, e você pode usá-los como exemplos de definições
bem-escritas (ééé!!!!) pra aprender jeitos de escrever as suas
definições.

3) Em ``matematiquês'' a gente quase não usa termos como ``ele'',
``ela'', ``isso'', ``aquilo'' e ''lá'' --- ao invés disso a gente dá
nomes curtos pros objetos ou usa expressões matemáticas pra eles cujo
resultado é o objeto que a gente quer (como nas pags
\pageref{comprehension-ex123} e \pageref{projecoes})... mas {\sl
  quando a gente está discutindo problemas no papel ou no quadro} a
gente pode ser referir a determinados objetos {\sl apontando pra eles
  com o dedo} e dizendo ``esse aqui''.

4) Se você estiver em dúvida sobre o que um problema quer dizer tente
escrever as suas várias hipóteses --- a prática de escrever as suas
idéias é o que vai te permitir aos poucos conseguir resolver coisas de
cabeça.

5) Muitas coisas aparecem nestas folhas escritas primeiro de um jeito
detalhado, e depois aos poucos de jeitos cada vez mais curtos. Você
vai ter que aprender a completar os detalhes.

6) Alguns exercícios destas folhas têm muitos subcasos. Nos primeiros
subcasos você provavelmente vai precisar fazer as contas com todos os
detalhes e verificá-las várias vezes pra não errar, depois você vai
aprender a fazê-las cada vez mais rápido, depois vai poder fazê-las de
cabeça, e depois você vai começar a visualizar o que as contas
``querem dizer'' e vai conseguir chegar ao resultado graficamente, sem
contas; e se você estiver em dúvida se o seu ``método gráfico'' está
certo você vai poder conferir se o ``método gráfico'' e o ``método
contas'' dão aos mesmos resultados. Exemplo:
p.\pageref{sistemas-de-coordenadas}.

7) Uma solução bem escrita pode incluir, além do resultado final,
contas, definições, representações gráficas, explicações em português,
testes, etc. Uma solução bem escrita é fácil de ler e fácil de
verificar. Você pode testar se uma solução sua está bem escrita
submetendo-a às seguinte pessoas: a) você mesmo logo depois de você
escrevê-la --- releia-a e veja se ela está clara; b) você mesmo, horas
depois ou no dia seguinte, quando você não lembrar mais do que você
pensava quando você a escreveu; c) um colega que seja seu amigo; d) um
colega que seja menos seu amigo que o outro; e) o monitor ou o
professor. Se as outras pessoas acharem que ler a sua solução é um
sofrimento, isso é mau sinal; se as outras pessoas acharem que a sua
solução está claríssima e que elas devem estudar com você, isso é bom
sinal. {\sl GA é um curso de escrita matemática:} se você estiver
estudando e descobrir que uma solução sua pode ser reescrita de um
jeito bem melhor, não hesite --- reescrever é um ótimo exercício.

% 8) Estas notas {\sl vão ser} uma versão ampliada e melhorada destas
% notas aqui, do semestre passado:
%
% \url{http://angg.twu.net/LATEX/2017-2-GA-material.pdf}




\newpage

%  ____        _         _   _ _         _                 
% / ___| _   _| |__  ___| |_(_) |_ _   _(_) ___ __ _  ___  
% \___ \| | | | '_ \/ __| __| | __| | | | |/ __/ _` |/ _ \ 
%  ___) | |_| | |_) \__ \ |_| | |_| |_| | | (_| (_| | (_) |
% |____/ \__,_|_.__/|___/\__|_|\__|\__,_|_|\___\__,_|\___/ 
%                                                          
% «substituicao» (to ".substituicao")
% (gam181p 3 "substituicao")
\mypsection {substituicao} {Substituição}

Uma das coisas que vamos usar neste curso e que não costuma ser
apresentada em livros básicos --- mas que eu uso na optativa de
``Lógica pra Crianças'' --- é uma operação chamada {\sl substituição
  simultânea}. Exemplo:
%
$$((x+y)·z) \subst{ 
    x:=a+y \\
    y:=b+z \\
    z:=c+x \\
  }
  \;\;=\;\;
  ((a+y)+(b+z))·(c+x).
$$

Essa operação {\sl pode} ser aplicada em expressões que não fazem
sentido nenhum --- por exemplo:
%
$$\def\newa{\psm{∫⊙\\◻}}
  (\text{Vanessão 20 reais})
  \bmat{\text{a} := \newa}
  \;\;=\;\;
  (\text{V$\newa$ness$\widetilde{\newa}$o 20 re$\newa$is})
$$
%
e às vezes vamos usá-la para atribuir sentido para expresões
aparentemente abstratas. Por exemplo, na parte sobre sistemas de
coordenadas vamos ter definições como
%
$$
(a,b)_Σ = (10a+2,100b+3) \qquad \text{para $a,b∈\R$}
$$
%
que nos permite fazer
$$
\begin{array}{l}
  ((a,b)_Σ = (10a+2,100b+3))\subst{a:=4\\b:==5} \\
  = ((4,5)_Σ = (10·4+2,100·5+3)) \\
  = ((4,5)_Σ = (42,503)) \\
\end{array}
$$
%
e fazendo isto pra vários valores de $a$ e $b$ a gente consegue montar
uma tabela {\sl (ainda não fiz)} e entender geometricamente como a
operação $(a,b)_Σ$ funciona.

A substituição também serve pra gente testar equações:
%
$$
\begin{array}{rcl}
  (x^2-5x+6=0) \bmat{x:=1} &=& (1^2-5·1+6=0) \\
                           &=& (1-5+6=0) \\
                           &=& (2=0) \\
                           &=& 𝐛F \\
  (x^2-5x+6=0) \bmat{x:=2} &=& (2^2-5·2+6=0) \\
                           &=& (4-10+6=0) \\
                           &=& (0=0) \\
                           &=& 𝐛V \\
\end{array}
$$
%
e as ``set comprehensions'' das seções seguintes vão nos permitir
escrever o conjunto das soluções de uma equação de um jeito claro,
rápido, preciso e fácil de debugar {\sl sem usar português:}
%
$$
  \setofst{x∈\R}{x^2-5x+6=0} = \{2,3\}
$$

{\sl Aos poucos} a gente vai começar a usar substituições mais
complicadas usando implicitamente a idéia de ``tipos'' das próximas
páginas. Por exemplo, se $t∈\R$ esta substituição é válida,
%
$$
\begin{array}{l}
  ((a,b)_Σ = (10a+2,100b+3))\subst{a:=5t\\b:=6t} \\
  = ((5t,6t)_Σ = (10·5t+2,100·6t+3)) \\
\end{array}
$$
%
mas não é válido substituir $a$ ou $b$ por uma expressão cujo
resultado seja uma matriz.








\newpage






%  __  __       _        _
% |  \/  | __ _| |_ _ __(_)_______  ___
% | |\/| |/ _` | __| '__| |_  / _ \/ __|
% | |  | | (_| | |_| |  | |/ /  __/\__ \
% |_|  |_|\__,_|\__|_|  |_/___\___||___/
%
% «matrizes» (to ".matrizes")
% (gam181p 4 "matrizes")
\mypsection {matrizes} {Alguns ``tipos'' de objetos matemáticos familiares}

Multiplicação de matrizes:

\def\und#1#2{\underbrace{#1}_{#2}}

$\und{\pmat{1 & 2 & 3 \\ 4 & 5 & 6 \\ 7 & 8 & 9}}{3×3}
 \und{\pmat{1000 \\ 100 \\ 10}}{3×1}
 = \und{\pmat{1230 \\ 4560 \\ 7890}}{3×1}
$

$\und{\pmat{a & b \\ c & d \\ e & f}}{3×2}
 \und{\pmat{g & h & i & j \\ k & l & m & n}}{2×4}
 = \und{\pmat{ag+bk & ah+bl & ai+bm & aj+bn \\
              cg+dk & ch+dl & ci+dm & cj+dn \\
              eg+fk & eh+fl & ei+fm & ej+fn \\}}{3×4}
$

$\und{\pmat{g & h & i & j \\ k & l & m & n}}{2×4}
 \und{\pmat{a & b \\ c & d \\ e & f}}{3×2}
 = \; \text{erro \qquad (porque $4≠3$)}
$

$\pmat{1 & 2 \\ 3 & 4} \pmat{100 & 0 \\ 10 & 0} = \pmat{120 & 0 \\ 340 & 0}$

\ssk

$\pmat{100 & 0 \\ 10 & 0} \pmat{1 & 2 \\ 3 & 4} = \pmat{100 & 200 \\ 10 & 20}$

\ssk

$\pmat{2 \\ 3 \\ 4}^T \pmat{100 \\ 10 \\ 1} = \pmat{2 & 3 & 4} \pmat{100 \\ 10 \\ 1} = (234) = 234$

\bsk

Soma de matrizes:

$\pmat{10 & 20 & 30 \\ 40 & 50 & 60} + \pmat{2 & 3 & 4 \\ 5 & 6 & 7} = \pmat{12 & 23 & 34 \\ 45 & 56 & 67}$

$\pmat{10 & 20 & 30 \\ 40 & 50 & 60} + \pmat{2 & 3 \\ 5 & 6 } = \; \text{erro}$

\bsk

Multiplicação de número por matriz:

$10 \pmat{2 & 3 & 4 \\ 5 & 6 & 7} = \pmat{20 & 30 & 40 \\ 50 & 60 & 70}$



\bsk

\def\V{\mathbf{V}}
\def\F{\mathbf{F}}

Operações lógicas:

\ssk

$\begin{array}[t]{rcl}
 \text{``E'':} \\
 \F\&\F &=& \F \\
 \F\&\V &=& \F \\
 \V\&\F &=& \F \\
 \V\&\V &=& \V \\
 \end{array}
 %
 \quad
 %
 \begin{array}[t]{rcl}
 \text{``Ou'':} \\
 \F∨\F &=& \F \\
 \F∨\V &=& \V \\
 \V∨\F &=& \V \\
 \V∨\V &=& \V \\
 \end{array}
 %
 \quad
 %
 \begin{array}[t]{rcl}
 \text{``Implica'':}\hss \\
 \F→\F &=& \V \\
 \F→\V &=& \V \\
 \V→\F &=& \F \\
 \V→\V &=& \V \\
 \end{array}
 %
 \quad
 %
 \begin{array}[t]{rcl}
 \text{``Não'':} \\
 ¬\F &=& \V \\
 ¬\V &=& \F \\
 \end{array}
$

\bsk

Se $x=6$,

$\und{\und{2<\und{x}{6}}{\V} \&
      \und{\und{x}{6}<5}{\F}
     }{\F}
$




\newpage

%   ____                               _                    _
%  / ___|___  _ __ ___  _ __  _ __ ___| |__   ___ _ __  ___(_) ___  _ __
% | |   / _ \| '_ ` _ \| '_ \| '__/ _ \ '_ \ / _ \ '_ \/ __| |/ _ \| '_ \
% | |__| (_) | | | | | | |_) | | |  __/ | | |  __/ | | \__ \ | (_) | | | |
%  \____\___/|_| |_| |_| .__/|_|  \___|_| |_|\___|_| |_|___/_|\___/|_| |_|
%                      |_|
%
% «comprehension» (to ".comprehension")
% (gam181p 5 "comprehension")
\mypsection {comprehension} {``Set comprehensions''}

\def\und#1#2{\underbrace{#1}_{#2}}
\def\und#1#2{\underbrace{#1}_{\text{#2}}}
\def\ug#1{\und{#1}{ger}}
\def\uf#1{\und{#1}{filt}}
\def\ue#1{\und{#1}{expr}}

Notação explícita, com geradores, filtros,

e um ``;'' separando os geradores e filtros da expressão final:

$\begin{array}{lll}
\{\ug{a∈\{1,2,3,4\}}; \ue{10a}\}     &=& \{10,20,30,40\} \\
\{\ug{a∈\{1,2,3,4\}}; \ue{a}\}       &=& \{1,2,3,4\} \\
\{\ug{a∈\{1,2,3,4\}}, \uf{a≥3}; \ue{a}\} &=& \{3,4\} \\
\{\ug{a∈\{1,2,3,4\}}, \uf{a≥3}; \ue{10a}\} &=& \{30,40\} \\
\{\ug{a∈\{10,20\}}, \ug{b∈\{3,4\}}; \ue{a+b}\} &=& \{13,14,23,24\} \\
\{\ug{a∈\{1,2\}}, \ug{b∈\{3,4\}}; \ue{(a,b)}\} &=& \{(1,3),(1,4),(2,3),(2,4)\} \\
\end{array}
$





% (setq last-kbd-macro (kbd "C-w \\ uf{ C-y }"))
% (setq last-kbd-macro (kbd "C-w \\ ue{ C-y }"))

\msk
\msk

Notações convencionais, com ``$|$'' ao invés de ``;'':

Primeiro tipo --- expressão final, ``$|$'', geradores e filtros:

$\begin{array}{lll}
\setofst{10a}{a∈\{1,2,3,4\}} &=&
  \{\ug{a∈\{1,2,3,4\}}; \ue{10a}\} \\
\setofst{10a}{a∈\{1,2,3,4\}, a≥3} &=&
  \{\ug{a∈\{1,2,3,4\}}, \uf{a≥3}; \ue{10a}\} \\
\setofst{a}{a∈\{1,2,3,4\}} &=&
  \{\ug{a∈\{1,2,3,4\}}; \ue{a}\} \\
% \{\ug{a∈\{1,2\}}, \ug{b∈\{3,4\}}; \ue{(a,b)}\} \\
\end{array}
$

\msk

O segundo tipo --- gerador, ``$|$'', filtros ---

pode ser convertido para o primeiro...

o truque é fazer a expressão final ser a variável do gerador:

$\begin{array}{lll}
\setofst{a∈\{1,2,3,4\}}{a≥3} &=& \\
\setofst{a}{a∈\{1,2,3,4\}, a≥3} &=&
  \{\ug{a∈\{1,2,3,4\}}, \uf{a≥3}; \ue{a}\} \\
% \{\ug{a∈\{10,20\}}, \ug{b∈\{3,4\}}; \ue{a+b}\} \\
\end{array}
$

\msk

O que distingue as duas notacões ``$\{\ldots|\ldots\}$'' é

se o que vem antes da ``$|$'' é ou não um gerador.

\bsk

Observações:

$\setofst{\text{gerador}}{\text{filtros}} =
 \{\text{gerador},\text{filtros};\ue{\text{variável do gerador}}\}$

$\setofst{\text{expr}}{\text{geradores e filtros}} =
 \{\text{geradores e filtros}; \text{expr}\}
$

\msk

As notações ``$\{\ldots|\ldots\}$'' são padrão e são usadas em muitos livros de matemática.

A notação ``$\{\ldots;\ldots\}$'' é bem rara; eu aprendi ela em
artigos sobre linguagens de programação, e resolvi apresentar ela aqui
porque acho que ela ajuda a explicar as duas notações
``$\{\ldots|\ldots\}$''.


\newpage

%                                     _                    _               _____
%   ___ ___  _ __ ___  _ __  _ __ ___| |__   ___ _ __  ___(_) ___  _ __   |_   _|
%  / __/ _ \| '_ ` _ \| '_ \| '__/ _ \ '_ \ / _ \ '_ \/ __| |/ _ \| '_ \    | |
% | (_| (_) | | | | | | |_) | | |  __/ | | |  __/ | | \__ \ | (_) | | | |   | |
%  \___\___/|_| |_| |_| .__/|_|  \___|_| |_|\___|_| |_|___/_|\___/|_| |_|   |_|
%                     |_|
%
% «comprehension-tables» (to ".comprehension-tables")
% (gam181p 6 "comprehension-tables")
\mypsection {comprehension-tables} {``Set comprehensions'': como calcular usando tabelas}

\def\tbl#1#2{\fbox{$\begin{array}{#1}#2\end{array}$}}
\def\tbl#1#2{\fbox{$\sm{#2}$}}
\def\V{\mathbf{V}}
\def\F{\mathbf{F}}

% "Stop":
% (find-es "tex" "vrule")
\def\S{\omit$|$\hss}
\def\S{\omit\vrule\hss}
\def\S{\omit\vrule$($\hss}
\def\S{\omit\vrule$\scriptstyle($\hss}
\def\S{\omit\vrule\phantom{$\scriptstyle($}\hss}   % stop

Alguns exemplos:

\msk

\def\s{\mathstrut}
\def\s{\phantom{$|$}}
\def\s{\phantom{|}}
\def\s{}

Se $A := \{x∈\{1,2\}; (x,3-x)\}$

então $A = \{(1,2), (2,1)\}$:

\tbl{ccc}{
 \s x & (x,3-x) \\\hline
 \s 1 & (1,2) \\
 \s 2 & (2,1) \\
}

\msk

Se $I := \{x∈\{1,2,3\}, y∈\{3,4\}, x+y<6; (x,y)\}$

então $I = \{(1,3),(1,4),(1,5)\}$:

\tbl{ccc}{
 \s x & y & x+y<6 & (x,y) \\\hline
 \s 1 & 3 & \V & (1,3) \\
 \s 1 & 4 & \V & (1,4) \\
 \s 2 & 3 & \V & (2,3) \\
 \s 2 & 4 & \F & \S \\
 \s 3 & 3 & \F & \S \\
 \s 3 & 4 & \F & \S \\
}

\msk

Se $D := \setofst{(x,2x)}{x∈\{0,1,2,3\}}$

então $D = \{x∈\{0,1,2,3\}; (x,2x)\}$,

$D = \{(0,0), (1,2), (2,4), (3,6)\}$:

\tbl{ccc}{
 \s x & (x,2x) \\\hline
 \s 0 & (0,0) \\
 \s 1 & (1,2) \\
 \s 2 & (2,4) \\
 \s 3 & (3,6) \\
}

\msk

Se $P := \setofst {(x,y)∈\{1,2,3\}^2} {x≥y}$

então $P = \{(x,y)∈\{1,2,3\}^2, x≥y; (x,y)\}$,

$P = \{(1,1), (2,1), (2,2), (3,1), (3,2), (3,3)\}$:

\tbl{ccc}{
 \s (x,y) & x & y & x≥y & (x,y) \\\hline
 \s (1,1) & 1 & 1 & \V & (1,1) \\
 \s (1,2) & 1 & 2 & \F & \S    \\
 \s (1,3) & 1 & 3 & \F & \S    \\
 \s (2,1) & 2 & 1 & \V & (2,1) \\
 \s (2,2) & 2 & 2 & \V & (2,2) \\
 \s (2,3) & 2 & 3 & \F & \S    \\
 \s (3,1) & 3 & 1 & \V & (3,1) \\
 \s (3,2) & 3 & 2 & \V & (3,2) \\
 \s (3,3) & 3 & 3 & \V & (3,3) \\
}

\bsk

Obs: os exemplos acima correspondem aos

exercícios 2A, 2I, 3D e 5P das próximas páginas.




\newpage

%  _____                   _      _
% | ____|_  _____ _ __ ___(_) ___(_) ___  ___
% |  _| \ \/ / _ \ '__/ __| |/ __| |/ _ \/ __|
% | |___ >  <  __/ | | (__| | (__| | (_) \__ \
% |_____/_/\_\___|_|  \___|_|\___|_|\___/|___/
%
% «comprehension-ex123» (to ".comprehension-ex123")
% (gam181p 7 "comprehension-ex123")
\mypsection {comprehension-ex123} {Exercícios de ``set comprehensions''}

1) Represente graficamente:

$\begin{array}{rcl}
 A & := & \{(1,4), (2,4), (1,3)\} \\
 B & := & \{(1,3), (1,4), (2,4)\} \\
 C & := & \{(1,3), (1,4), (2,4), (2,4)\} \\
 D & := & \{(1,3), (1,4), (2,3), (2,4)\} \\
 E & := & \{(0,3), (1,2), (2,1), (3,0)\} \\
\end{array}
$

\msk

2) Calcule e represente graficamente:

$\begin{array}{rcl}
 A & := & \{x∈\{1,2\}; (x,3-x)\} \\
 B & := & \{x∈\{1,2,3\}; (x,3-x)\} \\
 C & := & \{x∈\{0,1,2,3\}; (x,3-x)\} \\
 D & := & \{x∈\{0,0.5,1, \ldots, 3\}; (x,3-x)\} \\
 E & := & \{x∈\{1,2,3\}, y∈\{3,4\}; (x,y)\} \\
 F & := & \{x∈\{3,4\}, y∈\{1,2,3\}; (x,y)\} \\
 G & := & \{x∈\{3,4\}, y∈\{1,2,3\}; (y,x)\} \\
 H & := & \{x∈\{3,4\}, y∈\{1,2,3\}; (x,2)\} \\
 I & := & \{x∈\{1,2,3\}, y∈\{3,4\}, x+y<6; (x,y)\} \\
 J & := & \{x∈\{1,2,3\}, y∈\{3,4\}, x+y>4; (x,y)\} \\
 K & := & \{x∈\{1,2,3,4\}, y∈\{1,2,3,4\}; (x,y)\} \\
 L & := & \{x,y∈\{0,1,2,3,4\}; (x,y)\} \\
 M & := & \{x,y∈\{0,1,2,3,4\}, y=3; (x,y)\} \\
 N & := & \{x,y∈\{0,1,2,3,4\}, x=2; (x,y)\} \\
 O & := & \{x,y∈\{0,1,2,3,4\}, x+y=3; (x,y)\} \\
 P & := & \{x,y∈\{0,1,2,3,4\}, y=x; (x,y)\} \\
 Q & := & \{x,y∈\{0,1,2,3,4\}, y=x+1; (x,y)\} \\
 R & := & \{x,y∈\{0,1,2,3,4\}, y=2x; (x,y)\} \\
 S & := & \{x,y∈\{0,1,2,3,4\}, y=2x+1; (x,y)\} \\
\end{array}
$

\msk

3) Calcule e represente graficamente:

$\begin{array}{rcl}
 A & := & \setofst{(x,0)}{x∈\{0,1,2,3\}} \\
 B & := & \setofst{(x,x/2)}{x∈\{0,1,2,3\}} \\
 C & := & \setofst{(x,x)}{x∈\{0,1,2,3\}} \\
 D & := & \setofst{(x,2x)}{x∈\{0,1,2,3\}} \\
 E & := & \setofst{(x,1)}{x∈\{0,1,2,3\}} \\
 F & := & \setofst{(x,1+x/2)}{x∈\{0,1,2,3\}} \\
 G & := & \setofst{(x,1+x)}{x∈\{0,1,2,3\}} \\
 H & := & \setofst{(x,1+2x)}{x∈\{0,1,2,3\}} \\
 I & := & \setofst{(x,2)}{x∈\{0,1,2,3\}} \\
 J & := & \setofst{(x,2+x/2)}{x∈\{0,1,2,3\}} \\
 K & := & \setofst{(x,2+x)}{x∈\{0,1,2,3\}} \\
 L & := & \setofst{(x,2+2x)}{x∈\{0,1,2,3\}} \\
 M & := & \setofst{(x,2)}{x∈\{0,1,2,3\}} \\
 N & := & \setofst{(x,2-x/2)}{x∈\{0,1,2,3\}} \\
 O & := & \setofst{(x,2-x)}{x∈\{0,1,2,3\}} \\
 P & := & \setofst{(x,2-2x)}{x∈\{0,1,2,3\}} \\
\end{array}
$


\newpage

%  ____                _                  _
% |  _ \ _ __ ___   __| |   ___ __ _ _ __| |_
% | |_) | '__/ _ \ / _` |  / __/ _` | '__| __|
% |  __/| | | (_) | (_| | | (_| (_| | |  | |_
% |_|   |_|  \___/ \__,_|  \___\__,_|_|   \__|
%
% «comprehension-prod» (to ".comprehension-prod")
% (gam181p 8 "comprehension-prod")
\mypsection {comprehension-prod} {Produto cartesiano de conjuntos}

$A×B:=\{a∈A,b∈B;(a,b)\}$

Exemplo: $\{1,2\}×\{3,4\} = \{(1,3),(1,4),(2,3),(2,4)\}$.

\ssk

Uma notação: $A^2 = A×A$.

Exemplo: $\{3,4\}^2 = \{3,4\}×\{3,4\} = \{(3,3),(3,4),(4,3),(4,4)\}$.

\msk

Sejam:

$A = \{1,2,4\}$,

$B = \{2,3\}$,

$C = \{2,3,4\}$.


\msk

{\bf Exercícios}

\ssk

4) Calcule e represente graficamente:

\begin{tabular}{lll}
a) $A×A$ & d) $B×A$ & g) $C×A$ \\
b) $A×B$ & e) $B×B$ & h) $C×B$ \\
c) $A×C$ & f) $B×C$ & i) $C×C$ \\
\end{tabular}

\msk

5) Calcule e represente graficamente:

$\begin{array}{rcl}
 A &:=& \{x,y∈\{0,1,2,3\};(x,y)\} \\
 B &:=& \{x,y∈\{0,1,2,3\}, y=2; (x,y)\} \\
 C &:=& \{x,y∈\{0,1,2,3\}, x=1; (x,y)\} \\
 D &:=& \{x,y∈\{0,1,2,3\}, y=x; (x,y)\} \\
 E &:=& \{x,y∈\{0,1,2,3,4\}, y=2x; (x,y)\} \\
 F &:=& \{(x,y)∈\{0,1,2,3,4\}^2, y=2x; (x,y)\} \\
 G &:=& \{(x,y)∈\{0,1,2,3,4\}^2, y=x; (x,y)\} \\
 H &:=& \{(x,y)∈\{0,1,2,3,4\}^2, y=x/2; (x,y)\} \\
 I &:=& \{(x,y)∈\{0,1,2,3,4\}^2, y=x/2+1; (x,y)\} \\
 J &:=& \setofst {(x,y)∈\{0,1,2,3,4\}^2} {y=2x} \\
 K &:=& \setofst {(x,y)∈\{0,1,2,3,4\}^2} {y=x} \\
 L &:=& \setofst {(x,y)∈\{0,1,2,3,4\}^2} {y=x/2} \\
 M &:=& \setofst {(x,y)∈\{0,1,2,3,4\}^2} {y=x/2+1} \\
 N &:=& \setofst {(x,y)∈\{1,2,3\}^2} {0x+0y=0} \\
 O &:=& \setofst {(x,y)∈\{1,2,3\}^2} {0x+0y=2} \\
 P &:=& \setofst {(x,y)∈\{1,2,3\}^2} {x≥y} \\
 \end{array}
$

\msk

6) Represente graficamente:

$\begin{array}{rcl}
 J' &:=& \setofst {(x,y)∈\R^2} {y=2x} \\
 K' &:=& \setofst {(x,y)∈\R^2} {y=x} \\
 L' &:=& \setofst {(x,y)∈\R^2} {y=x/2} \\
 M' &:=& \setofst {(x,y)∈\R^2} {y=x/2+1} \\
 N' &:=& \setofst {(x,y)∈\R^2} {0x+0y=0} \\
 O' &:=& \setofst {(x,y)∈\R^2} {0x+0y=2} \\
 P' &:=& \setofst {(x,y)∈\R^2} {x≥y} \\
 \end{array}
$





\newpage

%   ____       _                _ _
%  / ___| __ _| |__   __ _ _ __(_) |_ ___
% | |  _ / _` | '_ \ / _` | '__| | __/ _ \
% | |_| | (_| | |_) | (_| | |  | | || (_) |
%  \____|\__,_|_.__/ \__,_|_|  |_|\__\___/
%
% «comprehension-gab» (to ".comprehension-gab")
% (gam181p 9 "comprehension-gab")
% (to "picturedots")
\mypsection {comprehension-gab} {Gabarito dos exercícios de set comprehensions}

% \bhbox{$\picturedots(-1,-2)(5,5){ 3,1 3,2 3,3 }$}

1)
$
A = B = C = \picturedots(0,0)(3,4){ 1,4 2,4 1,3 }
\quad
D = \picturedots(0,0)(3,4){ 1,4 2,4 1,3 2,3 }
\quad
E = \picturedots(0,0)(4,4){ 0,3 1,2 2,1 3,0 }
$

\bsk

2)
$     A = \picturedots(0,0)(4,4){     1,2 2,1     }
\quad B = \picturedots(0,0)(4,4){     1,2 2,1 3,0 }
\quad C = \picturedots(0,0)(4,4){ 0,3 1,2 2,1 3,0 }
\quad D = \picturedots(0,0)(4,4){ 0,3 .5,2.5 1,2 1.5,1.5 2,1 2.5,.5 3,0 }
$

\msk

$
\quad E = \picturedots(0,0)(4,4){ 1,3 2,3 3,3   1,4 2,4 3,4 }
\quad F = \picturedots(0,0)(4,4){ 3,1 4,1   3,2 4,2   3,3 4,3 }
\quad G = \picturedots(0,0)(4,4){ 1,3 2,3 3,3   1,4 2,4 3,4 }
\quad H = \picturedots(0,0)(4,4){ 3,2 4,2 }
\quad I = \picturedots(0,0)(4,4){ 1,3 2,3       1,4         }
\quad J = \picturedots(0,0)(4,4){     2,3 3,3   1,4 2,4 3,4 }
$

\msk

$
\quad K = \picturedots(0,0)(4,4){     1,4 2,4 3,4 4,4
                                      1,3 2,3 3,3 4,3
                                      1,2 2,2 3,2 4,2
                                      1,1 2,1 3,1 4,1 }
\quad L = \picturedots(0,0)(4,4){ 0,4 1,4 2,4 3,4 4,4
                                  0,3 1,3 2,3 3,3 4,3
                                  0,2 1,2 2,2 3,2 4,2
                                  0,1 1,1 2,1 3,1 4,1
                                  0,0 1,0 2,0 3,0 4,0 }
\quad M = \picturedots(0,0)(4,4){ 0,3 1,3 2,3 3,3 4,3 }
\quad N = \picturedots(0,0)(4,4){ 2,0 2,1 2,2 2,3 2,4 }
\quad O = \picturedots(0,0)(4,4){ 0,3 1,2 2,1 3,0 }
\quad P = \picturedots(0,0)(4,4){ 0,0 1,1 2,2 3,3 4,4 }
$

\msk

$
\quad Q = \picturedots(0,0)(4,4){ 0,1 1,2 2,3 3,4 }
\quad R = \picturedots(0,0)(4,4){ 0,0 1,2 2,4 }
\quad S = \picturedots(0,0)(4,4){ 0,1 1,3 }
$

\bsk

3)
$     A = \picturedots(0,0)(4,4){ 0,0 1,0  2,0 3,0   }
\quad B = \picturedots(0,0)(4,4){ 0,0 1,.5 2,1 3,1.5 }
\quad C = \picturedots(0,0)(4,4){ 0,0 1,1  2,2 3,3   }
\quad D = \picturedots(0,0)(4,7){ 0,0 1,2  2,4 3,6   }
$

$
\quad E = \picturedots(0,0)(4,4){ 0,1 1,1   2,1 3,1   }
\quad F = \picturedots(0,0)(4,4){ 0,1 1,1.5 2,2 3,2.5 }
\quad G = \picturedots(0,0)(4,4){ 0,1 1,2   2,3 3,4   }
\quad H = \picturedots(0,0)(4,7){ 0,1 1,3   2,5 3,7   }
$

$
\quad I = \picturedots(0,0)(4,4){ 0,2 1,2   2,2 3,2   }
\quad J = \picturedots(0,0)(4,4){ 0,2 1,2.5 2,3 3,3.5 }
\quad K = \picturedots(0,0)(4,4){ 0,2 1,3   2,4 3,5   }
\quad L = \picturedots(0,0)(4,8){ 0,2 1,4   2,6 3,8   }
$

$
\quad M = \picturedots(0,0)(4,4){ 0,2 1,2   2,2 3,2   }
\quad N = \picturedots(0,0)(4,4){ 0,2 1,1.5 2,1 3,.5 }
\quad O = \picturedots(0,-1)(4,4){ 0,2 1,1   2,0 3,-1  }
\quad P = \picturedots(0,-5)(4,3){ 0,2 1,0   2,-2 3,-4   }
$

\bsk

4)
$     A×A = \picturedots(0,0)(4,4){ 1,1 2,1 4,1   1,2 2,2 4,2   1,4 2,4 4,4 }
\quad B×A = \picturedots(0,0)(4,4){ 2,1 3,1       2,2 3,2       2,4 3,4     }
\quad C×A = \picturedots(0,0)(4,4){ 2,1 3,1 4,1   2,2 3,2 4,2   2,4 3,4 4,4 }
$

\msk

$
\quad A×B = \picturedots(0,0)(4,4){ 1,2 2,2 4,2   1,3 2,3 4,3 }
\quad B×B = \picturedots(0,0)(4,4){ 2,2 3,2       2,3 3,3     }
\quad C×B = \picturedots(0,0)(4,4){ 2,2 3,2 4,2   2,3 3,3 4,3 }
$

\msk

$
\quad A×C = \picturedots(0,0)(4,4){ 1,2 2,2 4,2   1,3 2,3 4,3   1,4 2,4 4,4 }
\quad B×C = \picturedots(0,0)(4,4){ 2,2 3,2       2,3 3,3       2,4 3,4     }
\quad C×C = \picturedots(0,0)(4,4){ 2,2 3,2 4,2   2,3 3,3 4,3   2,4 3,4 4,4 }
$

\bsk

5)
$     A = \picturedots(0,0)(4,4){ 0,3 1,3 2,3 3,3
                                  0,2 1,2 2,2 3,2
                                  0,1 1,1 2,1 3,1
                                  0,0 1,0 2,0 3,0 }
\quad B = \picturedots(0,0)(4,4){ 0,2 1,2 2,2 3,2 }
\quad C = \picturedots(0,0)(4,4){ 1,0 1,1 1,2 1,3 }
\quad D = \picturedots(0,0)(4,4){ 0,0 1,1 2,2 3,3 }
\quad E = \picturedots(0,0)(4,4){ 0,0 1,2 2,4 }
$

\msk

$
\quad F = \picturedots(0,0)(4,4){ 0,0 1,2 2,4         }
\quad G = \picturedots(0,0)(4,4){ 0,0 1,1 2,2 3,3 4,4 }
\quad H = \picturedots(0,0)(4,4){ 0,0 2,1 4,2         }
\quad I = \picturedots(0,0)(4,4){ 0,1 2,2 4,3         }
$

\msk

$
\quad J = \picturedots(0,0)(4,4){ 0,0 1,2 2,4         }
\quad K = \picturedots(0,0)(4,4){ 0,0 1,1 2,2 3,3 4,4 }
\quad L = \picturedots(0,0)(4,4){ 0,0 2,1 4,2         }
\quad M = \picturedots(0,0)(4,4){ 0,1 2,2 4,3         }
$

\msk

$
\quad N = \picturedots(0,0)(4,4){ 1,3 2,3 3,3
                                  1,2 2,2 3,2
                                  1,1 2,1 3,1 }
\quad O = \picturedots(0,0)(4,4){             }
\quad P = \picturedots(0,0)(4,4){         3,3
                                      2,2 3,2
                                  1,1 2,1 3,1 }
$


\newpage

%           _            
%  _ __ ___| |_ __ _ ___ 
% | '__/ _ \ __/ _` / __|
% | | |  __/ || (_| \__ \
% |_|  \___|\__\__,_|___/
%                        
% «retas» (to ".retas")
% (gam181p 10 "retas")
\mypsection {retas} {Retas}

Sejam:

$\begin{array}{rclcrcl}
 R_{a,b}   &=& \setofst{(x,y)∈\{0,1,2,3,4,5\}^2}{y=ax+b}  \\
 r_{a,b}   &=& \setofst{(x,y)∈\R^2}{y=ax+b} \\
 R_{a,b,c} &=& \setofst{(x,y)∈\{0,1,2,3,4,5\}^2}{ax+by=c} \\
 r_{a,b,c} &=& \setofst{(x,y)∈\R^2}{ax+by=c} \\
 \end{array}
$

\ssk

{\bf Exercícios:}

1) Represente graficamente:

a) $R_{0,0}, R_{1,0}, R_{2,0}$.

b) $R_{0,1}, R_{1,1}, R_{2,1}$.

c) $R_{0,2}, R_{1,2}, R_{2,2}$.

d) $r_{0,0}, r_{1,0}, r_{2,0}$.

e) $r_{0,1}, r_{1,1}, r_{2,1}$.

f) $r_{0,2}, r_{1,2}, r_{2,2}$.

\ssk

Dicas:

Todo conjunto da forma $r_{a,b}$ para $a,b∈\R$ é uma reta.

Se você comparar os resultados dos exercícios acima você vai conseguir

entender --- ou pelo menos fazer hipóteses sobre --- o que ``querem
dizer''

o $a$ e o $b$ em $r_{a,b}$.

Se você souber dois pontos de uma reta $r$ você consegue traçá-la.

\ssk

{\bf Mais exercícios:}

2) Represente graficamente:

a) $R_{1,1,1}$, $R_{1,1,2}$, $R_{1,1,3}$.

b) $r_{1,1,1}$, $r_{1,1,2}$, $r_{1,1,3}$.

c) $R_{2,3,6}$, $r_{2,3,6}$. 

d) $r_{1/2,0}$, $r_{1/2,1}$, $r_{1/2,2}$.

e) $r_{-1/2,0}$, $r_{-1/2,1}$, $r_{-1/2,2}$.

f) $R_{2,2,2}$, $R_{2,2,4}$, $R_{2,2,6}$.

g) $r_{2,2,2}$, $r_{2,2,4}$, $r_{2,2,6}$.

\msk

Mais dicas:

Dois conjuntos são diferentes se existe algum ponto que pertence a um

e não pertence a outro --- por exemplo, $(2,4)∈r_{2,0}$ e
$(2,4)\not∈r(3,0)$,

portanto $r_{2,0} ≠ r_{3,0}$.

Duas retas são iguais se existem dois pontos diferentes que pertencem

a ambas.

\msk

{\bf Mais exercícios:}

3) Encontre $a,b∈\R$ tais que $r_{a,b}=r_{1,3,3}$. Dica: chutar e testar.

4) Encontre $a,b∈\R$ tais que $r_{a,b}=r_{2,-1,4}$. Dica: chutar e testar.

5) Represente graficamente $R_{0,0,1}$ e $R_{0,0,0}$. Dica: item 7 da p.2.

6) Represente graficamente $r_{0,0,1}$ e $r_{0,0,0}$. Dica: item 7 da p.2.



\newpage

%                    _                               _
%  _ __   ___  _ __ | |_ ___  ___    ___  __   _____| |_ ___  _ __ ___  ___
% | '_ \ / _ \| '_ \| __/ _ \/ __|  / _ \ \ \ / / _ \ __/ _ \| '__/ _ \/ __|
% | |_) | (_) | | | | || (_) \__ \ |  __/  \ V /  __/ || (_) | | |  __/\__ \
% | .__/ \___/|_| |_|\__\___/|___/  \___|   \_/ \___|\__\___/|_|  \___||___/
% |_|
%
% «pontos-e-vetores» (to ".pontos-e-vetores")
% (gam181p 11 "pontos-e-vetores")
\mypsection {pontos-e-vetores} {Pontos e vetores}

Se $a,b,c$ são números então

$(a,b)$ é um ponto de $\R^2$,

$\VEC{a,b}$ é um vetor em $\R^2$,

$(a,b,c)$ é um ponto de $\R^3$,

$\VEC{a,b,c}$ é um vetor em $\R^3$.

\msk

Por enquanto nós só vamos usar $\R^2$ --

a {\sl terceira parte do curso} vai ser sobre $\R^3$.

\msk

Podemos pensar que a {\sl operação} $(\_,\_)$ recebe dois números e
``monta'' um ponto de $\R^2$ com eles; a operação $\VEC{\_,\_}$ é
similar, mas ela monta um vetor. Também temos operações , $\__1$,
$\__2$ que ``desmontam'' pontos e vetores e retornam a primeira ou a
segunda componente deles: $(3,4)_1=3$, $(3,4)_2=4$, $\VEC{3,4}_1=3$,
$\VEC{3,4}_2=4$. Se $\vv=\VEC{4,5}$ então $\vv=\VEC{\vv_1,\vv_2}$.

\msk

{\bf Operações com pontos e vetores} (obs: $a,b,c,d,k∈\R$):

\ssk

% (gaq161 1)

1) $(a,b) + \VEC{c,d} = (a+c,b+d)$

2) $\VEC{a,b} + \VEC{c,d} = \VEC{a+c,b+d}$

3) $(a,b) - (c,d) = \VEC{a-c,b-d}$

4) $(a,b) - \VEC{c,d} = (a-c,b-d)$

5) $\VEC{a,b} - \VEC{c,d} = \VEC{a-c,b-d}$

6) $k·\VEC{a,b} = \VEC{ka,kb}$

7) $\VEC{a,b}·\VEC{c,d} = ac+bd$ \quad (!!!!)

\ssk

As outras operações dão erro. Por exemplo:

$\VEC{a,b}+(c,d) = \erro$

$(a,b)+(c,d) = \erro$

$(a,b)·k = \erro$

\bsk

{\bf Exercícios}

\ssk

\def\V(#1){\VEC{#1}}
\def\und#1#2{\underbrace{#1}_{#2}}
\def\unds#1#2#3{\und {#1} {\sm{ \text{[regra #2]} \\ #3 }} }

% (find-es "tex" "boxedminipage")

6) Calcule:

\begin{minipage}[t]{2.25in}

a) $(2,3)+(\V(4,5)+\V(10,20))$

b) $((2,3)+\V(4,5))+\V(10,20)$

c) $4·((20,30)-(5,10))$

d) $\V(2,3)·\V(5,10)$

e) $\V(5,10)·\V(2,3)$

f) $(\V(2,3)·\V(5,10))·\V(10,100)$

g) $\V(2,3)·(\V(5,10)·\V(10,100))$

h) $(\V(5,10)·\V(10,100))·\V(2,3)$

i) $(\V(10,100)·\V(5,10))·\V(2,3)$

j) $(\V(10,100)·\V(2,3))·\V(5,10)$

\end{minipage}
%
\begin{minipage}[t]{2in}

Obs: dois modos de resolver o 6a:

(o segundo é o modo padrão)

\msk

a)
$\unds {(2,3)+(\unds {\V(4,5)+\V(10,20)}
                     2
                     {=\;\V(14,25)}     )}
       1
       {=\;(16,28)}
$

\msk

a) $\begin{array}[t]{l}
      (2,3)+(\V(4,5)+\V(10,20)) \\
    = (2,3)+\V(14,25) \\
    = (16,28) \\
    \end{array}
   $

\end{minipage}


\newpage

%     _                                 __ 
%    / \    _ __   __   __ _ _ __ __ _ / _|
%   / _ \ _| |\ \ / /  / _` | '__/ _` | |_ 
%  / ___ \_   _\ V /  | (_| | | | (_| |  _|
% /_/   \_\|_|  \_/    \__, |_|  \__,_|_|  
%                      |___/               
%
% «pontos-e-vetores-graficamente» (to ".pontos-e-vetores-graficamente")
% (gam181p 12 "pontos-e-vetores-graficamente")

\mypsection {pontos-e-vetores-graficamente} {Como representar pontos e vetores graficamente}

1) Represente num gráfico os pontos $A=(2,1)$, $B=(4,0)$, $C=(3,3)$ e
escreva perto de cada um destes pontos o seu nome --- $A$, $B$, $C$.
Repare que se o seu gráfico estiver claro o suficiente o leitor vai
entender que os seus pontos têm coordenadas inteiras e vai conseguir
descobrir as coordenadas de $A$, $B$ e $C$.

2) Vetores correspondem a {\sl deslocamentos} vão ser representados
como setas indo de um ponto a outro. O vetor $\vv=\VEC{2,1}$
corresponde a um deslocamento de duas unidades para a direita e uma
unidade pra cima; uma seta indo do ponto $D=(0,3)$ para o ponto
$D+\vv=(0+2,3+1)$ é uma representação do vetor $\vv$ ``apoiado no
ponto $D$''. Um bom modo de representar graficamente o vetor $\vv$
apoiado no ponto $D$ é representando os pontos $D$ e $D+\vv$ ---
lembre de escrever os nomes $D$ e $D+\vv$ perto destes pontos no
gráfico --- e fazer uma seta de $D$ para $D+\vv$ e escrever $\vv$
perto dela. Represente graficamente o vetor $\vv$ apoiado no ponto
$D$.

3) Um bom modo de representar graficamente a soma $D+\vv$ é fazer o
mesmo que no item anterior, representando graficamente os pontos $D$,
$D+\vv$ e a seta (``$\vv$'') indo de um para o outro. Represente as
somas $A+\vv$, $B+\vv$, $C+\vv$ e $D+\vv$ num gráfico só, e repare que
você vai ter várias setas com o nome ``$\vv$'' --- todas elas {\sl são
  o mesmo vetor}, mas representado ``apoiado em pontos diferentes''.

4) A ``aula 1'' no livro do CEDERJ, que usa uma abordagem diferente da
nossa, se chama ``Vetores no plano - segmentos orientados''. Dê uma
olhada nas três primeiras páginas da ``aula 1'' (até o fim da
``Definição 1'') pra ter uma noção de como ele faz as coisas ---
repare que o início do livro não usa coordenadas, elas só aparecem
depois! --- e depois dê uma olhada nas proposições 1 e 2 do livro e
nas definições 2 e 3.

5) Seja $\ww=\VEC{0,-1}$. Represente graficamente $A+\ww$, $B+\ww$,
$C+\ww$, $D+\ww$.

6) Calcule $\vv+\ww$ e $\ww+\vv$ usando as regras definidas na página
anterior.

7) Represente graficamente $B+\vv$, $(B+\vv)+\ww$ e $B+(\vv+\ww)$ num
gráfico só.

8) Leia a ``Definição 4'' no livro do CEDERJ e compare os desenhos
dele com os seus desenhos do item anterior.

9) Represente graficamente $B+\vv$, $(B+\vv)+\ww$, $B+(\vv+\ww)$,
$B+\ww$ e $(B+\ww)+\vv$ num gráfico só.

10) Leia a p.23 do livro do CEDERJ.

11) Calcule $2\vv$, $3\vv$, $0\vv$ e $(-1)\vv$ usando as regras da
página anterior.

12) Represente graficamente os vetores $2\vv$, $3\vv$, $0\vv$ e
$(-1)\vv$ apoiando-os no ponto $B$.

13) Dê uma olhada nas ``Propriedades da adição de vetores'' e na
``Definição 5'' no livro do CEDERJ (páginas 23--25).




\newpage

%           _
%  _ __ ___| |_ __ _ ___
% | '__/ _ \ __/ _` / __|
% | | |  __/ || (_| \__ \
% |_|  \___|\__\__,_|___/
%
% «retas-de-novo» (to ".retas")
% (gam181p 13 "retas-de-novo")
\mypsection {retas-de-novo} {Retas (de novo)}

{\bf Exercícios}

\ssk

% (find-LATEXfile "2016-1-GA-material.tex" "em sala em 16/dez/2015")

1) Represente graficamente as retas abaixo.

Dica: encontre dois pontos de cada reta e marque-os no gráfico.

Dica 2: quando você tiver dificuldade substitua $\R^2$ por
$\{0,1,2,3,4,5\}^2$.

Nas parametrizadas indique no gráfico os pontos associados a $t=0$ e $t=1$.

$r_a = \setofxyst{ x+2y=0 }$

$r_b = \setofxyst{ x+2y=4 }$

$r_c = \setofxyst{ x+2y=2 }$

$r_d = \setofxyst{ 2x+3y=0 }$

$r_e = \setofxyst{ 2x+3y=6 }$

$r_f = \setofxyst{ 2x+3y=3 }$

$r_l = \setofxyst{ y=4 }$

$r_m = \setofxyst{ y=4+x }$

$r_n = \setofxyst{ y=4-2x }$

$r_g = \setofpt 3 -1 -1 1 $

$r_h = \setofpt 3 -1 -2 1 $

$r_i = \setofpt 3 -1 1 -1 $

$r_j = \setofpt 0 3 2 0 $

$r_k = \setofpt 2 0 0 1 $

$s_a = \setofxyst{ \V(x,y)·\V(1,2)=0 }$

$s_b = \setofxyst{ \V(x,y)·\V(1,2)=4 }$

$s_c = \setofxyst{ \V(x,y)·\V(1,2)=2 }$

$s_d = \setofxyst{ \V(x,y)·\V(2,3)=0 }$

$s_e = \setofxyst{ \V(x,y)·\V(2,3)=6 }$

$s_f = \setofxyst{ \V(x,y)·\V(2,3)=3 }$

$r'_l = \setofxyst{ 0x+1y=4 }$

$r'_m = \setofxyst{ (-1)x+1y=4 }$

$r'_n = \setofxyst{ 2x+1y=4 }$

$s_l = \setofxyst{ \V(x,y)·\V(0,1)=4 }$

$s_m = \setofxyst{ \V(x,y)·\V(-1,1)=4 }$

$s_n = \setofxyst{ \V(x,y)·\V(2,1)=4 }$

\msk

{\setlength{\parindent}{0em}

Se você tiver dificuldade com os exercícios envolvendo o produto
$\VEC{a,b}·\VEC{c,d}=ac+bd$ faça os exercícios abaixo e depois volte
pro exercício 1 acima.

2) Encontre soluções para $\VEC{x,y}·\VEC{2,3}=4$ para: a) $x=1$, b)
$x=10$, c) $y=5$.

3) Sejam $\uu_1 = \VEC{4,5}$ e $\uu_2 = \VEC{6,7}$. Calcule
$\uu_1+\uu_2$, $(\uu_1)_1$, $(\uu_1)_2$.

4) Encontre 5 soluções diferentes para $\VEC{x,y}·\VEC{1,2}=0$.
Chame-as de $\vv_1$, $\vv_2$, $\vv_3$, $\vv_4$, $\vv_5$. Desenhe no
mesmo gráfico os vetores $\VEC{1,2}$, $\vv_1$, $\vv_2$, $\vv_3$,
$\vv_4$, $\vv_5$ apoiando todos no ponto $(0,0)$.

}


\newpage

%         _       _                  
%  _ __  (_)_ __ | |_ ___ _ __   ___ 
% | '__| | | '_ \| __/ _ \ '__| / __|
% | |    | | | | | ||  __/ |    \__ \
% |_|    |_|_| |_|\__\___|_|    |___/
%                                    
% «intersecoes-de-retas» (to ".intersecoes-de-retas")
% (gam181p 14 "intersecoes-de-retas")
% (gaap162 14 "parametrizadas")
{\setlength{\parindent}{0em}

\mypsection {intersecoes-de-retas} {Interseções de retas parametrizadas}

%L r0, rv = v(2,3), v(1,1)
%L s0, sw = v(2,3), v(2,-1)
%L rt = function (t) return r0 + t*rv end
%L su = function (u) return s0 + u*sw end
\pu
\def\rt#1{\expr{rt(#1):xy()}}
\def\su#1{\expr{su(#1):xy()}}

% \rt 0 \rt 1 \rt 2
% \su 0 \su 1 \su 2


Se $r = \setofpt 3 3 2 -1 $

e $s = \setofpu 4 1 -1 1 $,

então $r$ e $s$ se intersectam no ponto $P=(1,4)$,

que está associado a $t=-1$ (em $r$) e a $u=3$ (em $s$).

Graficamente,

\msk

%L inter  = v(1,4)
%L r0, rv = v(3,3), v(2,-1)
%L s0, sw = v(4,1), v(-1,1)
\pu
% (find-pgfmanualpage  44 "3.9    Adding Labels Next to Nodes")
% (find-pgfmanualtext  44 "3.9    Adding Labels Next to Nodes")
$\tikzp{[scale=0.5,auto]
    \mygrid (-1,-1) (7,5);
    \draw[mycurve] \rt{-2} -- \rt{5};
    \draw[mycurve] \su{-2} -- \su{5};
    \node [cldot] at \rt{0} [label=60:${t{=}0}$] {};
    \node [cldot] at \rt{1} [label=60:${t{=}1}$] {};
    \node [cldot] at \su{0} [label=200:${u{=}0}$] {};
    \node [cldot] at \su{1} [label=200:${u{=}1}$] {};
    \node [cldot] at \su{3} [label=60:$P$] {};
  }
$

\msk

Algebricamente, podemos convencer alguém do nosso resultado assim:

$(1,4) = (3,3)+(-1)\VEC{2,-1} ∈ r$,

$(1,4) = (4,1)+3\VEC{-1,1} ∈ s$,

$(1,4) ∈ r∩s$.

\ssk

Repare que poderíamos ter encontrado $(x,y)=P∈r∩s$ usando um sistema:

$(x,y) = (3+2t, 3-t)$

$(x,y) = (4-u, 1+u)$

Primeiro encontramos $t$ e $u$ tais que $(3+2t, 3-t) = (4-u, 1+u)$,

depois encontramos $(x,y) = (3+2t, 3-t) = (4-u, 1+u)$.

\msk

{\bf Exercício}

\ssk

1) Em cada um dos casos abaixo represente graficamente $r$ e $s$,

encontre $P∈r∩s$, e verifique algebricamente que o seu $P$ está certo.

a) $r = \setofpt 1 0 0 3 $, $s = \setofpu 0 4 2 0 $

b) $r = \setofpt 1 0 3 1 $, $s = \setofpu 0 2 2 3 $

c) $r = \setofet{ (1+3t,t) }$, $s = \setofeu{ (2u,2+3u) } $

d) $r = \setofpt 0 3 2 -1 $, $s = \setofpu 1 0 1 3 $

\ssk

Obs: no (d) o olhômetro não basta, você vai precisar resolver um sistema.

}

\newpage



%   ___
%  / _ \    _   _    __   __
% | | | |  | | | |   \ \ / /
% | |_| |  | |_| |_   \ V /
%  \___( )  \__,_( )   \_/
%      |/        |/
%
% «sistemas-de-coordenadas» (to ".sistemas-de-coordenadas")
% (gam181p 15 "sistemas-de-coordenadas")
% (gaap162 11 "coordenadas")
{\setlength{\parindent}{0em}

\mypsection {sistemas-de-coordenadas} {Sistemas de coordenadas}

Um ``sistema de coordenadas'' $Σ=(O,\uu,\vv)$ em $\R^2$ é uma tripla
formada por um ponto e dois vetores; por exemplo, podemos ter
$Σ=((2,3),\VEC{2,1},\VEC{0,-1})$ --- aí $O=(2,3)$, $\uu=\VEC{2,1}$,
$\vv=\VEC{0,-1}$. Até agora nós só usamos pontos com coordenadas $x$ e
$y$, mas agora vamos começar a falar das ``coordenada $a$ e $b$'' de
um ponto, e elas vão depender da escolha do sistema de coordenadas
$Σ$. A definição importante (que só vale para esta página e a
seguinte!) é:

\ssk

$(a,b)_Σ = O+a\uu+b\vv$.

\msk

{\bf Exercícios}

\ssk

1) $((a,b)_Σ = O+a\uu+b\vv) \subst{a:=3 \\ b:=4 } = ?$

2) $((a,b)_Σ = O+a\uu+b\vv) \subst{O:=(3,1) \\ \uu:=\VEC{2,1} \\ \vv:=\VEC{-1,1} } = ?$

3) Em cada um dos casos $a$ até $f$ abaixo descubra quem são $O$,
$\uu$ e $\vv$ olhando para o gráfico.

{

\unitlength=11pt
\def\closeddot{\circle*{0.4}}
\def\cellfont{\scriptsize}
\def\cellfont{}

a)
$\vcenter{\hbox{%
   \beginpicture(-1,-1)(11,9)%
   \eval{O,uu,vv = v(3,1), v(2,1), v(-1,1)}
   \pictOuv(0.5, 0.7)
   \pictABCDE(180, 180, 0, 180, 0)
   \end{picture}%
  }}
$
%
\quad
%
b)
$\vcenter{\hbox{%
   \beginpicture(-1,-1)(6,6)
   \eval{O, uu, vv  = v(2, 2), v(1, 0), v(0, 1)}
   \pictOuv(0.5, 0.7)
   \end{picture}%
  }}
$

c)
$\unitlength=9pt
 \vcenter{\hbox{%
   \beginpicture(-6,-1)(3,6)
   \eval{O, uu, vv  = v(-5, 1), v(2, 0), v(0, 1)}
   \pictOuv(0.5, 0.7)
   \end{picture}%
 }}
$
%
\quad
%
d)
$\unitlength=9pt
 \vcenter{\hbox{%
   \beginpicture(-1,-1)(5,9)
   \eval{O, uu, vv = v(1, 1), v(1, 0), v(0, 2)}
   \pictOuv(0.5, 0.7)
   \end{picture}%
 }}
$
%
\quad
%
e)
$\vcenter{\hbox{%
   \beginpicture(-1,-1)(6,6)
   \eval{O, uu, vv = v(2, 2), v(0, 1), v(1, 0)}
   \pictOuv(-0.5, 0.7)
   \end{picture}%
 }}
$


f)
$\unitlength=8pt
 \vcenter{\hbox{%
    \beginpicture(-8,-4)(6,8)
    \eval{O, uu, vv = v(4, 4), v(-2, 1), v(-1, -2)}
   \pictOuv(0.5, 0.7)
   \end{picture}%
 }}
$
%
\quad
%
g)
$\vcenter{\hbox{%
   \beginpicture(-4,-1)(5,6)
   \eval{O, uu, vv = v(-3, 1), v(1, 0), v(1, 1)}
   \pictOuv(0.5, 0.7)
   \end{picture}%
 }}
$

}

\msk

4a) Represente graficamente os pontos $(0,0)_Σ$, $(1,0)_Σ$, $(2,0)_Σ$,
$(0,1)_Σ$, $(1,1)_Σ$, $(2,1)_Σ$, $(0,2)_Σ$, $(1,2)_Σ$, $(2,2)_Σ$ num
gráfico só para o $Σ$ do item (a) acima, e escreva perto de cada ponto
o nome dele --- por exemplo, ``$(1,2)_Σ$''.

4b, 4c, 4d, 4e, 4f, 4g) Faça o mesmo para o $Σ$ do item (b), do item
(c), etc.

}

\newpage


%   ___                       ____  
%  / _ \    _   _    __   __ |___ \ 
% | | | |  | | | |   \ \ / /   __) |
% | |_| |  | |_| |_   \ V /   / __/ 
%  \___( )  \__,_( )   \_/   |_____|
%      |/        |/                 

% «sistemas-de-coordenadas-2» (to ".sistemas-de-coordenadas-2")
% (gam181p 16 "sistemas-de-coordenadas-2")
{\setlength{\parindent}{0em}

\mypsection {sistemas-de-coordenadas-2} {Sistemas de coordenadas (2)}

Cada uma das figuras abaixo usa um sistema de coordenadas
$Σ=(O,\uu,\vv)$ diferente; lembre que $(a,b)_Σ = O+a\uu+b\vv$. Sejam:

$B = (1,3)_Σ$, \phantom{$E = (2,2)_Σ$} $C = (3,3)_Σ$,

$D = (1,2)_Σ$, $E = (2,2)_Σ$,

$A = (1,1)_Σ$.

{\bf Exercício.} Em cada um dos casos abaixo desenhe a figura formada
pelos pontos $A$, $B$, $C$, $D$ e $E$ e pelos segmentos de reta
$\overline{AB}$, $\overline{BC}$ e $\overline{DE}$.

(O item (a) já está feito.)

{

\unitlength=12pt
\def\closeddot{\circle*{0.4}}
\def\cellfont{\scriptsize}
\def\cellfont{}

a)
$\vcenter{\hbox{%
   \beginpicture(-1,-1)(11,9)%
   \eval{O,uu,vv = v(3,1), v(2,1), v(-1,1)}
   \pictOuv(0.5, 0.7)
   \pictABCDE(180, 180, 0, 180, 0)
   \end{picture}%
  }}
$
%
\quad
%
b)
$\vcenter{\hbox{%
   \beginpicture(-1,-1)(6,6)
   \eval{O, uu, vv  = v(2, 2), v(1, 0), v(0, 1)}
   \pictOuv(0.5, 0.7)
   \end{picture}%
  }}
$

c)
$\unitlength=10pt
 \vcenter{\hbox{%
   \beginpicture(-6,-1)(3,6)
   \eval{O, uu, vv  = v(-5, 1), v(2, 0), v(0, 1)}
   \pictOuv(0.5, 0.7)
   \end{picture}%
 }}
$
%
\quad
%
d)
$\unitlength=10pt
 \vcenter{\hbox{%
   \beginpicture(-1,-1)(5,9)
   \eval{O, uu, vv = v(1, 1), v(1, 0), v(0, 2)}
   \pictOuv(0.5, 0.7)
   \end{picture}%
 }}
$
%
\quad
%
e)
$\vcenter{\hbox{%
   \beginpicture(-1,-1)(6,6)
   \eval{O, uu, vv = v(2, 2), v(0, 1), v(1, 0)}
   \pictOuv(-0.5, 0.7)
   \end{picture}%
 }}
$


f)
$\unitlength=10pt
 \vcenter{\hbox{%
    \beginpicture(-8,-4)(6,8)
    \eval{O, uu, vv = v(4, 4), v(-2, 1), v(-1, -2)}
   \pictOuv(0.5, 0.7)
   \end{picture}%
 }}
$
%
\quad
%
g)
$\vcenter{\hbox{%
   \beginpicture(-4,-1)(5,6)
   \eval{O, uu, vv = v(-3, 1), v(1, 0), v(1, 1)}
   \pictOuv(0.5, 0.7)
   \end{picture}%
 }}
$

}




}

\msk



\newpage



%  ____  _     _
% / ___|(_)___| |_ ___ _ __ ___   __ _ ___
% \___ \| / __| __/ _ \ '_ ` _ \ / _` / __|
%  ___) | \__ \ ||  __/ | | | | | (_| \__ \
% |____/|_|___/\__\___|_| |_| |_|\__,_|___/
%
% «sistemas» (to ".sistemas")
% (gam181p 17 "sistemas")
% (gaap162 12 "sistemas")

{\setlength{\parindent}{0em}

\mypsection {sistemas} {Sistemas de equações e sistemas de coordenadas}

%L p = function (a, b) return O + a*uu + b*vv end

\begin{minipage}[t]{2.5in}

No item (f) da página anterior temos:

\ssk


$\unitlength=8pt
 \def\cellfont{}
 \def\cellfont{\footnotesize}
 \vcenter{\hbox{%
   \beginpicture(-8,-4)(6,8)
   \eval{O, uu, vv = v(4, 4), v(-2, 1), v(-1, -2)}
   \pictOuv(0.5, 0.7)
   \end{picture}%
 }}
 \quad
 {\footnotesize
 \begin{array}{l}
 O       = (4,4) \\
 \uu     = \V(-2,1) \\
 \vv     = \V(-1,-2) \\
 \end{array}
 }
$

% \ssk

$(a,b)_Σ = (4,4) + a\V(-2,1) + b\V(-1,-2)$

$(a,b)_Σ = (4-2a-b, 4+a-2b) \quad\; (*)$

\ssk

$\begin{array}[t]{rcl}
   (a,b)_Σ &=& (x,y) \\\hline
   %----------------
   (0,0)_Σ &=& (4,4) \\
   (1,0)_Σ &=& (2,5) \\
   (0,1)_Σ &=& (3,2) \\
 A=(1,1)_Σ &=& ?_a \\
 B=(1,3)_Σ &=& ?_b \\
 C=(3,3)_Σ &=& ?_c \\
 D=(1,2)_Σ &=& ?_d \\
 E=(2,2)_Σ &=& ?_e \\
       ?_f &=& (0,6) \\
       ?_g &=& (-1,4) \\
       ?_h &=& (5,1) \\
       ?_i &=& (1,2) \\
       ?_j &=& (1,1) \\
       ?_k &=& (2,1) \\
 \end{array}
 %
$

\ssk

Os itens (a) até (h) acima (``$?_a$'' a ``$?_h$'') são fáceis de
resolver ``no olhômetro'' usando o gráfico, e é fácil conferir os
resultados algebricamente usando a fórmula $(*)$.

\msk

No item (i) dá pra ver pelo gráfico que os valores de $a$ e $b$ em
$(a,b)_Σ = (1,2)$ vão ser fracionários e difíceis de chutar -- mas
podemos obtê-los {\sl algebricamente}, resolvendo um {\sl sistema de
  equações}.

\end{minipage}
%
\qquad
%
\begin{minipage}[t]{2.25in}

\begin{boxedminipage}[t]{2.25in}

\footnotesize

Solução do ``$?_i$'':

\ssk

$\begin{array}{rcl}
   (a,b)_Σ &=& (1,2) \\
   (4-2a-b, 4+a-2b) &=& (1,2) \\
   4-2a-b &=& 1 \\
   4+a-2b &=& 2 \\
   -2a-b  &=& -3 \\
   a-2b   &=& -2 \\
   -2a+3  &=& b \\
   a      &=& -2+2b \\
   -2(-2+2b)+3 &=& \color{red}{b} \\
   4-4b+3 &=& b \\
   7      &=& 5b \\
   b      &=& \frac 7 5 \\
   a      &=& -2 + 2 \frac 7 5 \\
          &=& \frac{-10}{5} + \frac{14}{5} \\
          &=& \frac{4}{5} \\
   (\frac{4}{5}, \frac{7}{5})_Σ &=& (1,2) \\
 \end{array}
 %
$

\end{boxedminipage}

\bsk

\begin{boxedminipage}[t]{2.25in}
\footnotesize

Uma generalização:

\ssk

$\begin{array}{rcl}
   (a,b)_Σ &=& (x,y) \\
   (4-2a-b, 4+a-2b) &=& (x,y) \\
   4-2a-b &=& x \\
   4+a-2b &=& y \\
   4-2a-x &=& b \\
 \end{array}
$

\ssk

$\begin{array}{rcl}
   a &=& y+2b-4 \\
     &=& y+2(4-2a-x)-4 \\
     &=& y+8-4a-2x-4 \\
     &=& y-2x+4-4a \\
   5a &=& y-2x+4 \\
    a &=& (y-2x+4)/5 \\
      &=& \frac15 y - \frac25 x + \frac45 \\
      &=& \frac45 - \frac25 x + \frac15 y \\
  % b &=& 4-2(\frac15 y - \frac25 x + \frac45)-x \\
    b &=& 4-2(\frac45 - \frac25 x + \frac15 y)-x \\
      &=& \frac{20}5 - \frac85 + \frac45 x - \frac25 y  -\frac55 x \\
      &=& \frac{12}5 - \frac15 x - \frac25 y \\
 \end{array}
 %
$

\ssk

$(\frac45 - \frac25 x + \frac15y,
  \frac{12}5 - \frac15 x - \frac25 y)_Σ = (x,y)
$

\ssk

Vamos chamar a fórmula acima de $(**)$.

\end{boxedminipage}

\end{minipage}

\bsk

{\bf Exercícios}

1) Resolva ``$?_j$'' pelo sistema.

2) Resolva ``$?_k$'' pelo sistema.

3) Verifique que as suas soluções de ``$?_a$'' até ``$?_k$'' obedecem
$(*)$ e $(**)$.

4) Resolva ``$?_j$'' e ``$?_k$'' por $(**)$.

}

\newpage



%  ____  _     _                             ____
% / ___|(_)___| |_ ___ _ __ ___   __ _ ___  |___ \
% \___ \| / __| __/ _ \ '_ ` _ \ / _` / __|   __) |
%  ___) | \__ \ ||  __/ | | | | | (_| \__ \  / __/
% |____/|_|___/\__\___|_| |_| |_|\__,_|___/ |_____|
%

% «sistemas-2» (to ".sistemas-2")
% (gam181p 18 "sistemas-2")
% (gaap162 13 "sistemas-2")

{\setlength{\parindent}{0em}

\mypsection {sistemas-2} {Sistemas de equações e sistemas de coordenadas (2)}

Um outro modo de organizar os problemas da página anterior é o
seguinte.

Temos as equações $[x]$, $[y]$, $[a]$, $[b]$ abaixo,

\ssk

$\begin{array}{crcl}
   {}[x] & x &=& 4-2a-b \\
   {}[y] & y &=& 4+a-2b \\
   {}[a] & a &=& \frac45 -\frac25 x + \frac15 y  \\
   {}[b] & b &=& \frac{12}5 -\frac15 x - \frac25 y \\
 \end{array}
$

\ssk

e queremos preencher a tabela abaixo de tal forma que em cada linha

as equações $[x]$, $[y]$, $[a]$, $[b]$ sejam obedecidas:

$\begin{array}{rrrr}
   a & b & x & y \\\hline
   %----------------
 0 & 0 & 4 & 4 \\
 1 & 0 & 2 & 5 \\
 0 & 1 & 3 & 2 \\
 1 & 1 & · & · \\
 1 & 3 & · & · \\
 3 & 3 & · & · \\
 1 & 2 & · & · \\
 2 & 2 & · & · \\
 · & · & 0 & 6 \\
 · & · &-1 & 4 \\
 · & · & 5 & 1 \\
 · & · & 1 & 2 \\
 · & · & 1 & 1 \\
 · & · & 2 & 1 \\
 \end{array}
 %
$

\msk

Note que:

1) quando as lacunas são em $x$ e $y$ é mais rápido usar as equações
$[x]$ e $[y]$,

2) quando as lacunas são em $a$ e $b$ é mais rápido usar as equações
$[a]$ e $[b]$,

3) as equações $[a]$ e $[b]$ são {\sl consequências} das $[x]$ e $[y]$,

4) $[x]$ e $[y]$ são consequências de $(a,b)_Σ = (4-2a-b, 4+a-2b) = (x,y)$,

5) $\psm{x\\ y\\}
    = \psm{4-2a-b\\ 4+a-2b\\}
    = \psm{4\\ 4\\} + \psm{-2a-b\\ a-2b\\}
    = \psm{4\\ 4\\} + \psm{-2 & -1\\ 1 & -2\\} \psm{a\\ b\\}
   $

6) $\psm{x\\ y\\}
    = \psm{O_1 +au_1 +bv_1 \\ O_2 + au_2 + bv_2\\}
    = \psm{O_1\\ O_2\\} + \psm{u_1 & v_1\\ u_2 & v_2\\} \psm{a\\ b\\}
   $

\msk

{\bf Exercícios}

\ssk

1) No item (g) duas páginas atrás temos $O=(-3,1)$, $\uu=\V(1,0)$,
$\vv=\V(1,1)$, $(a,b)_Σ = (-3+a+b, 1+b)$. Obtenha as equações $[x]$,
$[y]$, $[a]$, $[b]$ para este caso.

2) Faça o mesmo para o item (a), onde $O=(3,1)$, $\uu=\V(2,1)$,
$\vv=\V(-1,1)$.

}


\newpage



% __     __         _                                     _     
% \ \   / /_ _ _ __(_) __ _ ___    ___ ___   ___  _ __ __| |___ 
%  \ \ / / _` | '__| |/ _` / __|  / __/ _ \ / _ \| '__/ _` / __|
%   \ V / (_| | |  | | (_| \__ \ | (_| (_) | (_) | | | (_| \__ \
%    \_/ \__,_|_|  |_|\__,_|___/  \___\___/ \___/|_|  \__,_|___/
%                                                               
% «varias-coords» (to ".varias-coords")
% (gam181p 19 "varias-coords")
% (gaap162 15 "sistemas-3")
\mypsection {varias-coords} {Vários sistemas de coordenadas ao mesmo tempo}

\def\xx{\vec x}
\def\yy{\vec y}
\def\aa{\vec a}
\def\bb{\vec b}
\def\cc{\vec c}
\def\dd{\vec d}
\def\ee{\vec e}
\def\ff{\vec f}
\def\gg{\vec g}
\def\hh{\vec h}

\ssk

Há muitas notações possíveis para lidar com situações em que temos
vários sistemas de coordenadas ao mesmo tempo -- vamos ver {\sl uma}
delas.

Vamos ter:

$\bullet$ as coordenadas $x,y$ e os eixos $x$ e $y$,

$\bullet$ as coordenadas $a,b$ e os eixos $a$ e $b$,

$\bullet$ as coordenadas $c,d$ e os eixos $d$ e $d$,

$\bullet$ as coordenadas $e,f$ e os eixos $e$ e $f$,

\noindent e além disso vamos ter as origens $O_{xy}$, $O_{ab}$,
$O_{cd}$, $O_{ef}$ de cada um dos sistemas de coordenadas e os vetores
$\xx$, $\yy$, $\aa$, $\bb$, $\cc$, $\dd$, $\ee$, $\ff$.

\msk

Um exemplo concreto:

$\unitlength=15pt
 \def\closeddot{\circle*{0.2}}
 \def\cellfont{\scriptsize}
 \def\cellfont{}
 \vcenter{\hbox{%
   \beginpicture(-1,-2)(6,6)%
   \pictgrid%
   \pictaxes%
   {\linethickness{1.0pt}
    \expr{pictOOuuvv(v(0,0),  v(1,0),   v(0,1),  "!;!;O_{xy}", "!xx", "!yy", 0.5, 0.7)}
    \expr{pictOOuuvv(v(2,-1), v(1,0),   v(0,1),  "!;!;O_{ab}", "!aa", "!bb", 0.5, 0.7)}
    \expr{pictOOuuvv(v(5,5),  v(-2,0),  v(0,-2), "!;!;O_{cd}", "!cc", "!dd", 0.5, 0.5)}
    \expr{pictOOuuvv(v(1,5),  v(-1,-1), v(1,-1), "!;!;O_{ef}", "!ee", "!ff", 0.5, 0.5)}
   }
   \put(1,1){\closeddot}
   \put(3,1){\closeddot}
   \put(5,1){\closeddot}
   \put(1,3){\closeddot}
   \put(3,3){\closeddot}
   \end{picture}%
  }}%
  %
  \qquad
  %
  \begin{array}{l}
    \begin{array}{lll}
      O_{xy}=(0,0)  & \xx=\V(1,0) & \yy=\V(0,1) \\
      O_{ab}=(2,-1) & \aa=\V(1,0) & \bb=\V(0,1) \\
      O_{cd}=(5,5)  & \cc=\V(-2,0) & \dd=\V(0,-2) \\
      O_{ef}=(1,5)  & \ee=\V(-1,-1) & \ff=\V(1,-1) \\
    \end{array}
    %
    \\[5pt]
    \\
    %
    \begin{array}{lllll}
      (x,y)_{xy} & = & O_{xy} + x\xx + y\yy & = & (x,y) \\
      (a,b)_{ab} & = & O_{ab} + a\aa + b\bb & = & (a+2,b-1) \\
      (c,d)_{cd} & = & O_{cd} + c\cc + d\dd \\
      (e,f)_{ef} & = & O_{ef} + e\ee + f\ff \\
    \end{array}
  \end{array}
$

\bsk

Um modo de entender esta notação é:
%
$$\begin{array}{ll}
  \left( (c,d)_{cd} \; = \; O_{cd} + c\cc + d\dd \right)
    \subst{ O_{cd}:=(5,5) \\ \cc:=\V(-2,0) \\ \dd=\V(0,-2) \\ }
    \subst{ c:=3 \\ d:=4 \\ }
  \\[12pt] =
  \left( (c,d)_{cd} \; = \; (5,5) + c\V(-2,0) + d\V(0,-2) \right)
    \subst{ c:=3 \\ d:=4 \\ }
  \\[5pt] =
  \left( (3,4)_{cd} \; = \; (5,5) + 3\V(-2,0) + 4\V(0,-2) \right)
  \end{array}
$$

Nós vimos (na p.\pageref{pontos-e-vetores}) que as notações ``$P_1$''
e ``$P_2$'' dão a ``primeira componente'' e a ``segunda componente''
de um ponto $P$. Se usarmos as notações $P_x$, $P_x$, $P_a$, $P_b$,
$P_c$, $P_d$, $P_e$, $P_f$ para as ``coordenadas'' $x$, $y$, $a$, $b$,
$c$, $d$, $e$, $f$ de um ponto $P$ temos:

\ssk

$P = (P_x,P_y)_{xy} = (P_a,P_b)_{ab} = (P_c,P_d)_{cd} = (P_e,P_f)_{ef}$

\ssk

\noindent e se considerarmos que $x$, $y$, $a$, $b$, $c$, $d$, $e$,
$f$ são variáveis que ``variam juntas'' (como nas págs
\pageref{sistemas} e \pageref{sistemas-2}), a condição que elas
obedecem é:

\ssk

$(x,y) = (x,y)_{xy} = (a,b)_{ab} = (c,d)_{cd} = (e,f)_{ef}$

\ssk

{\bf Exercícios}

1) Digamos que $P=(3,1)$. Descubra $P_x$, $P_x$, $P_a$, $P_b$, $P_c$,
$P_d$, $P_e$, $P_f$.

2) Digamos que $(x,y)=(5,1)$. Descubra $a$, $b$, $c$, $d$, $e$, $f$.


\newpage



% __     __         _                                     _       ____  
% \ \   / /_ _ _ __(_) __ _ ___    ___ ___   ___  _ __ __| |___  |___ \ 
%  \ \ / / _` | '__| |/ _` / __|  / __/ _ \ / _ \| '__/ _` / __|   __) |
%   \ V / (_| | |  | | (_| \__ \ | (_| (_) | (_) | | | (_| \__ \  / __/ 
%    \_/ \__,_|_|  |_|\__,_|___/  \___\___/ \___/|_|  \__,_|___/ |_____|
%                                                                       
% «varias-coords-2» (to ".varias-coords-2")
% (gam181p 20 "varias-coords-2")
\mypsection {varias-coords-2} {Vários sistemas de coordenadas ao mesmo tempo (2)}

Podemos usar o diagrama da página anterior --- reproduzido abaixo ---
para desenhar ``grids'' como os das páginas
\pageref{sistemas-de-coordenadas} e
\pageref{sistemas-de-coordenadas-2}. Esse diagrama define sistemas de
coordenadas $(O_{xy}, \xx, \yy)$, $(O_{ab}, \aa, \bb)$, $(O_{cd}, \cc,
\dd)$, $(O_{ef}, \ee, \ff)$,

\msk

$\unitlength=15pt
 \def\closeddot{\circle*{0.2}}
 \def\cellfont{\scriptsize}
 \def\cellfont{}
 \vcenter{\hbox{%
   \beginpicture(-1,-2)(6,6)%
   % \pictgrid%
   \pictaxes%
   {\linethickness{1.0pt}
    \expr{pictOOuuvv(v(0,0),  v(1,0),   v(0,1),  "!;!;O_{xy}", "!xx", "!yy", 0.5, 0.7)}
    \expr{pictOOuuvv(v(2,-1), v(1,0),   v(0,1),  "!;!;O_{ab}", "!aa", "!bb", 0.5, 0.7)}
    \expr{pictOOuuvv(v(5,5),  v(-2,0),  v(0,-2), "!;!;O_{cd}", "!cc", "!dd", 0.5, 0.5)}
    \expr{pictOOuuvv(v(1,5),  v(-1,-1), v(1,-1), "!;!;O_{ef}", "!ee", "!ff", 0.5, 0.5)}
   }
   \put(1,1){\closeddot}
   \put(3,1){\closeddot}
   \put(5,1){\closeddot}
   \put(1,3){\closeddot}
   \put(3,3){\closeddot}
   \end{picture}%
  }}%
  %
  \qquad
  %
  \begin{array}{l}
    \begin{array}{lll}
      O_{xy}=(0,0)  & \xx=\V(1,0) & \yy=\V(0,1) \\
      O_{ab}=(2,-1) & \aa=\V(1,0) & \bb=\V(0,1) \\
      O_{cd}=(5,5)  & \cc=\V(-2,0) & \dd=\V(0,-2) \\
      O_{ef}=(1,5)  & \ee=\V(-1,-1) & \ff=\V(1,-1) \\
    \end{array}
    %
    \\[5pt]
    \\
    %
    \begin{array}{lllll}
      (x,y)_{xy} & = & O_{xy} + x\xx + y\yy & = & (x,y) \\
      (a,b)_{ab} & = & O_{ab} + a\aa + b\bb & = & (a+2,b-1) \\
      (c,d)_{cd} & = & O_{cd} + c\cc + d\dd \\
      (e,f)_{ef} & = & O_{ef} + e\ee + f\ff \\
    \end{array}
  \end{array}
$

\msk

{\setlength{\parindent}{0em}

{\bf Exercícios}

1) Trace:

uma reta contendo os pontos $(0,0)_{ef}$, $(1,0)_{ef}$, $(2,0)_{ef}$ e
chame-a de ``$f=0$'',

uma reta contendo os pontos $(0,1)_{ef}$, $(1,1)_{ef}$, $(2,1)_{ef}$ e
chame-a de ``$f=0$'',

uma reta contendo os pontos $(0,2)_{ef}$, $(1,2)_{ef}$, $(2,2)_{ef}$ e
chame-a de ``$f=0$'',

uma reta contendo os pontos $(0,0)_{ef}$, $(0,1)_{ef}$, $(0,2)_{ef}$ e
chame-a de ``$e=0$'',

uma reta contendo os pontos $(1,0)_{ef}$, $(1,1)_{ef}$, $(1,2)_{ef}$ e
chame-a de ``$e=1$'',

uma reta contendo os pontos $(2,0)_{ef}$, $(2,1)_{ef}$, $(2,2)_{ef}$ e
chame-a de ``$e=2$''.

\ssk

2) Faça a mesma coisa para as retas ``$c=0$'', ``$c=1$'', ``$c=2$'',
``$d=0$'', ``$d=1$'' e ``$d=2$'', mas agora só visualizando essas
retas mentalmente, sem desenhá-las.

\ssk

3) Idem para ``$a=0$'', ``$a=1$'', ``$a=2$'', ``$b=0$'', ``$b=1$'' e
``$b=2$''.

\ssk

4) Verifique que o ponto $(1,2)_{ef}$ está na interseção das retas
$e=1$ e $f=2$.

\ssk

5) Verifique que o ponto $(2,3)_{cd}$ está na interseção das retas
$c=2$ e $d=3$.

}

\msk

Obs: os exercícios acima vão ser bem importantes para a parte 2 do
curso, que é sobre cônicas em $\R^2$... por exemplo, pra desenhar uma
``hipérbole torta'' em $\R^2$ a gente vai desenhar $O_{uv}$, $\uu$ e
$\vv$, depois as retas $u=0$ e $v=0$ e depois os pontos
$(1/2,2)_{uv}$, $(1,1)_{uv}$, $(2,1/2)_{uv}$, $(-1/2,-2)_{uv}$,
$(-1,-1)_{uv}$, $(-2,-1/2)_{uv}$.

\msk

{\setlength{\parindent}{0em}

6) Complete, usando o diagrama acima e olhômetro:

$\begin{array}{cllll}
 \text{ponto} & (\_,\_)_{xy} & (\_,\_)_{ab} & (\_,\_)_{cd} & (\_,\_)_{ef} \\\hline
 P & (1,1)_{xy} & (-1,2)_{ab} & (2,2)_{cd} &            \\
 Q & (3,1)_{xy} & (1,2)_{ab}  & (1,2)_{cd} & (1,3)_{ef} \\
 R & (5,1)_{xy} \\
 S & (1,3)_{xy} \\
 T & (3,3)_{xy} \\
 \end{array}
$

}

\newpage


%  _____    __                  __
% |  ___|  / / __  __    _   _  \ \
% | |_    | |  \ \/ /   | | | |  | |
% |  _|   | |   >  < _  | |_| |  | |
% |_|     | |  /_/\_( )  \__, |  | |
%          \_\      |/   |___/  /_/
%
% «Fxy» (to ".Fxy")
% (mpgp 24 "Fxy")
% (mpg     "Fxy")
% (gam181p 21 "Fxy")
% (find-LATEX "2016-2-C2-integral.tex" "pict2e")
% (find-LATEX "edrxgac2.tex" "pict2e")
{%\setlength{\parindent}{0em}
\mypsection {Fxy} {Visualizando $F(x,y)$}

\unitlength=8pt
\celllower=3pt
\def\cellfont{\scriptsize}

Um bom modo de começar a entender visualmente o comportamento de uma
função $F(x,y):\R^2→\R$ é fazendo diagramas como os abaixo, em que a
gente escreve sobre cada ponto $(x,y)$ o valor de $F(x,y)$ naquele
ponto... por exemplo, se $F(x,y)=x^2+y^2$ então $F(3,2)=9+4=13$, e a
gente escreve ``13'' no ponto $(3,2)$. Exemplos:

\msk

\def\smF#1{\sm{F(x,y) \\ #1} ⇒}

$\smF{\;\;\;\;=\, x}
 \pictureFxy(-1,-2)(5,2){x}
 \quad
 \smF{\;\;=\, 2y}
 \pictureFxy(-1,-2)(5,2){2*y}
 \quad
 \smF{=\,x+y}
 \pictureFxy(-1,-2)(5,2){x+y}
$

$\smF{\;\;\;=\,xy}
 \pictureFxy(-3,-3)(3,3){x*y}
 \quad
 \smF{\;\;\;=\,x^2+y^2}
 \pictureFxy(-3,-2)(3,2){x*x+y*y}
 % \quad
 % \sm{F(x,y) \\ =\,xy} ⇒
 % \pictureFxy(-1,-2)(5,2){x*y}
$

\msk

\noindent Repare que dá pra usar o diagrama de $F(x,y)=x+y$ pra ver onde
$x+y=0$, onde $x+y=3$, etc.

\msk

{\bf Exercícios}

\ssk

1) Faça diagramas como os acima para as funções:

a) $F(x,y) = \V(x,y)·\V(2,3)$

b) $F(x,y) = \V(x,y)·\V(3,1)$

c) $F(x,y) = \V(x,y)·\V(2,-1)$

d) $F(x,y) = x^2+y^2$ \qquad ($x,y∈\{-5,-4,\ldots,5\}^2$)

e) $F(x,y) = x^2-y$

f) $F(x,y) = y^2-x$

g) $F(x,y) = xy$

\msk

2) Use os diagramas do exercício anterior para esboçar os conjuntos abaixo

(que vão ser retas ou curvas):

\noindent\phantom{!!!}
\begin{tabular}{lll}
a0) $\setofxyst{ \V(x,y)·\V(2,3)=0 }$    & d25) $\setofxyst{ x^2+y^2=25 }$  \\
a2) $\setofxyst{ \V(x,y)·\V(2,3)=2 }$    & d4) $\setofxyst{ x^2+y^2=4 }$    \\
a4) $\setofxyst{ \V(x,y)·\V(2,3)=4 }$    & d1) $\setofxyst{ x^2+y^2=1 }$    \\
a-2) $\setofxyst{ \V(x,y)·\V(2,3)=-2 }$  & d0) $\setofxyst{ x^2+y^2=0 }$    \\
b0) $\setofxyst{ \V(x,y)·\V(3,1)=0 }$    & d-1) $\setofxyst{ x^2+y^2=-1 }$  \\
b3) $\setofxyst{ \V(x,y)·\V(3,1)=3 }$    & e0) $\setofxyst{ x^2-y=0 }$      \\  
b6) $\setofxyst{ \V(x,y)·\V(3,1)=6 }$    & e1) $\setofxyst{ x^2-y=1 }$      \\  
b-3) $\setofxyst{ \V(x,y)·\V(3,1)=-3 }$  & f0) $\setofxyst{ y^2-x=0 }$      \\  
c0) $\setofxyst{ \V(x,y)·\V(2,-1)=0 }$   & f1) $\setofxyst{ y^2-x=1 }$      \\  
c2) $\setofxyst{ \V(x,y)·\V(2,-1)=2 }$   & g0) $\setofxyst{ xy=0    }$      \\  
c4) $\setofxyst{ \V(x,y)·\V(2,-1)=4 }$   & g1) $\setofxyst{ xy=1 }$         \\
c-2) $\setofxyst{ \V(x,y)·\V(2,-1)=-2 }$ & g4) $\setofxyst{ xy=4 }$         \\
\end{tabular}

}


\newpage



%  ____  _ _                                  
% |  _ \(_) |_ __ _  __ _  ___  _ __ __ _ ___ 
% | |_) | | __/ _` |/ _` |/ _ \| '__/ _` / __|
% |  __/| | || (_| | (_| | (_) | | | (_| \__ \
% |_|   |_|\__\__,_|\__, |\___/|_|  \__,_|___/
%                   |___/                     
%
% «pitagoras» (to ".pitagoras")
% (gam181p 22 "pitagoras")
\mypsection {pitagoras} {O teorema de Pitágoras}

Para calcular a hipotenusa $h$ de um triângulo retângulo com catetos
$a$ e $b$ podemos fazer estas figuras:
%
$$\unitlength=15pt
 %
 \vcenter{\hbox{%
   \beginpicture(-4,-3)(7,7)%
   \pictgrid%
   %\pictaxes%
   {\linethickness{1.0pt}
    \eval{a=3; b=4}
    \Calcpoints{!polygon<0,0><a,0><a,-a><0,-a>}
    \Calcpoints{!polygon<0,0><0,b><-b,b><-b,0>}
    \Calcpoints{!polygon<0,0><0,a+b><a+b,a+b><a+b,0>}
    \Calcpoints{!polygon<a,0><a+b,a><b,a+b><0,b>}
    % Areas dos quadrados:
    \Calcpoints{!put<a/2,-a/2>{!cell{a^2}}}
    \Calcpoints{!put<-b/2,b/2>{!cell{b^2}}}
    % Areas dos triangulos:
    \Calcpoints{!put<a/4,b/4>{!cell{!frac{ab}2}}}
    \Calcpoints{!put<a+0.75*b,a/4>{!cell{!frac{ab}2}}}
    \Calcpoints{!put<a/4,b+0.75*a>{!cell{!frac{ab}2}}}
    \Calcpoints{!put<a+0.75*b,b+0.75*a>{!cell{!frac{ab}2}}}
    % Area central:
    \Calcpoints{!put<(a+b)/2,(a+b)/2>{!cell{h^2}}}
   }
   \end{picture}%
  }}%
  \quad
  \vcenter{\hbox{%
   \beginpicture(-4,-3)(7,7)%
   \pictgrid%
   %\pictaxes%
   {\linethickness{1.0pt}
    \eval{a=3; b=4}
    \Calcpoints{!polygon<0,0><a,0><a,-a><0,-a>}
    \Calcpoints{!polygon<0,0><0,b><-b,b><-b,0>}
    \Calcpoints{!polygon<0,0><0,a+b><a+b,a+b><a+b,0>}
    %\Calcpoints{!polygon<a,0><a+b,a><b,a+b><0,b>}
    % Areas dos quadrados:
    \Calcpoints{!put<a/2,-a/2>{!cell{a^2}}}
    \Calcpoints{!put<-b/2,b/2>{!cell{b^2}}}
    % Areas dos triangulos:
    %\Calcpoints{!put<a/4,b/4>{!cell{!frac{ab}2}}}
    %\Calcpoints{!put<a+0.75*b,a/4>{!cell{!frac{ab}2}}}
    %\Calcpoints{!put<a/4,b+0.75*a>{!cell{!frac{ab}2}}}
    %\Calcpoints{!put<a+0.75*b,b+0.75*a>{!cell{!frac{ab}2}}}
    % Area central:
    \Calcpoints{!put<(a+b)/2,(a+b)/2>{!cell{(a+b)^2}}}
   }
   \end{picture}%
  }}%
$$

Temos:

$h^2 + 4 \frac{ab}2 = (a+b)^2$

$h^2 + 4 \frac{ab}2 = h^2 + 2ab = (a+b)^2 = a^2+2ab+b^2$

$h^2 + 2ab = a^2+2ab+b^2$

$h^2 = a^2 + b^2$

$h = \sqrt{a^2 + b^2}$

\msk

A figura acima tem $a=3$ e $b=4$ e portanto (ééé!!!) tem $h=5$.

\msk

{\bf Exercícios}

1) Faça uma figura parecida com a acima mas com $a=1$ e $b=2$, e use-a
pra se convencer de que num triângulo retângulo com catetos de
comprimentos $1$ e $2$ a hipotenusa tem comprimento $\sqrt{5}$.

\msk

Repare que o teorema de Pitágoras nos dá um modo de calcular {\sl
  distâncias} em $\R^2$. Por exemplo, digamos que $A=(2,1)$ e
$B=(4,5)$ e que queremos calcular $d(A,B)$; basta definir $C=(4,1)$ e
calcular a hipotenusa do triângulo $ΔABC$... seus catetos têm
comprimentos $d(A,C) = d((2,1),(4,1)) = 2$ e $d(B,C) = d((4,5),(4,1))
= 4$ --- distâncias entre pontos na mesma horizontal ou na mesma
vertical são muito fáceis de calcular! --- e $d(A,C) = \sqrt{2^2+4^2}
= \sqrt{20}$.

\msk

Nos próximos exercícios suponha que $P=(a,b)$ e $Q=(c,d)$.

2) Verifique que $d(P,Q) = \sqrt{(c-a)^2+(d-b)^2}$.

3) Verifique que $d(P,Q) = \sqrt{(Q-P)·(Q-P)}$.

4) Verifique que se $P=(0,0)$ então $d(P,Q) = \sqrt{c^2+d^2}$.

5) Verifique que se $Q=(0,0)$ então $d(P,Q) = \sqrt{a^2+b^2}$.

6) Verifique que se $b=d$ então $d(P,Q) = \sqrt{(c-a)^2} = |c-a|$.

7) Mostre que nem sempre $\sqrt{(c-a)^2} = c-a$.


\newpage

%  _   _                                
% | \ | | ___  _ __ _ __ ___   __ _ ___ 
% |  \| |/ _ \| '__| '_ ` _ \ / _` / __|
% | |\  | (_) | |  | | | | | | (_| \__ \
% |_| \_|\___/|_|  |_| |_| |_|\__,_|___/
%                                       
% «normas» (to ".normas")
% (gam181p 23 "normas")
\mypsection {normas} {Comprimentos (``normas'') de vetores e ortogonalidade}

Vamos definir quatro operações novas:

$||\vv||$ é a {\sl norma} (ou o {\sl comprimento}) do vetor $\vv$.

$\Vec{PQ}$ é o {\sl vetor que vai do ponto $P$ para o ponto $Q$}.

$d(P,Q)$ é a {\sl distância do ponto $P$ ao ponto $Q$}.

$\vv⊥\ww$ é ``{\sl o vetor $\vv$ é ortogonal ao vetor $\ww$}''.

\ssk

Formalmente:

$||\vv|| = \sqrt{\vv·\vv}$,

$\Vec{PQ} = Q-P$,

$d(P,Q) = ||\Vec{PQ}||$,

$\vv⊥\ww = (\vv·\ww=0)$.

\ssk

Note que $\vv⊥\ww$ responde $𝐛V$ ou $𝐛F$. Por exemplo,

$\VEC{1,2}⊥\VEC{3,4} = (\VEC{1,2}·\VEC{3,4} = 0) = (1·3+2·4=0) = 𝐛F$, e

$\VEC{1,2}⊥\VEC{20,-10}= (20-20=0) = 𝐛V$.

\msk

{\bf Exercícios}

Calcule:

\begin{tabular}[t]{l}
1) $||\VEC{1,2}||$     \\
2) $||\VEC{3,4}||$     \\
3) $||\VEC{4,-3}||$    \\
4) $||10·\VEC{3,4}||$  \\
5) $||-10·\VEC{3,4}||$ \\
6) $||-10·\VEC{3,4}||$ \\
7) $\Vec{(2,0)(3,4)}$  \\
8) $\Vec{(2,0)((3,4)+\VEC{1,1})}$  \\
9) $d((3,4),(2,0))$  \\
10) $d((2,0),(3,4))$  \\
11) $d((2,0),(3,4)+\VEC{1,1})$  \\
12) $d((a,b),(a,b)+\VEC{c,d})$  \\
13) $\VEC{a,b}·\VEC{b,-a}$      \\
14) $\VEC{a,b}·(k·\VEC{b,-a})$  \\
15) $\VEC{a,b}⊥\VEC{b,-a}$      \\
16) $\VEC{a,b}⊥(k·\VEC{b,-a})$  \\
17) $\VEC{1,2}⊥\VEC{3,4}$  \\
\end{tabular}
\quad





\newpage

%  ____                                           _       
% |  _ \  ___ _ __ ___     ___ _ __ _ __ __ _  __| | __ _ 
% | | | |/ _ \ '_ ` _ \   / _ \ '__| '__/ _` |/ _` |/ _` |
% | |_| |  __/ | | | | | |  __/ |  | | | (_| | (_| | (_| |
% |____/ \___|_| |_| |_|  \___|_|  |_|  \__,_|\__,_|\__,_|
%                                                         
% «uma-demonstracao-errada» (to ".uma-demonstracao-errada")
% (gam181p 24 "uma-demonstracao-errada")
\mypsection {uma-demonstracao-errada} {Uma demonstração errada}

% (find-angg ".emacs" "gaq172")
% (find-angg ".emacs" "gaq172" "20170906")
% (gaq172  9 "20170906" "Demonstrações; ||kv|| = k||v||")

Sejam:
%
$$\begin{array}{rcl}
  (PE) &=& \left( ||k\vv|| = k||\vv|| \right) \; , \\
  \\
  (DE) &=& \left(
           \begin{array}{rcl}
           ||k\VEC{a,b}|| &=& ||\VEC{ka,kb}|| \\
                          &=& \sqrt{(ka)^2+(kb)^2} \\
                          &=& \sqrt{k^2a^2+k^2b^2} \\
                          &=& \sqrt{k^2(a^2+b^2)} \\
                          &=& k\sqrt{a^2+b^2} \\
                          &=& k||\VEC{a,b}|| \\
           \end{array}
           \right) .\\
  \end{array}
$$

{\bf Exercícios}

1) Verifique se $(PE)$ é verdade nos seguintes casos:

a) $(PE) \subst{k:=2 \\ \vv=\VEC{3,0}}$

b) $(PE) \subst{k:=2 \\ \vv=\VEC{3,4}}$

c) $(PE) \subst{k:=0 \\ \vv=\VEC{3,4}}$

d) $(PE) \subst{k:=-10 \\ \vv=\VEC{3,4}}$

\msk

Uma demonstração está correta quando todos os seus passos estão
corretos e quando além disso é fácil entender porque cada passo dela é
verdade. A demonstração $(DE)$ é {\sl aparentemente} uma demonstração
correta, mas o exercício abaixo mostra um modo de encontrar o passo
errado dela.

\msk

2) Calcule o valor de cada expressão entre `$=$'s em $(DE)
\subst{k:=-10 \\ a:=3 \\ b:=4\\}$ e descubra qual é o passo errado.

\msk

O melhor modo de {\sl aprender} a fazer demonstrações é {\sl começar}
com demonstrações que são só séries de igualdades, e nas quais cada
igualdade é consequência de {\sl alguma} regra que o leitor já
conhece... isso depende do seu leitor! Se você estiver escrevendo para
um ``leitor burro'' cada passo seu tem que ser uma aplicação de uma
regra só, e onde você usar uma regra mais complicada você tem que
deixar claro que regra é essa.

O melhor modo de {\sl começar} a aprender a fazer demonstrações é
escrevendo {\sl para um leitor burro} demonstrações que são só séries
de igualdades --- as dos exercícios da próxima página.


\newpage

%  ____                        _               _               
% |  _ \ _ __ ___  _ __  ___  | |__   __ _ ___(_) ___ __ _ ___ 
% | |_) | '__/ _ \| '_ \/ __| | '_ \ / _` / __| |/ __/ _` / __|
% |  __/| | | (_) | |_) \__ \ | |_) | (_| \__ \ | (_| (_| \__ \
% |_|   |_|  \___/| .__/|___/ |_.__/ \__,_|___/_|\___\__,_|___/
%                 |_|                                          
%
% «propriedades-basicas» (to ".propriedades-basicas")
% (gam181p 25 "propriedades-basicas")
\mypsection {propriedades-basicas} {Propriedades das operações básicas com pontos e vetores}

\def\eqo#1{\overset{#1}{=}}
\def\eqq{\eqo{?}}
\def\eqN{\eqo{\text{NÃO!}}}
\def\pab{(a,b)}
\def\pcd{(c,d)}
\def\pef{(e,f)}
\def\vab{\VEC{a,b}}
\def\vcd{\VEC{c,d}}
\def\vef{\VEC{e,f}}

Algumas propriedades de operações básicas como `$+$', `$-$' e `$·$'
têm nomes famosos: comutatividade, associatividade e distributividade.
Sejam:
%
$$\begin{array}{rcl}
  (CA) &=& (A+B = B+A) \\
  (CM) &=& (A·B = B·A) \\
  (CS) &=& (A-B = B-A) \\
  (AA) &=& ((A+B)+C = A+(B+C)) \\
  (AM) &=& ((A·B)·C = A·(B·C)) \\
  (AS) &=& ((A-B)-C = A-(B-C)) \\
  (DMA) &=& (A·(B+C) = A·B+A·C) \\
  (DMS) &=& (A·(B-C) = A·B-A·C) \\
  (DAM) &=& ((A+B)·C = A·C+B·C) \\
  (DSM) &=& ((A-B)·C = A·C-B·C) \\
  (AM') &=& ((A·B)·C = (A·C)·B) \\
  \end{array}
$$

Nem todas elas são verdadeiras para números --- por exemplo, $(CS)
\subst{A:=2\\B:=3}$ é falsa --- e algumas delas são verdadeiras para
números mas não para matrizes --- a p.\pageref{matrizes} tem dois
exemplos de que $(CM)$ é falsa para matrizes. Nossos primeiros
exercícios de demonstrações vão ser exercícios de ``V/F/Justifique''
adaptando as ``propriedades'' acima para as operações com pontos e
vetores.

\msk

{\bf Exemplos}

$(DMA)\subst{A:=k \\ B:=\vab \\ C:=\vcd}$ é verdadeira porque:
%
$$\def\pr#1{\text{(pela regra #1 da p.\pageref{pontos-e-vetores})}}
  \begin{array}{rclcl}
  k·(\vab+\vcd) &=& k·\VEC{a+c,b+d}           && \pr2 \\
                &=& \VEC{k(a+c),k(b+d)}       && \pr6 \\
                &=& \VEC{ka+kc,kb+kd}         && \\
                &=& \VEC{ka,kb} + \VEC{kc,kd} && \pr2 \\
                &=& k\vab + k\vcd             && \pr6 \\
  \end{array}
$$

$(CS)\subst{A:=\pab \\ B:=\vcd}$ é falsa porque $\pab + \vcd =
  (a+c,b+d)$ mas

$\vcd+\pab = \text{erro}$.

\msk

{\bf Exercícios}

(V/F/justifique; use as dicas da próxima página)

\begin{tabular}[t]{l}
1) $(CA)\subst{A:=\vab \\ B:=\vcd \\}$            \\
2) $(AA)\subst{A:=\pab \\ B:=\vcd \\ C:=\vef \\}$ \\
3) $(AA)\subst{A:=\vab \\ B:=\vcd \\ C:=\vef \\}$ \\
\end{tabular}
\quad
\begin{tabular}[t]{l}
4) $(AM) \subst{A:=\vab \\ B:=\vcd \\ C:=\vef \\}$ \\
5) $(AM')\subst{A:=\vab \\ B:=\vcd \\ C:=\vef \\}$ \\
6) $(AM) \subst{A:=k    \\ B:=\vab \\ C:=\vcd \\}$ \\
\end{tabular}
\quad
\begin{tabular}[t]{l}
7) $(DMA) \subst{A:=k    \\ B:=\vab \\ C:=\vcd \\}$ \\
8) $(DMA) \subst{A:=\vab \\ B:=\vcd \\ C:=\vef \\}$ \\
9) $(DAM) \subst{A:=a    \\ B:=b    \\ C:=\vcd \\}$ \\
\end{tabular}




\newpage

% __     _________ __  _           _   _  __ _                  
% \ \   / / /  ___/ / | |_   _ ___| |_(_)/ _(_) __ _ _   _  ___ 
%  \ \ / / /| |_ / /  | | | | / __| __| | |_| |/ _` | | | |/ _ \
%   \ V / / |  _/ / |_| | |_| \__ \ |_| |  _| | (_| | |_| |  __/
%    \_/_/  |_|/_/ \___/ \__,_|___/\__|_|_| |_|\__, |\__,_|\___|
%                                                 |_|           
%
% «dicas-V-F-justifique» (to ".dicas-V-F-justifique")
% (gam181p 26 "dicas-V-F-justifique")
\mypsection {dicas-V-F-justifique} {Dicas para problemas de ``V/F/Justifique''}

1) Releia o item 7 da p.\pageref{dicas}. Você vai ter que aprender
a reler as suas próprias demonstrações fazendo o papel de ``leitor
burro''.

2) O modo mais fácil de demonstrar que uma proposição é {\sl falsa} é
dando um contra-exemplo pra ela --- pra mostrar que uma proposição não
é verdadeira {\sl sempre} basta mostrar {\sl um caso} em que ela é
falsa! Por exemplo:
%
$$\left(\sqrt{a}+\sqrt{b} = \sqrt{a+b}\right) \subst{a:=9\\b:=16}$$

3) Em problemas que dão uma proposição e dizem ``V/F/Justifique'' você
vai ter que primeiro decidir se a proposição é verdadeira ou falsa e
depois demonstrar se ela é verdadeira (por uma série de igualdades) ou
se ela é falsa (por contra-exemplo). {\sl Note que a técnica pra
  demonstrar que uma proposição é verdadeira é totalmente diferente da
  técnica pra mostrar que ela é falsa!}

4) Releia cada demonstração de que uma proposição é verdadeira e faça
anotações nela --- por exemplo, escreva um `?' em cada `$=$' que não é
{\sl muito} claro para um leitor burro (`$=$' $→$ `$\eqq$') e escreva
um `NÃO!' em cada `$=$' que parece estar usando uma regra errada. Por
exemplo:
%
$$||\vv||(\sqrt{a}+\sqrt{b}) \eqN ||\vv||\sqrt{a+b}$$

5) Algumas pessoas tentam ``demonstrar'' proposições só
``traduzindo-as pro português'' e aí acreditando que a versão em
português da proposição é ``óbvia''. {\sl Não seja como estas
  pessoas!} Neste ponto do curso ``demonstrações'' feitas em português
estão ERRADAS!

6) Aprenda a fazer demonstrações ``em matematiquês'' usando a notação
adequada e fazendo com que cada passo da sua demonstração seja uma
aplicação de alguma regra conhecida e se possível de alguma regra com
nome, {\sl ou senão eu vou reprovar você com o maior sorriso de orelha
  a orelha que você já viu.} DEPOIS nós vamos ver como reescrever em
português algumas partes das demonstrações desta parte do curso ---
mas repare: ``{\sl algumas partes}'' e ``{\sl depois}''!

7) O ``matematiquês'' permite algumas palavras em português, como
``seja'', ``se'', ``então'' e ``supondo''.

8) A notação de substituição simultânea da p.\pageref{substituicao}
não é usada em nenhum livro básico que eu conheça... se você for
comparar a notação daqui com a dos livros de GA recomendados pro curso
você vai ver que eles usam expressões em português pra indicar
substituição --- por exemplo, ``substituindo $k$ por $-10$, $a$ por
$3$ e $b$ por $4$ na demonstração $(DE)$ temos ...''.



\newpage


%  ____                        _               _                 ____  
% |  _ \ _ __ ___  _ __  ___  | |__   __ _ ___(_) ___ __ _ ___  |___ \ 
% | |_) | '__/ _ \| '_ \/ __| | '_ \ / _` / __| |/ __/ _` / __|   __) |
% |  __/| | | (_) | |_) \__ \ | |_) | (_| \__ \ | (_| (_| \__ \  / __/ 
% |_|   |_|  \___/| .__/|___/ |_.__/ \__,_|___/_|\___\__,_|___/ |_____|
%                 |_|                                                  
%
% «propriedades-basicas-2» (to ".propriedades-basicas-2")
% (gam181p 27 "propriedades-basicas-2")
\mypsection {propriedades-basicas-2} {Propriedades das operações básicas com pontos e vetores (2)}

As propriedades da p.\pageref{propriedades-basicas} podem ser postas
numa forma mais curta.

Por exemplo:

\msk

$k·(\uu+\vv) = k\uu + k\vv$ é sempre verdade para $k∈\R$ e $\uu,\vv$
vetores em $\R^2$.

Demonstração. Sejam $\uu=\vab$ e $\vv=\vcd$. Então:
%
$$\def\pr#1{\text{(pela regra #1 da p.\pageref{pontos-e-vetores})}}
  \begin{array}{rclcl}
  k·(\uu+\vv)   &=& k·(\vab+\vcd)             &&      \\
                &=& k·\VEC{a+c,b+d}           && \pr2 \\
                &=& \VEC{k(a+c),k(b+d)}       && \pr6 \\
                &=& \VEC{ka+kc,kb+kd}         &&      \\
                &=& \VEC{ka,kb} + \VEC{kc,kd} && \pr2 \\
                &=& k\vab + k\vcd             && \pr6 \\
                &=& k\uu + k\vv               &&      \\
  \end{array}
$$

\msk

Normalmente a gente usa uma convenção que diz que as letras
$a,b,c,k,x,y$ representam números reais, $P,Q,R$ representam pontos em
$\R^2$ e $\uu,\vv,\ww$ representam vetores em $\R^2$, mas essa
convenção muda de acordo com o contexto --- daqui a pouco quando
introduzirmos círculos o $R$ vai passar a denotar o raio de um
determinado círculo e vai passar a ser um número, e quando passarmos
para $\R^3$ as letras $P,Q,R$ vão passar a denotar pontos de $\R^3$ e
$\uu$, $\vv$, $\ww$ vão passar a ser vetores em $\R^3$.

\msk

{\bf Exercícios}

V/F/Justifique:

\begin{tabular}[t]{l}
1) $P+\vv = \vv+P$       \\
2) $\vv+\ww = \ww+\vv$   \\
3) $P+(\vv+\ww) = (P+\vv)+\ww$ \\
4) $(\uu·\vv)·\ww = \uu·(\vv·\ww)$ \\
5) $(\uu·\vv)·\ww = (\uu·\ww)·\vv$ \\
6) $(a+b)·\uu = a\uu + b\uu$ \\
7) $(\uu+\vv)·\ww = \uu·\ww + \vv·\ww$ \\
\end{tabular}

\msk

Observação MUITO importante: se você estiver escrevendo para um leitor
que tem acesso às demonstrações que você fez na
p.\pageref{propriedades-basicas} você vai poder encurtar as suas
demonstrações desta página bastante --- você pode usar como
justificativa de um passo algo como ``pela demonstração do exercício 9
da p.\pageref{propriedades-basicas}, com {\sl [...e aí aqui você
    indica a substituição necessária]}''. Ah, e quando você estiver
escrevendo pra leitores menos burros você {\sl às vezes} vai poder
omitir qual é a substituição --- mas só com bastante treino a gente
aprende o que a gente pode omitir sem perder a clareza.











\newpage

%  ____                                                             
% |  _ \ _ __ ___  _ __  ___   _ __   ___  _ __ _ __ ___   __ _ ___ 
% | |_) | '__/ _ \| '_ \/ __| | '_ \ / _ \| '__| '_ ` _ \ / _` / __|
% |  __/| | | (_) | |_) \__ \ | | | | (_) | |  | | | | | | (_| \__ \
% |_|   |_|  \___/| .__/|___/ |_| |_|\___/|_|  |_| |_| |_|\__,_|___/
%                 |_|                                               
%
% «propriedades-de-normas» (to ".propriedades-de-normas")
% (gam181p 28 "propriedades-de-normas")
\mypsection {propriedades-de-normas} {Propriedades de normas e distâncias}

{\bf Exercícios}

1) V/F/justifique. Cada um dos itens abaixo pode ser feito ``abrindo
os vetores'', isto é, começando com algo como ``digamos que $\uu=\vab$
e $\vv=\vcd$'', mas também pode ser feito usando propriedades ``em
forma mais curta'' como as da p.\pageref{propriedades-basicas-2}.
Dica: o (1e) tem uma solução {\sl bem} curta se você souber invocar a
propriedade certa.

a) $\uu·\vv=||\uu||\,||\vv||$

b) $||\uu+\vv||^2 = ||\uu||^2 + 2\uu·\vv + ||\vv||^2$.

c) $||\uu+\vv||^2 + ||\uu+\vv||^2 = 2(||\uu||^2 + ||\vv||^2)$.

c') $||\uu+\vv||^2 + ||\uu-\vv||^2 = 2(||\uu||^2 + ||\vv||^2)$.

d) $||\uu+\vv||^2 + ||\uu-\vv||^2 = 4\uu·\vv$.

d') $||\uu+\vv||^2 - ||\uu-\vv||^2 = 4\uu·\vv$.

e) $||\,||\uu||\,\vv\,|| = ||\,||\vv||\,\uu\,||$


\msk

2) V/F/justifique. Aprenda a lidar com proposição com hipóteses (os
``se''s) e use um pouco de criatividade.

a) Se $α\uu+β\vv=\vec0$ então $α=0$ e $β=0$.

b) Se $||\uu|| = ||\vv||$ então $(\uu-\vv)·(\uu+\vv)=0$.

c) Se $\uu ≠ \vec0$ e $\uu·\vv=\uu·\ww$ então $\vv=\ww$.

d) Existe uma reta que contém os pontos $A=(1,3)$, $B=(-1,2)$ e $C=(5,4)$. 

d') Existe uma reta que contém os pontos $A=(1,3)$, $B=(-1,2)$ e $C=(5,5)$. 

e) O triângulo com vértices $A=(1,0)$, $B=(0,2)$ e $C=(-2,1)$ é retângulo. 

e') O triângulo com vértices $A=(1,0)$, $B=(0,2)$ e $C=(-4,0)$ é retângulo. 

f) Todo vetor em $\R^2$ é combinação linear de $\uu=\VEC{2,1}$ e $\vv=\VEC{4,2}$. 


\bsk

Obs: quase todos os exercícios desta página faziam parte da primeira
lista de exercícios de GA de um curso daqui do PURO de alguns anos
atrás.

% (find-LATEXfile "2016-1-GA-material.tex" "primeira lista do Reginaldo")

% \Reg{2m} Se $\uu \vec0$, $\vv \vec0$ e $\Pr_{\vv}\uu = \vec0$ então $\uu \vv$.

% \Reg{2b} Seja $ABCD$ um quadrilátero...


\newpage


%  ____                                         _ 
% |  _ \  ___ _ __ ___     ___ ___  _ __ ___   / |
% | | | |/ _ \ '_ ` _ \   / __/ _ \| '_ ` _ \  | |
% | |_| |  __/ | | | | | | (_| (_) | | | | | | | |
% |____/ \___|_| |_| |_|  \___\___/|_| |_| |_| |_|
%                                                 
% «demonstracao-comentada» (to ".demonstracao-comentada")
% (gam181p 29 "demonstracao-comentada")
\mypsection {demonstracao-comentada} {Uma demonstração (comentada)}

% (gam181p 25 "propriedades-basicas")
% (gam181     "propriedades-basicas")
% (gam181     "propriedades-basicas" "4) $(AM)")
% (gaq181  9 "20180418" "Demonstrações, introdução ao Pr")

Em 18/abril eu mostrei no quadro como eu faria a demonstração do

exercício 4 da p.\pageref{propriedades-basicas}... a minha
demonstração seria assim:

\bsk

Queremos ver se esta proposição é sempre verdadeira:
%
$$(AM) \subst{A:=\vab \\ B:=\vcd \\ C:=\vef \\}$$

Repare que esta proposição é:
%
$$((A·B)·C = A·(B·C)) \subst{A:=\vab \\ B:=\vcd \\ C:=\vef \\}$$
%
que é:
%
$$\qquad\qquad (\vab·\vcd)·\vef = \vab·(\vcd·\vef)  \qquad\qquad (\bigstar)$$

Calculando o lado esquerdo de $(\bigstar)$, temos:
%
$$\begin{array}{rcl}
  (\vab·\vcd)·\vef &=& (ac+bd)·\vef \\
                   &=& \VEC{(ac+bd)e, (ac+bd)f}, \\
  \end{array}
$$
%
que dá um vetor, e o lado direito de $(\bigstar)$ dá:
%
$$\begin{array}{rcl}
  \vab·(\vcd·\vef) &=& \vab·(\und{ce+df}{\text{número!}}) \\
                   &=& \erro, \\
  \end{array}
$$
%
portanto a igualdade $(\bigstar)$ é falsa --- o lado esquerdo dela dá
um vetor, e o lado direito dá erro.

\bsk

Esse exercício é um ``V/F/Justifique'' sobre algo que num primeiro
momento nem parece uma proposição. Eu começo a demonstração dando a
entender, com o ``Queremos ver se esta proposição...'' que a expressão
$(AM)[\ldots]$ é uma proposição ``disfarçada''. Os passos ``Repare que
esta proposição é'' e ``que é:'' reescrevem a expressão $(AM)[\ldots]$
até ela virar algo que é claramente uma proposição sobre vetores, e
que eu nomeio como ``$(\bigstar)$''. Até esse momento eu não dei
nenhum indício pro leitor se a $(AM)[\ldots]$, que é equivalente a
$(\bigstar)$, é verdadeira ou falsa; aí eu mostro como calcular o lado
esquerda da $(\bigstar)$, depois como calcular o lado direito dela, e
mostro que o resultado do lado esquerdo é {\sl sempre} diferente do do
lado direito --- o que neste caso é mais fácil do que encontrar um
contra-exemplo.

Repare que a demonstração fica bem mais curta e clara com o truque de
dar um nome para a proposição $(\bigstar)$ --- eu avisei lá na dica 3
da p.\pageref{dicas} que era útil aprender a nomear objetos. $=)$


\newpage

%  ____       
% |  _ \ _ __ 
% | |_) | '__|
% |  __/| |   
% |_|   |_|   
%             
% «projecao-ortogonal» (to ".projecao-ortogonal")
% (gam181p 30 "projecao-ortogonal")
\mypsection {projecao-ortogonal} {Projeção ortogonal}

Digamos que $λ\uu+\vv = \ww$. Então $\vv=\ww-λ\uu$.

Vamos fazer duas figuras disto para o caso em que $\uu=\VEC{2,0}$ e
$\ww=\VEC{5,3}$.
%
$$\unitlength=15pt
  %
  \begin{array}{c}
    \text{Se $λ=2$,} \\
    %
    \vcenter{\hbox{%
     \beginpicture(0,0)(7,5)%
     \pictgrid%
     %\pictaxes
     {\linethickness{1.0pt}
      \Vector(1,1)(6,4)     \put(3,3){\cell{\ww}}
      \Vector(1,1)(5,1)     \put(4,0.2){\cell{λ\uu}}
      \Vector(5,1)(6,4)     \put(6,2.2){\cell{\vv}}
      \Vector(1,0.7)(3,0.7) \put(2,0){\cell{\uu}}   
     }
     \end{picture}%
    }}%
  \end{array}
  %
  \qquad
  %
  \begin{array}{c}
    \text{Se $λ=3$,} \\
    %
    \vcenter{\hbox{%
     \beginpicture(0,0)(8,5)%
     \pictgrid%
     %\pictaxes
     {\linethickness{1.0pt}
      \Vector(1,1)(6,4)     \put(3,3){\cell{\ww}}   
      \Vector(1,1)(7,1)     \put(5,0.2){\cell{λ\uu}}
      \Vector(7,1)(6,4)     \put(7,2.2){\cell{\vv}} 
      \Vector(1,0.7)(3,0.7) \put(2,0){\cell{\uu}}   
     }
     \end{picture}%
    }}%
  \end{array}
$$

Repare que nenhum dos triângulos acima é retângulo.

Como podemos encontrar o $λ$ que faça $\ang(λ\uu,\vv)=90°$?

\msk

Suponha que $\uu$ e $\ww$ estão dados e que $\uu≠\Vec0$.

Se as nossas condições são $λ\uu+\vv = \ww$ e $\uu⊥\vv$ então temos $\vv=\ww-λ\uu$,

$λ\uu⊥\vv$, e só vai existir um valor $λ∈\R$ que obedece estas condições.
Veja:
%
$$\begin{array}{rcl}
  \vv &=& \ww-λ\uu, \\
  \uu &⊥& \ww-λ\uu, \\
  \uu · (\ww-λ\uu) &=& 0, \\
  \uu·\ww - \uu·λ\uu &=& 0, \\
  \uu·\ww - λ(\uu·\uu) &=& 0, \\
  \uu·\ww &=& λ(\uu·\uu), \\
  \frac{\uu·\ww}{\uu·\uu} &=& λ \\
  \end{array}
$$


\msk

{\bf Uma nova operação: a projeção ortogonal, $\Pr$}

Notação: $\Pr_\uu \ww$

Pronúncia: $\Pr_\uu \ww$ é ``projeção {\sl sobre} $\uu$ de $\ww$''.

{\sl Bem} informalmente, é como se os raios do sol fossem ortogonais a
$\uu$, e o

sol projeta uma ``sombra'' do vetor $\ww$ sobre o
prolongamento do vetor $\uu$.

Definição 1 (fácil de calcular): $\Pr_\uu \ww = \frac{\uu·\ww}{\uu·\uu} \uu$. 

Definição 2 (fácil de visualizar): $\Pr_\uu \ww$ é o múltiplo $λ\uu$
do vetor $\uu$ que faz

com que $\uu⊥\vv$ quando $\vv$ é o vetor que obedece $\ww=λ\uu+\vv$.


\msk

{\bf Exercícios}

1) Em cada um dos casos abaixo calcule $λ$, $\Pr_\uu \ww$ e $\vv$ e represente

graficamente $\ww = λ\uu + \vv$.

$\begin{tabular}{lll}
 a) $\uu=\VEC{3,0}$, $\ww=\VEC{-1,2}$ && c) $\uu=\VEC{3,1}$, $\ww=\VEC{1,3}$ \\
 b) $\uu=\VEC{2,2}$, $\ww=\VEC{0,1}$  &&  \\
 \end{tabular}
$

2) Calcule e represente graficamente:

$\begin{tabular}{lll}
   a) $\Pr_{\VEC{3,1}} \VEC{0,2}$ && c) $\Pr_{\VEC{3,1}} \VEC{3,2}$ \\
   b) $\Pr_{\VEC{3,1}} \VEC{3,0}$ && \\
 \end{tabular}
$






\newpage

%  ____            _
% |  _ \ _ __ ___ (_) ___  ___ ___   ___  ___
% | |_) | '__/ _ \| |/ _ \/ __/ _ \ / _ \/ __|
% |  __/| | | (_) | |  __/ (_| (_) |  __/\__ \
% |_|   |_|  \___// |\___|\___\___/ \___||___/
%               |__/
%
% «projecoes-no-olhometro» (to ".projecoes-no-olhometro")
% (gam181p 31 "projecoes-no-olhometro")
% (gam172p 25 "projecoes")
% (gam172     "projecoes")
\mypsection {projecoes-no-olhometro} {Projeções no olhômetro}

Quando a gente tem um pouco de prática com o ``significado
geométrico'' da operação $\Pr$ a gente consegue 1) visualizar $\Pr_\uu
\ww$, 2) calcular exatamente o resultado de $\Pr_\uu \ww$ ``no
olhômetro'' em casos simples. Os exercícios abaixo são pra você
melhorar a sua capacidade de calcular projeções ortogonais no
olhômetro; lembre que toda vez que você tiver dúvidas você pode
recorrer às contas.

\ssk

{\bf Exercícios}

1) Sejam $\ww = \V(3,4)$, $\uu = \V(0,1)$, $A=(2,0)$, $B=A+\ww$.
Represente graficamente $A$, $B$, $\uu$, $\ww$, e para cada
$λ∈\{1,2,3,4,5\}$ desenhe no seu gráfico o triângulo $\ww = λ\uu+\vv$
correspondente e calcule $\vv$ e $\uu·\vv$. Qual o $λ$ que faz com que
$\uu⊥\vv$?

2) Faça a mesma coisa que no (1), mas mudando o $\uu$ para
$\uu=\V(1,1)$.

3) Digamos que $\Pr_{\uu} \ww_1 = λ_1 \uu$, $\Pr_{\uu} \ww_2 = λ_2
\uu$, etc. Determine $λ_1$, $λ_2$, etc na figura abaixo à esquerda.

4) Digamos que $\Pr_{\uu} \ww_1 = λ_1 \uu$, $\Pr_{\uu} \ww_2 = λ_1
\uu$, etc. Determine $λ_1$, $λ_2$, etc na figura abaixo à direita.

%L p = function (a, b) return O + a*uu + b*vv end
%L O, uu, vv = v(1, 1), v(2, 0), v(0, 2)

%L myvec = function (a, b, label)
%L     local bprint, out = makebprint()
%L     local AA, BB = p(0,0), p(a,b)
%L     local AB = BB-AA
%L     local CC = BB + AB:unit(0.7)
%L     local f = function (str) return (str:gsub("!", "\\")) end
%L     bprint("\\Vector%s%s", AA, BB)
%L     bprint("\\put%s{\\cell{%s}}", CC, f(label))
%L     return out()
%L   end
\pu

% (find-LATEX "edrxgac2.tex" "pict2e")

$\unitlength=10pt
 \def\closeddot{\circle*{0.2}}
 \def\cellfont{\scriptsize}
 \def\cellfont{}
 \vcenter{\hbox{%
   \beginpicture(-7,-7)(9,9)%
   %\pictgrid%
   {\color{GrayPale}\expr{pictpgrid(-3,-3,3,3)}}%
   \pictaxes%
   {\linethickness{1.0pt}
    \expr{myvec(2, 0, "!uu")}
    \expr{myvec(3, 1, "!ww_1")}
    \expr{myvec(3, 2, "!ww_2")}
    \expr{myvec(3, 3, "!ww_3")}
    \expr{myvec(2, 3, "!ww_4")}
    \expr{myvec(1, 3, "!ww_5")}
    \expr{myvec(0, 3, "!ww_6")}
    \expr{myvec(-1, 3, "!ww_7")}
    \expr{myvec(-2, 3, "!ww_8")}
    \expr{myvec(-3, 3, "!ww_9")}
    \expr{myvec(-3, 2, "!ww_{10}")}
    \expr{myvec(-3, 1, "!ww_{11}")}
    \expr{myvec(-3, 0, "!ww_{12}")}
    \expr{myvec(-3, -1, "!ww_{13}")}
    \expr{myvec(-3, -2, "!ww_{14}")}
    \expr{myvec(-3, -3, "!ww_{15}")}
    \expr{myvec(-2, -3, "!ww_{16}")}
    \expr{myvec(-1, -3, "!ww_{17}")}
    \expr{myvec(-0, -3, "!ww_{18}")}
   }
   \end{picture}%
  }}%
  \quad
  %
%L O, uu, vv = v(1, 1), v(1, 1), v(-1, 1)
  \pu
  %
  \unitlength=12pt
  \vcenter{\hbox{%
   \beginpicture(-7,-6)(7,8)%
   %\pictgrid%
   {\color{GrayPale}\expr{pictpgrid(-3,-3,3,3)}}%
   \pictaxes%
   {\linethickness{1.0pt}
    \expr{myvec(2, 0, "!uu")}
    \expr{myvec(3, 1, "!ww_1")}
    \expr{myvec(3, 2, "!ww_2")}
    \expr{myvec(3, 3, "!ww_3")}
    \expr{myvec(2, 3, "!ww_4")}
    \expr{myvec(1, 3, "!ww_5")}
    \expr{myvec(0, 3, "!ww_6")}
    \expr{myvec(-1, 3, "!ww_7")}
    \expr{myvec(-2, 3, "!ww_8")}
    \expr{myvec(-3, 3, "!ww_9")}
    \expr{myvec(-3, 2, "!ww_{10}")}
    \expr{myvec(-3, 1, "!ww_{11}")}
    \expr{myvec(-3, 0, "!ww_{12}")}
    \expr{myvec(-3, -1, "!ww_{13}")}
    \expr{myvec(-3, -2, "!ww_{14}")}
    \expr{myvec(-3, -3, "!ww_{15}")}
    \expr{myvec(-2, -3, "!ww_{16}")}
    \expr{myvec(-1, -3, "!ww_{17}")}
    \expr{myvec(-0, -3, "!ww_{18}")}
   }
   \end{picture}%
  }}%
$

\bsk

5) Sejam $A=(1,1)$, $B=(3,1)$, $C=(4,4)$.

Calcule e represente graficamente:

$\begin{tabular}{l}
 AB) $P = A + \Pr_{\Vec{AB}} \Vec{AC}$ \\
 AC) $Q = A + \Pr_{\Vec{AC}} \Vec{AB}$ \\
 BA) $R = B + \Pr_{\Vec{BA}} \Vec{BC}$ \\
 \end{tabular}
 %
 \qquad
 %
 \begin{tabular}{l}
 BC) $S = B + \Pr_{\Vec{BC}} \Vec{BA}$ \\
 CA) $T = C + \Pr_{\Vec{CA}} \Vec{CB}$ \\
 CB) $U = C + \Pr_{\Vec{CB}} \Vec{CA}$ \\
 \end{tabular}
 %
 \qquad
 \quad
 %
 \unitlength=10pt
 \def\closeddot{\circle*{0.2}}
 \vcenter{\hbox{%
   \beginpicture(0,0)(5,5)%
   \pictgrid
   % {\color{GrayPale}\expr{pictpgrid(-3,-3,3,3)}}%
   \pictaxes%
   {\linethickness{1.0pt}
    \polygon(1,1)(3,1)(4,4)
    \put(1,1.6){\cell{A}}
    \put(3.8,0.7){\cell{B}}
    \put(4.8,3.7){\cell{C}}
   }
   \end{picture}%
  }}%
$


\bsk

6) Leia a p.55 do livro do CEDERJ. Compare a abordagem dele com a nossa.

7) Leia as págs 35 a 38 do Reis/Silva. Compare a abordagem dele com a
nossa.

Repare que tanto o livro do CEDERJ quanto o do Reis/Silva começam a
mencionar senos, cossenos e tangentes bem antes de definirem o produto
$\uu·\vv$!



% (find-GA1page (+ -2 55) "Projecao ortogonal")

% (find-reissilvapage (+ -14 35) "2.7 Projeção de vetores")


\newpage

%  ____                        ____       
% |  _ \ _ __ ___  _ __  ___  |  _ \ _ __ 
% | |_) | '__/ _ \| '_ \/ __| | |_) | '__|
% |  __/| | | (_) | |_) \__ \ |  __/| |   
% |_|   |_|  \___/| .__/|___/ |_|   |_|   
%                 |_|                     
%
% «propriedades-da-projecao» (to ".propriedades-da-projecao")
% (gam181p 32 "propriedades-da-projecao")
% (gam172p 29 "propriedades-do-Pr")
% (gaq171 14 "20170426" "Propriedades do Pr")
\mypsection {propriedades-da-projecao} {Propriedades da projeção}

{\bf Exercícios}

\ssk

1) V/F/Justifique:

\begin{tabular}[t]{l}
a) (\;\;) $\Pr_\uu(\vv+\ww) = \Pr_\uu \vv + \Pr_\uu \ww$      \\
b) (\;\;) $\Pr_{(\uu+\vv)} \ww = \Pr_\uu \ww + \Pr_\vv \ww$   \\
c) (\;\;) $\Pr_\uu \ww = \Pr_\ww \uu$                         \\
d) (\;\;) $\Pr_{(k\uu)} \ww = k \, \Pr_\uu \ww$               \\
e) (\;\;) $\Pr_{(k\uu)} \ww = |k| \, \Pr_\uu \ww$             \\
f) (\;\;) $\Pr_{(k\uu)} \ww = \Pr_\uu \ww$                    \\
g) (\;\;) $\Pr_\uu (k\ww) = k \, \Pr_\uu \ww$                 \\
h) (\;\;) $\Pr_\uu (k\ww) = |k| \, \Pr_\uu \ww$               \\
i) (\;\;) $\Pr_\uu (k\ww) = \Pr_\uu \ww$                      \\
\end{tabular}
\qquad
\begin{tabular}[t]{l}
j) (\;\;) $||k\vv|| = k||\vv||$                    \\
k) (\;\;) $||k\vv|| = |k|\,||\vv||$                \\
l) (\;\;) $||k\vv|| = ||\vv||$                     \\
m) (\;\;) Se $ab = ac$ então $b=c$                 \\
n) (\;\;) Se $a\uu = b\uu$ então $a=b$             \\
o) (\;\;) Se $a\uu = a\vv$ então $\uu=\vv$         \\
p) (\;\;) Se $\uu·\vv = \uu·\ww$ então $\vv=\ww$   \\
\end{tabular}

\msk

2) Demonstre que se $\vv$ e $\ww$ são não-nulos e $\vv⊥\ww$ então:

\begin{tabular}[t]{l}
a) $\Pr_\vv (k\ww) = \Vec 0$ \\
b) $\Pr_\vv (k\vv) = k\vv$ \\
c) $\Pr_\vv (a\vv+b\ww) + \Pr_\ww (a\vv+b\ww) = a\vv+b\ww$ \\
\end{tabular}

\msk

3) Demonstre:

\begin{tabular}[t]{l}
a) Se $\uu⊥\vv$ então $||\uu+\vv||=||\uu-\vv||$ \\
b) Se $\uu⊥\vv$ então $||\uu+\vv||^2=||\uu||^2+||\vv||^2$ \\
c) Se $||\uu+\vv||=||\uu-\vv||$ então $\uu⊥\vv$ \\
d) Se $||\uu+\vv||^2=||\uu||^2+||\vv||^2$ então $\uu⊥\vv$ \\
\end{tabular}

\msk

4) Demonstre:

\begin{tabular}[t]{l}
a) Se $\uu⊥\vv$ então $||\uu||^2 ≤ ||\uu+\vv||^2$ \\
b) Se $\uu⊥\vv$ então $||\uu||   ≤ ||\uu+\vv||$ \\
c) Se $\uu⊥\vv$ e $\vv≠\Vec0$ então $||\uu||^2 < ||\uu+\vv||^2$ \\
d) Se $\uu⊥\vv$ e $\vv≠\Vec0$ então $||\uu||   < ||\uu+\vv||$ \\
\end{tabular}

\msk

5) Digamos que $r=\setofst{A+t\uu}{t∈\R}$ seja uma reta, que $B$ seja
um ponto

de $\R^2$ e que $\uu⊥\Vec{AB}$. Seja $\vv=\Vec{AB}$. Demonstre que:

\begin{tabular}[t]{l}
a) $d(A+t\uu,B)$ é mínimo quando $d(A+t\uu,B)^2$ é mínimo \\
b) $d(A+t\uu,B)^2 = ||\Vec{AB}-t\uu||^2$ \\
c) $||\Vec{AB}-t\uu||^2 = ||\vv||^2+||t\uu||^2$ \\
d) $||\Vec{AB}-t\uu||^2 = ||\vv||^2 + t^2 ||\uu||^2$ \\
\end{tabular}



\newpage


%  ____                                    
% / ___|  ___ _ __     ___    ___ ___  ___ 
% \___ \ / _ \ '_ \   / _ \  / __/ _ \/ __|
%  ___) |  __/ | | | |  __/ | (_| (_) \__ \
% |____/ \___|_| |_|  \___|  \___\___/|___/
%                                          
% «senos-e-cossenos» (to ".senos-e-cossenos")
% (gam181p 32 "senos-e-cossenos")
\mypsection {senos-e-cossenos} {Senos e cossenos}

Vamos começar com uma revisão rápida de graus e radianos...

Lembre que $180°=π$ e que radianos são ``adimensionais'' --- a gente
escreve ``$π$ radianos'' só como ``$π$''. 

O símbolo ``${}°$'' pode ser interpretado como uma multiplicação por
uma determinada constante: $180°=π$, $90°=π/2$, $45°=π/4$, $1°=π/180$,
$234°=234 \frac{π}{180}$, $x°=x\frac{π}{180}$.

Vou usar a expressão ``nossos ângulos preferidos'' pra me referir aos
ângulos que têm senos e cossenos fáceis de lembrar e de calcular.
Formalmente,
%
$$\begin{array}{rcl}
  A &=& \setofst{k·90°+a}{k∈\{0,1,2,3\}, a∈\{0°, 30°, 45°, 60°\}} \\
    &=& \{0°, 30°, 45°, 60°, 90°, 120°, \ldots\}
  \end{array}
$$

Algumas pessoas viram a tabela à esquerda abaixo no ensino médio...
%
$$\begin{array}{cccc}
    θ  & (\cosθ,\senθ) \\\hline
    0° & (√4/2,√0/2) \\
   30° & (√3/2,√1/2) \\
   45° & (√2/2,√2/2) \\
   60° & (√1/2,√3/2) \\
   90° & (√0/2,√4/2) \\
  120° & (-√1/2,√3/2) \\
  135° & (-√2/2,√2/2) \\
  150° & (-√3/2,√1/2) \\
  180° & (-√4/2,√0/2) \\
  210° & (-√3/2,-√1/2) \\
  225° & (-√2/2,-√2/2) \\
  240° & (-√1/2,-√3/2) \\
  270° & (√0/2,-√4/2) \\
  300° & (√1/2,-√3/2) \\
  315° & (√2/2,-√2/2) \\
  330° & (√3/2,-√1/2) \\
  360° & (√4/2,√0/2) \\
  \end{array}
  %
  \qquad
  %
  \unitlength=80pt
  \def\closeddot{\circle*{0.08}}
  \def\putc(#1,#2){\put(#1,#2){\closeddot}}
  \vcenter{\hbox{%
   \beginpicture(-1,-1)(1,1)%
   \pictgrid%
   \pictaxes%
   \putc( 1    ,  0    )
   \putc( 0.965,  0.259)
   \putc( 0.866,  0.5  )
   \putc( 0.707,  0.707)
   \putc( 0.5,    0.866)
   \putc( 0.259,  0.965)
   \putc( 0,      1    )
   \putc(-0.259,  0.965)
   \putc(-0.5,    0.866)
   \putc(-0.707,  0.707)
   \putc(-0.866,  0.5  )
   \putc(-0.965,  0.259)
   \putc(-1,      0    )
   \putc(-0.965, -0.259)
   \putc(-0.866, -0.5  )
   \putc(-0.707, -0.707)
   \putc(-0.5,   -0.866)
   \putc(-0.259, -0.965)
   \putc( 0,     -1    )
   \putc( 0.259, -0.965)
   \putc( 0.5,   -0.866)
   \putc( 0.707, -0.707)
   \putc( 0.866, -0.5  )
   \putc( 0.965, -0.259)
   {\linethickness{1.0pt}
   }
   \end{picture}%
  }}%
$$

% (cos (/ pi 12))
% (sin (/ pi 12))
% (* 0.5 (sqrt 2))
% (* 0.5 (sqrt 3))

{\bf Exercícios}

Dicas: $√0/2 = 0$, $√1/2 = 0.5$, $√4/2 = 1$, e use as aproximações
$√2/2 ≈ 0.7$ e $√3/2 ≈ 0.85$. Nas contas com os ``nossos ângulos
preferidos'' comece sempre com os múltiplos de $90°$ --- em que as
contas são facílimas ---, depois inclua os múltiplos de $45°$, e só
depois inclua os múltiplos de $30°$.

1) Identifique cada ponto da forma $(\cosθ,\senθ)$, onde $θ∈A$, com
pontos da figura à direita acima.

2) Verifique que o triângulo $Δ(0,0)(√2/2,0)(√2/2,√2/2)$ é retângulo,
isósceles e tem hipotenusa 1.

3) Verifique que o triângulo $Δ(0,0)(1,0)(0.5,√3/2)$ é equilátero.

4) Verifique que o triângulo $Δ(0,0)(0.5,0)(0.5,√3/2)$ é retângulo,
com hipotenusa 1, e um dos seus catetos tem comprimento $1/2$.





\newpage

%     _                                    _      _       
%    / \   _ __ ___  __ _ ___    ___    __| | ___| |_ ___ 
%   / _ \ | '__/ _ \/ _` / __|  / _ \  / _` |/ _ \ __/ __|
%  / ___ \| | |  __/ (_| \__ \ |  __/ | (_| |  __/ |_\__ \
% /_/   \_\_|  \___|\__,_|___/  \___|  \__,_|\___|\__|___/
%                                                         
% «areas-e-determinantes» (to ".areas-e-determinantes")
% (gam181p 34 "areas-e-determinantes")
\mypsection {areas-e-determinantes} {Áreas e determinantes em $R^2$}
% (gar181p 1 "areas-em-R3")
% (gar181    "areas-em-R3")

Notações: se $\uu$ e $\vv$ são vetores em $\R^2$ então
$\Area(\uu,\vv)$ é a área do paralelogramo gerado por $\uu$ e $\vv$;
se $A$, $B$, $C$ são pontos de $\R^2$ então $\Area(ΔABC)$ é a área do
{\sl triângulo} $ΔABC$. Triângulos são ``metades de paralelogramos''.

O slogan ``a área de um triângulo é base vezes altura sobre 2'' pode
ser interpretado de várias formas. Podemos pensar que a base do
triângulo $ΔABC$ é a distância $d(A,B)$ (um número), ou que a base é o
segmento $\overline{AB}$; e podemos usar o slogan pra mudar um ponto
do triângulo original mantendo a mesma base e a mesma altura, obtendo
um triângulo $ΔABC'$ com $\Area(ΔABC) = \Area(ΔABC')$. Quando a base
$\overline{AB}$ é um segmento horizontal ``manter a mesma altura''
quer dizer deslizar o ponto $C$ ao longo de uma reta horizontal;
quando $\overline{AB}$ é um segmento qualquer ``manter a mesma
altura'' quer dizer deslizar $C$ ao longo de uma reta $r$ parelela a
$AB$ e que passa por $C$ --- ou seja,
%
$$C' ∈ r = \setofst{C+t\Vec{AB}}{t∈\R}.$$

No diagrama à esquerda abaixo temos $\Area(ΔABC) = \Area(ΔABC') =
\Area(ΔABC'')$; no diagrama à direita abaixo também.
%
% {\sl (Ainda não fiz esses diagramas --- vou fazê-los no quadro)}
%
$$\unitlength=12pt
  \def\closeddot{\circle*{0.3}}
  \def\putc(#1,#2){\put(#1,#2){\closeddot}}
  %
  \vcenter{\hbox{%
   \beginpicture(-1,-1)(5,4)%
   \pictgrid%
   %\pictaxes
   {\linethickness{1.0pt}
                             \putc(0,0)   \put(0,-1){\cell{A}}
                             \putc(2,0)   \put(2,-1){\cell{B}}
    \polygon(0,0)(2,0)(0,3)  \putc(0,3)   \put(0,3.2){\cell{C}}
    \polygon(0,0)(2,0)(1,3)  \putc(1,3)   \put(1,3.2){\cell{C'}}
    \polygon(0,0)(2,0)(4,3)  \putc(4,3)   \put(4,3.2){\cell{C''}}
   }
   \end{picture}%
  }}%
  %
  \qquad
  %
  \vcenter{\hbox{%
   \beginpicture(-1,-1)(4,4)%
   \pictgrid%
   %\pictaxes
   {\linethickness{1.0pt}
                             \putc(0,1)  \put(-0.5,0.3){\cell{A}}
                             \putc(1,0)  \put(0.5,-0.7){\cell{B}}
    \polygon(0,1)(1,0)(0,3)  \putc(0,3)  \put(0,3.2){\cell{C}}
    \polygon(0,1)(1,0)(1,2)  \putc(1,2)  \put(1.3,2.2){\cell{C'}}
    \polygon(0,1)(1,0)(2,1)  \putc(2,1)  \put(2.8,1){\cell{C''}}
   }
   \end{picture}%
  }}%
$$


\msk

{\bf Idéias.} Temos $\Area(ΔABC) = \frac12 \Area(\Vec{AB},\Vec{AC})$.
Quando $\uu⊥\vv$ a área pode ser calculada de forma bem fácil:
$\Area(\uu,\vv) = ||\uu||·||\vv||$. Quando $\ang(\Vec{AB},\Vec{AC}) ≠
90°$ podemos calcular $\Area(\Vec{AB},\Vec{AC})$ encontrando um ponto
$C'$ ``por deslizamento'', isto é, tal que $C'=C+t\Vec{AB}$, que
obedeça $\Vec{AB}⊥\Vec{AC'}$. A tradução disto pra vetores é
$\Area(\uu,\vv) = \Area(\uu,\vv+t\uu)$. Podemos deslocar os pontos $A$
e $B$ ao invés de $C$: se $A'=A+α\Vec{BC}$ e $B'=B+β\Vec{AC}$ então
$\Area(ΔABC) = \Area(ΔA'BC) = \Area(ΔAB'C)$; $\Area(\uu,\vv) =
\Area(\uu+t\vv,\vv)$.

{\bf Determinantes.} A notação usual para determinantes de matrizes
$2×2$ é $\vsm{a & b \\ c & d} = \left| \psm{a & b \\ c & d} \right| =
ad-bc$, mas também vamos usar $\det(\VEC{a,b},\VEC{c,d}) = \vsm{a & b
  \\ c & d}$. Determinantes calculam a área ``com sinal''; truque:
$\Area(\uu,\vv) = |\det(\uu,\vv)|$.

\newpage

%     _                                    _      _         ____  
%    / \   _ __ ___  __ _ ___    ___    __| | ___| |_ ___  |___ \ 
%   / _ \ | '__/ _ \/ _` / __|  / _ \  / _` |/ _ \ __/ __|   __) |
%  / ___ \| | |  __/ (_| \__ \ |  __/ | (_| |  __/ |_\__ \  / __/ 
% /_/   \_\_|  \___|\__,_|___/  \___|  \__,_|\___|\__|___/ |_____|
%                                                                 
% «areas-e-determinantes-2» (to ".areas-e-determinantes-2")
% (gam181p 35 "areas-e-determinantes-2")
\mypsection {areas-e-determinantes-2} {Áreas e determinantes em $R^2$ (2)}
% (gar181p 1 "areas-em-R3")
% (gar181    "areas-em-R3")

{\bf Exercícios}

1) Para cada um dos casos abaixo represente graficamente o
paralelogramo gerado por $\uu$ e $\vv$ e calcule $\Area(\uu,\vv)$. Use
deslizamentos se precisar mas não use determinantes.

2) Para cada um dos casos abaixo represente graficamente o
paralelogramo gerado por $\uu$ e $\vv$ e calcule $\Area(\uu,\vv)$. Use
determinantes mas não use deslizamentos.

\begin{tabular}[t]{l}
a) $\uu=\VEC{2,0}$,  $\vv=\VEC{0,3}$    \\
b) $\uu=\VEC{2,0}$,  $\vv=\VEC{0,-3}$   \\
c) $\uu=\VEC{-2,0}$, $\vv=\VEC{0,3}$    \\
d) $\uu=\VEC{2,0}$,  $\vv=\VEC{1,4}$    \\
e) $\uu=\VEC{2,0}$,  $\vv=\VEC{1,3}$    \\
f) $\uu=\VEC{2,0}$,  $\vv=\VEC{1,0}$    \\
\end{tabular}
\quad
\begin{tabular}[t]{l}
g) $\uu=\VEC{2,1}$ $\vv=\VEC{-1,2}$   \\
h) $\uu=\VEC{2,1}$ $\vv=\VEC{-2,4}$   \\
i) $\uu=\VEC{2,1}$ $\vv=\VEC{1,3}$   \\
j) $\uu=\VEC{2,1}$ $\vv=\VEC{0,1}$   \\
k) $\uu=\VEC{1,0}$ $\vv=\VEC{-1,2}$   \\
\end{tabular}



\newpage

%  ____         _                       _               
% |  _ \    ___| | ___  ___  ___ _ __  | |_ ___    _ __ 
% | |_) |  / __| |/ _ \/ __|/ _ \ '__| | __/ _ \  | '__|
% |  __/  | (__| | (_) \__ \  __/ |    | || (_) | | |   
% |_|      \___|_|\___/|___/\___|_|     \__\___/  |_|   
%                                                       
% «pontos-mais-proximos» (to ".pontos-mais-proximos")
% (gam181p 36 "pontos-mais-proximos")
\mypsection {pontos-mais-proximos} {Pontos mais próximos e pontos simétricos}

{\bf Exercícios}

1) Sejam $A=(2,3)$, $\uu=\VEC{1,0}$, $r=\setofst{A+t\uu}{t∈\R}$,
$B=(4,0)$. Sejam $B'$ o ponto de $r$ mais próximo de $B$ e
$B''=B'+\Vec{BB'}$. Represente graficamente $A$, $\uu$, $r$, $B$,
$B'$, $B''$.

2) Sejam $A$, $\uu$ e $r$ como no exercício anterior. Seja $C=(5,1)$.
Sejam $C'$ o ponto de $r$ mais próximo de $C$ e $C''=C'+\Vec{CC'}$.
Represente graficamente $C$, $C'$, $C''$ no gráfico do exercício
anterior.

3) Vamos usar a mesma convenção dos exercícios anteriores para as
letras $D$, $E$, $\ldots$ --- $D'$ é o ponto de $r$ mais próximo a
$D$, $D''=D'+\Vec{DD'}$, etc. Sejam $D=(4,2)$ e $E=(3,3)$. Represente
graficamente $D$, $D'$, $D''$, $E$, $E'$, $E''$ no mesmo gráfico dos
exercícios 1 e 2.

4) Sejam $A=(4,1)$, $\uu=\VEC{0,1}$, $r=\setofst{A+t\uu}{t∈\R}$,
$B=(1,0)$, $C=(2,2)$, $D=(4,3)$, $E=(5,4)$. Represente graficamente
num gráfico só (separado do dos exercícios anteriores!) $A$, $\uu$,
$r$, $B$, $B'$, $B''$, $C$, $D'$, $C''$, $D$, $D'$, $D''$, $E$, $E'$,
$E''$.

5) Idem, mas agora $A=(0,4)$, $\uu=\VEC{1,-1}$, $B=(1,4)$, $C=(2,4)$,
$D=(2,1)$, $E=(1,1)$.

6) Idem, mas agora $A=(0,4)$, $\uu=\VEC{2,-1}$, $B=(1,1)$, $C=(3,0)$,
$D=(4,2)$, $E=(3,5)$, $F=(3,4)$, $G=(3,3)$, $H=(3,2)$. Obs: aqui nem
todos os pontos são fáceis de {\sl calcular}, mas você sabe desenhar
aproximações para ele no olhômetro.

\msk

Repare que quando $r=\setofst{A+t\uu}{t∈\R}$ a gente sempre pode
calcular o ``ponto de $r$ mais próximo de $B$'', $B'$, fazendo $B' =
A+\Pr_{\uu} \Vec{AB}$ --- e isto nos dá uma primeira fórmula para
calcular $d(B,r)$:
%
$$\begin{array}{rcl}
  d(B,r) &=& d(B,B') \\
         &=& d(B,A+\Pr_{\uu} \Vec{AB}) \\
         &=& ||B - (A+\Pr_{\uu} \Vec{AB})|| \\
         &=& ||\Vec{AB} - \Pr_{\uu} \Vec{AB}|| \\
         &=& ||\Pr_{\uu} \Vec{AB} - \Vec{AB}|| \\
  \end{array}
$$

7) Calcule $d(B,r)$, $d(C,r)$, $d(D,r)$, $d(E,r)$ no gráfico dos
exercícios 1, 2 e 3 acima, de dois modos: primeiro aproveite que você
conhece $B'$, $C'$, etc e use $d(B,r) = d(B,B')$, $d(C,r) = d(C,C')$,
etc; depois use a fórmula $d(B,r) = ||\Vec{AB} - \Pr_{\uu}
\Vec{AB}||$.

8) Faça o mesmo para $d(B,r)$, $\ldots$, $d(E,r)$ no gráfico do
exercício 4.

9) Faça o mesmo para $d(B,r)$, $\ldots$, $d(E,r)$ no gráfico do
exercício 5.

10) Faça o mesmo para o exercício 6.

\msk

{\bf Exercícios sobre coeficientes}

11) Encontre $a,b,c,d∈\R$ tais que $\Pr_{\VEC{2,-1}} \VEC{x,y} =
\VEC{ax+by,cx+dy}$.

12) Encontre $a,b,c,d∈\R$ tais que $\Pr_{\VEC{3,4}} \VEC{x,y} =
\VEC{ax+by,cx+dy}$.

% 5 . . . E . . .
% 4 r . . . . . .
% 3 . . r . . . .
% 2 . . . . r . .
% 1 . B . . . . .
% 0 . . . C . . .
%   0 1 2 3 4 5 6 7 8




\newpage

%   ____ _                _           
%  / ___(_)_ __ ___ _   _| | ___  ___ 
% | |   | | '__/ __| | | | |/ _ \/ __|
% | |___| | | | (__| |_| | | (_) \__ \
%  \____|_|_|  \___|\__,_|_|\___/|___/
%                                     
% «circulos» (to ".circulos")
% (gam181p 37 "circulos")
% (gaq172 28 "20171106" "áreas e círculos")
\mypsection {circulos} {Círculos (via pontos óbvios)}

Seja $S = \setofxyst{(x-4)^2 + (y-3)^2 = 9}$.

O conjunto $S$ é um círculo (juro! Depois vamos ver porquê), e se
con\-se\-guir\-mos um número suficiente de pontos de $S$ vamos
conseguir desenhar o círculo, determinar o seu centro e o seu raio,
etc.

Uma gambiarra: é fácil encontrar os ``quatro pontos óbvios'' que são
soluções de $(x-4)^2 + (y-3)^2 = 9$ --- a gente primeiro faz $(x-4)^2
= 0$ e encontra os dois valores de $y$ que são soluções de $0 +
(y-3)^2 = 9$, depois a gente faz $(y-3)^2 = 0$ e encontra os dois
valores de $x$ que são soluções de $(x-4)^2 + 0 = 9$.
%
$$\def\obvio#1#2#3#4#5#6#7{
    \begin{array}{c}
    \und{(\und{\und{x}{#1}- \;\; 4}{#2})^2}{#3} +
    \und{(\und{\und{y}{#4}- \;\; 3}{#5})^2}{#6} = 9 \\[45pt]
    \qquad\qquad⇒\quad (x,y)=#7 \\
    \end{array}
  }
  \begin{array}{ccc}
  \obvio 400639 {(4,6)} && \obvio 4000{-3}9 {(4,0)} \\ \\ \\
  \obvio 739300 {(7,3)} && \obvio 1{-3}9300 {(1,3)} \\
  \end{array}
$$

Os pontos óbvios vão ser o ponto mais alto do círculo, o mais baixo, o
mais à esquerda e o mais à direita.

\msk

{\bf Exercícios}

1) Cada um dos conjuntos abaixo é um círculo.

$C = \setofxyst{(x-3)^2 + (y-5)^2 = 4}$

$C' = \setofxyst{(x-3)^2 + (y-5)^2 = 1}$

$C'' = \setofxyst{(x-2)^2 + (y-5)^2 = 1}$

$C''' = \setofxyst{(x-2)^2 + (y-3)^2 = 1}$

$C'''' = \setofxyst{(x-2)^2 + (y-3)^2 = 25}$

Para cada um deles

a) encontre os 4 pontos óbvios do círculo,

b) represente graficamente o círculo,

c) dê o centro e o raio do círculo.

\msk

2) Tente fazer o mesmo para estes círculos degenerados.

$C = \setofxyst{(x-2)^2 + (y-5)^2 = 0}$

$C' = \setofxyst{(x-2)^2 + (y-5)^2 = -1}$





\newpage

%   ____ _                _             ____  
%  / ___(_)_ __ ___ _   _| | ___  ___  |___ \ 
% | |   | | '__/ __| | | | |/ _ \/ __|   __) |
% | |___| | | | (__| |_| | | (_) \__ \  / __/ 
%  \____|_|_|  \___|\__,_|_|\___/|___/ |_____|
%                                             
% «circulos-2» (to ".circulos-2")
% (gam181p 38 "circulos-2")
% (gaq172 29 "20171108" "círculos")
\mypsection {circulos-2} {Círculos e cônicas}

Quase todos os problemas das listas da Ana Isabel sobre círculos usam

equações desta forma:

(A) \quad $ax^2 + bx + cy^2 + dy + e = 0$

onde $a, b, c, d, e ∈ \R$.

Os que nós vimos têm equações desta forma:

(B) \quad $(x-a)^2 + (y-b)^2 = c$

onde $a, b, c ∈ \R$.

Uma {\sl equação de cônica} é uma equação desta forma:

(C) \quad $ax^2 + bx + c + dxy + ey + fy^2 = 0$

onde $a, b, c, d, e, f ∈ \R$.



\msk

{\bf Exercícios}

Converta as seguintes equações da forma (B) para a forma (A).

1) $(x-3)^2 + (y=4)^2 = 25$

2) $(x+2)^2 + (y-3)^2 = 16$

Converta as seguintes equações da forma (A) para a forma (B).

3) $x^2 + 2x + y^2 - 2y - 7 = 0$

4) $x^2 + y^2 - 6y - 9 = 0$

5) $x^2 + y^2 - 6y + 9 = 0$

6) $x^2 + y^2 + 6y - 8 = 0$

7) $x^2 - y^2 = 0$

\msk

Dica: ``completar quadrados''...

$(x+a)^2 = x^2 + 2ax + a^2$

$(x+a)^2 + b = x^2 + 2ax + a^2 + b$

$(x+a)^2 + b - a^2 = x^2 + 2ax + b$

\bsk

{\bf Interseção de círculo e reta (algebricamente)}

Método: comece com equações

(D) \quad $ax^2 + bx + dx^2 + ey + f = 0$,

(E) \quad $y = gx + h$

e substitua cada $y$ em (D) por $gx+h$. Converta a equação que você

obteve para a forma

(F) \quad $ix^2 + jx + k = 0$,

resolva-a por Bháskara e chame as soluções de $x_1$ e $x_2$.

Use (E) para definir $y_1 = gx_1 + h$ e $y_2 = gx_2 + h$. 

Os pontos $(x_1,y_1)$ e $(x_2,y_2)$ são as interseçõe do círculo com a
reta.

\msk

Exercícios. Sejam $C$ o círculo de centro $(5,5)$ e raio 5,

$r = \setofxyst{x+y=3}$, $r' = \setofxyst{y=5-3x}$.

8) Calcule $C∩r$. 

9) Calcule $C∩r'$. 





\newpage

%  ____                                                     
% |  _ \  ___  ___ ___  _ __ ___  _ __     __ _ _ __   __ _ 
% | | | |/ _ \/ __/ _ \| '_ ` _ \| '_ \   / _` | '_ \ / _` |
% | |_| |  __/ (_| (_) | | | | | | |_) | | (_| | | | | (_| |
% |____/ \___|\___\___/|_| |_| |_| .__/   \__,_|_| |_|\__, |
%                                |_|                  |___/ 
%
% «decomp-ang» (to ".decomp-ang")
% (gam181p 39 "decomp-ang")
\mypsection {decomp-ang} {Uma decomposição (e várias utilidades para ângulos)}

Sejam $A$ e $B$ dois pontos diferentes de $\R^2$.

Seja $C$ um ponto de $\R^2$.

Seja $r$ a reta que passa por $A$ e $B$.

Seja $s$ uma reta ortogonal a $r$ que passa por $A$.

Seja $D$ o ponto de $r$ mais próximo de $C$.

Seja $E$ o ponto de $s$ mais próximo de $C$.

$$\unitlength=12pt
  \def\closeddot{\circle*{0.3}}
  \def\putc(#1,#2){\put(#1,#2){\closeddot}}
  %
  \vcenter{\hbox{%
   \beginpicture(-1,-1)(6,5)%
   \pictgrid%
   %\pictaxes
   {\linethickness{1.0pt}
                             \putc(0,0)   \put(0,-1){\cell{A}}
    \Vector(0,0)(5,0)        \putc(5,0)   \put(5,-1){\cell{B}}
    \Vector(0,0)(3,4)        \putc(3,4)   \put(3.3,4.4){\cell{C}}
    \Vector(0,0)(3,0)        \putc(3,0)   \put(3,-1){\cell{D}}
    \Vector(0,0)(0,4)        \putc(0,4)   \put(0,4.3){\cell{E}}
   }
   \end{picture}%
  }}%
$$


Essa construção decompõe o vetor $\Vec{AC}$ em uma componente,
$\Vec{AD}$,

paralela a $\Vec{AB}$, e outra componente, $\Vec{AE}$,
ortogonal a $\Vec{AB}$.

Formalmente: $\Vec{AC} = \Vec{AD} + \Vec{AE}$, com $\Vec{AD}
\myparallel \Vec{AB}$ e $\Vec{AD}⊥\Vec{AE}$.

($\Vec{AD} \myparallel \Vec{AB}$ quer dizer $\Vec{AD} = λ\Vec{AB}$
para algum $λ∈\R$.)

\msk

A função ``$\ang$'' sempre responde um ângulo entre $0°$ e $180°$ (lembre da `$√·$'!);

daí sempre temos $0 ≤ \sen(\ang(\uu,\vv)) ≤ 1$, mas $\cosθ<0$ para $θ$
obtuso.

\msk

Seja $θ = \ang(\Vec{AB}, \Vec{AC}) = \ang(\Vec{AD}, \Vec{AC})$. Lembre
que ``cosseno é cateto

adjacente sobre hipotenusa'' e ``seno é cateto oposto sobre
hipotenusa''.

Repare que $ADCE$ é um retângulo e que:

1) $|\cosθ| = d(A,D) / d(A,C)$,

2) $\senθ = d(D,C) / d(A,C) = d(A,E) / d(A,C)$,

3) $d(A,D) = |\cos(θ)| \, d(A,C)$,

4) $d(A,E) = \sen(θ) \, d(A,C)$,

5) $\Vec{AD} = \Pr_{\Vec{AB}} \Vec{AC}$,

6) $\Vec{AB} · \Vec{AC} = \Vec{AB} · (\Vec{AD} + \Vec{AE}) = \Vec{AB} ·
\Vec{AD} + \Vec{AB} · \Vec{AE} = \Vec{AB} · \Vec{AD}$

7) $\Pr_{\Vec{AB}} \Vec{AC} = \Pr_{\Vec{AB}} (\Vec{AD} + \Vec{AE}) =
\Pr_{\Vec{AB}} \Vec{AD} + \Pr_{\Vec{AB}} \Vec{AE} = \Pr_{\Vec{AB}}
\Vec{AD} = \Vec{AD}$

7') $||\Pr_{\Vec{AB}} \Vec{AC}|| = ||\Vec{AD}|| = |\cosθ| \, ||\Vec{AC}||$

8) $\Area(\Vec{AB}, \Vec{AC}) = \Area(\Vec{AB}, \Vec{AE}) =
||\Vec{AB}|| · ||\Vec{AE}|| = ||\Vec{AB}|| · (\senθ \, ||\Vec{AC}||)$

8') $\Area(\Vec{AB}, \Vec{AC}) = \senθ \, ||\Vec{AB}|| \,
||\Vec{AC}||$

9) $\uu·\vv = \cos(\ang(\uu,\vv)) \, ||\uu|| \, ||\vv||$ (``fórmula do
cosseno'')

Note que as afirmações 1--9 acima não usam $θ$, só $\sen θ$ e $\cos
θ$.

\msk

{\bf Exercícios}

1) Verifique as afirmações 1--9 no caso $\Vec{AB}=\VEC{5,0}$,
$\Vec{AC}=\VEC{3,4}$.

2) Verifique as afirmações 1--9 no caso $\Vec{AB}=\VEC{4,4}$,
$\Vec{AC}=\VEC{0,2}$.

3) Verifique as afirmações 1--9 no caso $\Vec{AB}=\VEC{4,4}$,
$\Vec{AC}=\VEC{-2,0}$.




\newpage

%      _  ______       __  
%   __| |/ /  _ \   _ _\ \ 
%  / _` | || |_) | | '__| |
% | (_| | ||  __/  | |  | |
%  \__,_| ||_| ( ) |_|  | |
%        \_\   |/      /_/ 
%
% «distancia-ponto-reta» (to ".distancia-ponto-reta")
% (gam181p 40 "distancia-ponto-reta")
\mypsection {distancia-ponto-reta} {Distância entre ponto e reta: segundo modo}

% (gaq172 20 "20171004" "d(P,r)")
% (gaq172 24 "20171016" "dois modos de calcular d(P,r)")
% (gaq172 25 "20171018" "dois modos de calcular d(P,r): segunda fórmula")

Sejam $A$ e $B$ dois pontos diferentes de $\R^2$ e seja $r$ a reta que
contém $A$ e $B$. Seja $P$ um ponto qualquer de $\R^2$. Na
p.\pageref{pontos-mais-proximos} nós vimos um primeiro modo de
calcular $d(C,r)$ --- a gente encontrava o ponto $C'∈r$ mais próximo
de $C$ e depois calculava $d(C,C')$. Agora vamos ver um outro modo no
qual as contas ficam bem mais rápidas, mas que só faz sentido se a
gente entende ângulos.

Seja $s$ uma reta vertical que passa por $C$ e seja $D∈r∩s$.

Seja $r = \setofxyst{y=mx+b}$; $m$ é o coeficiente angular de $r$.
%
%L uu  = V {1,2}
%L vv  = V {2,-1}
%L pA  = V {0,-1}
%L pB  = pA + 4*uu
%L pD  = pA + 1*uu
%L pCC = pD + 2*uu
%L pC  = pCC - vv
%L
%L pE  = V {0,0}
%L pF  = V {1,0}
%L pG  = V {1,2}
\pu
%
$$\unitlength=10pt
  \def\closeddot{\circle*{0.3}}
  \def\putc(#1,#2){\put(#1,#2){\closeddot}}
  %
  \vcenter{\hbox{%
   \beginpicture(-2,-2)(6,8)%
   \pictgrid%
   \pictaxes%
   {\linethickness{1.0pt}
    \CalcPoints{
      !Line<pA - 0.5*uu, pA + 4.5*uu>
      !putc<pA>         !put<pA + V{.5,-.5}>{!cell{A}}
      !putc<pB>         !put<pB + V{.5,-.5}>{!cell{B}}
      !putc<pC>         !put<pC + V{ 0, .5}>{!cell{C}}
      !putc<pCC>        !put<pCC+ V{.7,-.5}>{!cell{C'}}
      !putc<pD>         !put<pD + V{.5,-.5}>{!cell{D}}
      !Line<pCC, pC, pD>
    }
    %\Calcpoints{!put<a/2,-a/2>{!cell{a^2}}}
    %\Calcpoints{!put<-b/2,b/2>{!cell{b^2}}}
   }
   \end{picture}%
  }}%
  %
  \qquad
  \unitlength=20pt
  \def\closeddot{\circle*{0.15}}
  \def\putc(#1,#2){\put(#1,#2){\closeddot}}
  %
  \vcenter{\hbox{%
   \beginpicture(-1,-1)(2,3)%
   \pictgrid%
   %\pictaxes%
   {\linethickness{1.1pt}
    \CalcPoints{
      !polygon<pE, pF, pG>
      !putc<pE>         !put<pE + V{-.3,-.6}>{!cell{E}}
      !putc<pF>         !put<pF + V{ .3,-.6}>{!cell{F}}
      !putc<pG>         !put<pG + V{ 0, .4}>{!cell{G}}
    }
    %\Calcpoints{!put<a/2,-a/2>{!cell{a^2}}}
    %\Calcpoints{!put<-b/2,b/2>{!cell{b^2}}}
   }
   \end{picture}%
  }}%
$$

O triângulo retângulo $ΔCC'D$ é semelhante a um outro triângulo bem
mais simples, $ΔEFG$, que tem um cateto horizontal e outro vertical e
cuja hipotenusa é paralela à reta $r$; temos $\Vec{EG} = \VEC{1,m}$
(hipotenusa), $\Vec{EF} = \VEC{1,0}$ (cateto horizontal) e $\Vec{FG} =
\VEC{0,m}$ (cateto vertical).

\msk

Queremos calcular $d(C,C')$ mas é trabalhoso fazer isto diretamente,
então vamos calcular $d(C,D)$, que é fácil, e a proporção
$d(C,C')/d(C,D)$ (que é o cosseno de um ângulo --- qual?)... temos:

• se $C=(C_x,C_y)$ então $D = (C_x, mC_x+b)$,

• $d(C,D) = |C_y - (mC_x+b)| = |mC_x+b - C_y|$,

• $d(C,C') = \frac{d(C,C')}{d(C,D)} d(C,D)$,

• $\frac{d(C,C')}{d(C,D)} = \frac{d(E,F)}{d(E,G)} = \frac{1}{\sqrt{1+m^2}}$,

• $d(C,C') = \frac{1}{\sqrt{1+m^2}} |mC_x+b - C_y|$,

• $d(C,r) = \frac{1}{\sqrt{1+m^2}} |mC_x+b - C_y|$.

\msk

Obs: o meu truque pra lembrar é essa fórmula é: seja $dv(C,r)$ a
``distância vertical'' de $C$ até $r$, isto é, a distância entre $C$ e
o ponto $r∩s$, onde $s$ é uma reta vertical que passa por $C$. A
distância $d(C,r)$ é igual à ``distância vertical'' $dv(C,r)$ vezes
alguma coisa que tem $\sqrt{1+m^2}$ no meio, e os casos mais fáceis de
testar são os com coeficientes angulares iguais a 0, 1 ou 2...
lembrando isso eu faço alguns testes e encontro a fórmula certa.

\msk

{\bf Exercício.} Em cada um dos casos abaixo represente graficamente
$r$ e $F(x,y) = dv((x,y),r)$; use a notação da p.\pageref{Fxy} para
representar $F(x,y)$.

1) $r = \setofxyst{y=x+2}$

2) $r = \setofxyst{y=4-2x}$

3) $r = \setofxyst{y=3}$


\newpage

%   ____            _                        ____  _____ 
%  / ___|___  _ __ (_) ___ __ _ ___    ___  |  _ \|___ / 
% | |   / _ \| '_ \| |/ __/ _` / __|  / _ \ | |_) | |_ \ 
% | |__| (_) | | | | | (_| (_| \__ \ |  __/ |  _ < ___) |
%  \____\___/|_| |_|_|\___\__,_|___/  \___| |_| \_\____/ 
%                                                        
% «conicas-e-R3» (to ".conicas-e-R3")

% (find-twusfile "LATEX/"  "2018-1-GA-conicas.pdf")
% (find-twusfile "LATEX/"  "2018-1-GA-R3.pdf")
% http://angg.twu.net/LATEX/2018-1-GA-conicas.pdf
% http://angg.twu.net/LATEX/2018-1-GA-R3.pdf





\unitlength=10pt


%     _                       
%    / \   _ __ ___  __ _ ___ 
%   / _ \ | '__/ _ \/ _` / __|
%  / ___ \| | |  __/ (_| \__ \
% /_/   \_\_|  \___|\__,_|___/
%                             
% «areas-em-R3» (to ".areas-em-R3")
% (gar181p 1 "areas-em-R3")
% (gar181    "areas-em-R3")
%\mypsection {areas-em-R3} {Áreas de retângulos e paralelogramos em $\R^3$}
\mypsection {areas-em-R3} {Áreas de retângulos e paralelogramos em $R^3$}

Notação: se $\uu$ e $\vv$ são vetores em $\R^3$ então $\Area(\uu,\vv)$
é a área do paralelogramo gerado por $\uu$ e $\vv$. Quando $\uu⊥\vv$ a
área pode ser calculada de forma bem fácil: $\Area(\uu,\vv) =
||\uu||·||\vv||$.

\ssk

{\bf Exercícios}

1) Visualize os paralelogramos abaixo e calcule a área de cada
um deles. Em alguns casos você vai ter que usar truques pouco óbvios;
em outros casos talvez você vá ter que responder ``não sei''.

\begin{tabular}[t]{l}
a) $\Area(\VEC{2,0,0},\VEC{0,3,0})$   \\
b) $\Area(\VEC{0,3,0},\VEC{0,0,-4})$  \\
c) $\Area(\VEC{5,0,0},\VEC{0,5,0})$   \\
d) $\Area(\VEC{5,0,0},\VEC{4,3,0})$   \\
e) $\Area(\VEC{5,0,0},\VEC{3,4,0})$   \\
f) $\Area(\VEC{4,3,0},\VEC{-3,4,0})$  \\
\end{tabular}
\quad
\begin{tabular}[t]{l}
g) $\Area(\VEC{4,3,0},\VEC{4,3,0})$   \\
h) $\Area(\VEC{4,3,0},\VEC{3,4,0})$   \\
i) $\Area(\VEC{5,0,0},\VEC{0,4,3})$   \\
j) $\Area(\VEC{5,0,0},\VEC{0,3,4})$   \\
k) $\Area(\VEC{5,0,0},\VEC{0,0,5})$   \\
\end{tabular}

\msk

Podemos calcular áreas de paralelogramos em $\R^3$ usando um truque de
``deslizamento'' parecido com o que usamos para áreas e determinantes
em $\R^2$. Se $\uu⊥\vv$ e $k∈\R$, então $\Area(\uu,\vv) =
\Area(\uu,\vv+k\uu)$ --- e repare que $\Area(\uu,\vv)$ é a área de um
retângulo e $\Area(\uu,\vv+k\uu)$ é a área de um paralelogramo.

\ssk

2) Use o truque acima em cada um dos itens abaixo. Visualize o
paralelogramo $\Area(\uu,\vv+k\uu)$ e o retângulo $\Area(\uu,\vv)$
associado a ele, e calcule as áreas.

\begin{tabular}[t]{l}
a) $\Area(\VEC{4,0,0},\VEC{0,3,0}+\VEC{4,0,0})$          \\
b) $\Area(\VEC{4,0,0},\VEC{0,3,0}+\frac34\VEC{4,0,0})$   \\
c) $\Area(\VEC{4,0,0},\VEC{0,3,0}+\frac24\VEC{4,0,0})$   \\
d) $\Area(\VEC{4,0,0},\VEC{0,3,0}+\frac14\VEC{4,0,0})$   \\
e) $\Area(\VEC{4,0,0},\VEC{0,3,0})$   \\
f) $\Area(\VEC{4,3,0},\VEC{0,0,1})$  \\
g) $\Area(\VEC{4,3,0}+\VEC{0,0,1},\VEC{0,0,1})$  \\
h) $\Area(\VEC{4,3,0}+2\VEC{0,0,1},\VEC{0,0,1})$  \\
i) $\Area(\VEC{4,3,0}+3\VEC{0,0,1},\VEC{0,0,1})$  \\
\end{tabular}

\ssk

3) Faça o mesmo nos casos abaixo, mas agora você vai ter que escolher
os vetores $\uu$ e $\vv$ adequados você mesmo.

\begin{tabular}[t]{l}
a) $\Area(\VEC{4,0,0},\VEC{0,3,0}+\VEC{4,0,0})$         (mudar) \\
b) $\Area(\VEC{4,0,0},\VEC{0,3,0}+\frac34\VEC{4,0,0})$  (mudar) \\
c) $\Area(\VEC{4,0,0},\VEC{0,3,0}+\frac24\VEC{4,0,0})$  (mudar) \\
\end{tabular}

\msk

4) Demonstre que se $\uu⊥\vv$ e $a,k∈\R$ então:
%
$$\Area(\uu,a(\vv+k\uu)) = |a|\,\Area(\uu,\vv+k\uu).$$




\newpage





%  ____ /\ _____ 
% |  _ \/\|___ / 
% | |_) |   |_ \ 
% |  _ <   ___) |
% |_| \_\ |____/ 
%                
% «R3-retas-e-planos» (to ".R3-retas-e-planos")
% (gar181p 2 "R3-retas-e-planos")
% (gar181    "R3-retas-e-planos")

\mypsection {R3-retas-e-planos} {Retas e planos em $R^3$}

\ssk

Obs: adaptado da aula de 4/jul/2016:

\url{http://angg.twu.net/2016.1-GA/2016.1-GA.pdf}

\msk

% {\bf Retas em $\R^3$}

Sejam:

$r_1 = \setofexprt{(2,2,0)+t\V(0,-1,0)}$

$r_2 = \setofexprt{(2,2,1)+t\V(0,-1,0)}$

$r_3 = \setofexprt{(2,2,0)+t\V(0,1,1)}$

$r_4 = \setofexprt{(0,2,1)+t\V(1,0,0)}$

$r_4 = \setofexprt{(1,2,1)+t\V(2,0,0)}$

Quais destas retas se interceptam?

Em que pontos? Em que `$t$'s?

Quais destas retas são paralelas?

Quais destas retas são coincidentes?

A terminologia para retas que não se interceptam e não são

paralelas é estranha -- ``retas {\sl reversas}''.

\msk

As retas acima são {\sl parametrizadas}.

O que é uma {\sl equação de reta} em $\R^3$?

$\setofxyst{4x+5y=6}$ é uma reta em $\R^2$;

$\setofxyzst{4x+5y+6z=7}$ é um {\sl plano} em $\R^3$...

\msk

Exercício: encontre

três pontos não colineares de $\setofxyzst{z=0}$,

três pontos não colineares de $\setofxyzst{z=2}$,

três pontos não colineares de $\setofxyzst{x=1}$,

três pontos não colineares de $\setofxyzst{y=3}$,

três pontos não colineares de $\setofxyzst{\frac x2 + \frac y3 + \frac z4 = 1}$,

e visualize cada um destes planos.

\msk

Alguns dos nossos planos preferidos:

$π_{xy} = \setofxyzst{z=0}$ ($x$ e $y$ variam, $z=0$)

$π_{xz} = \setofxyzst{y=0}$ ($x$ e $z$ variam, $y=0$)

$π_{yz} = \setofxyzst{x=0}$ ($y$ e $z$ variam, $x=0$)

\ssk

Notação (temporária):

$[\text{equação}] = \setofxyzst{\text{equação}}$

Obs: $π_{xy} = [z=0]$, $π_{xz} = [y=0]$, $π_{yz} = [x=0]$.

\msk

Exercício: visualize:

$π_1 = [x=1]$,     \qquad $π_8 = [y=x]$,     
                                      
$π_2 = [y=1]$,     \qquad $π_9 = [y=2x]$,    
                                      
$π_3 = [z=1]$,     \qquad $π_{10} = [z=x]$,  
                                      
$π_4 = [z=4]$,     \qquad $π_{11} = [z=x+1]$,

$π_5 = [z=2]$,

Quais deles planos são paralelos?

Quais deles planos se cortam? Onde?



\newpage

%  ____ /\ _____     ________  
% |  _ \/\|___ /    / /___ \ \ 
% | |_) |   |_ \   | |  __) | |
% |  _ <   ___) |  | | / __/| |
% |_| \_\ |____/   | ||_____| |
%                   \_\    /_/ 
%
% «R3-retas-e-planos-2» (to ".R3-retas-e-planos-2")
% (gar181p 3 "R3-retas-e-planos-2")
% (gar181    "R3-retas-e-planos-2")
% (gar181p 30 "R3-retas-e-planos-2")
% (gam172p 30 "R3-retas-e-planos-2")

\mypsection {R3-retas-e-planos-2} {Retas e planos em $R^3$ (2)}

\ssk

Dá pra parametrizar planos em $\R^3$...

Sejam

$π_6 = \setofst{\und{(2,2,0) + a\V(1,0,0) + b\V(0,1,0)}
                    {(a,b)_{Σ_6}}
                }{a,b∈\R}$,

$π_7 = \setofst{\und{(3,2,1) + a\V(1,0,0) + b\V(0,1,0)}
                    {(a,b)_{Σ_7}}
                }{a,b∈\R}$.

Calcule e visualize:

$(0,0)_{Σ_6}$, $(1,0)_{Σ_6}$, $(0,1)_{Σ_6}$, $(1,1)_{Σ_6}$,

$(0,0)_{Σ_7}$, $(1,0)_{Σ_7}$, $(0,1)_{Σ_7}$, $(1,1)_{Σ_7}$,

e resolva:

$(a,b)_{Σ_6} = (0,3,0)$,

$(a,b)_{Σ_7} = (2,4,1)$,

$(a,b)_{Σ_7} = (2,4,0)$.

\msk

Nossos três modos preferidos de descrever planos em $\R^3$ (por equações) são:

$[z = ax+by+c]$ (``$z$ em função de $x$ e $y$''),

$[y = ax+bz+c]$ (``$y$ em função de $x$ e $z$''),

$[x = ay+bz+c]$ (``$x$ em função de $y$ e $z$'').



% (find-LATEX "2016-2-GA-algebra.tex" "Fxy")
\msk

Na p.10 nós vimos este tipo de diagrama aqui, que nos ajuda a visualizar

as curvas de nível de funções de $x$ e $y$:

$\sm{F(x,y)\\=\,x+2y} ⇒
 \pictureFxy(-1,-2)(5,2){x+2*y}
$

Use diagramas deste tipo para visualizar

$[z=x+y]$, 

$[z=x+y+2]$, 

$[z=x-y+4]$.

\msk

Sejam:

$π_{12} = [z = x+y]$,

$π_{13} = [z = x-y+4]$ 

Exercício: encontre pontos de $r=π_{12}∩π_{13}$ tais que

a) $x=0$, b) $x=1$, c) $x=3$; depois

d) encontre uma parametrização para $r$,

e) encontre uma parametrização para $r$ na qual $t=x$.

\msk

Alguns dos nossos modos preferidos de descrever retas em $\R^3$:

$[y=ax+b, z=cx+d]$ (``$y$ e $z$ em função de $x$''),

$[x=ay+b, z=cy+d]$ (``$x$ e $z$ em função de $y$''),

$[x=az+b, y=cz+d]$ (``$x$ e $y$ em função de $z$'').

Encontre uma descrição da forma $[y=ax+b, z=cx+d]$ para a $r$ acima.

(Dica: use o ``chutar e testar''!)



\newpage


%  ____       _       
% |  _ \  ___| |_ ___ 
% | | | |/ _ \ __/ __|
% | |_| |  __/ |_\__ \
% |____/ \___|\__|___/
%                     
% «determinantes-em-R3» (to ".determinantes-em-R3")
% (gar181p 4 "determinantes-em-R3")
% (gar181    "determinantes-em-R3")
% (gar181p 31 "determinantes-em-R3")
% (gam172p 31 "determinantes-em-R3")

\mypsection {determinantes-em-R3} {Determinantes em $R^3$}

\ssk

Lembre que o determinante em $\R^2$ mede {\sl áreas} (de paralelogramos),

e às vezes ele responde números negativos:
%
$$\vsm{a&b\\c&d\\}
  = ac-bd \qquad
  \vsm{c&d\\a&b\\} = bd-ac = -\vsm{a&b\\c&d\\}
$$

Vamos usar a seguinte notação (temporária):

$[\uu,\vv]
  = [\V(u_1, u_2), \V(v_1, v_2)]
  = \vsm{u_1 & u_2 \\ v_1 & v_2 \\}
  \qquad \text{(em $\R^2$)}
$

$[\uu,\vv,\ww]
  = [\V(u_1, u_2, u_3), \V(v_1, v_2, v_3), \V(w_1, w_2, w_3)]
  = \vsm{u_1 & u_2 & u_3 \\ v_1 & v_2 & v_3 \\ w_1 & w_2 & w_3 \\}
  \qquad \text{(em $\R^3$)}
$

``$[\uu,\vv]$'' e ``$[\uu,\vv,\ww]$'' querem dizer

``empilhe os vetores numa matriz quadrada e tire o determinante dela''.

\msk

A definição de determinante em $\R^3$ -- como conta -- é:

$$\begin{array}{rcl}
  \vmat{u_1 & u_2 & u_3 \\ v_1 & v_2 & v_3 \\ w_1 & w_2 & w_3 \\}
  &=& \pmat{u_1v_2w_3 + u_2v_3w_4 + u_3v_4w_5 \\
         -u_3v_2w_1 - u_4v_3w_2 - u_5v_4w_3 \\
        } \\
  &=& \pmat{u_1v_2w_3 + u_2v_3w_1 + u_3v_1w_2 \\
         -u_3v_2w_1 - u_1v_3w_2 - u_2v_1w_3 \\
        }
  \end{array}
$$

\def\ii{\vec{\mathbf{i}}}
\def\jj{\vec{\mathbf{j}}}
\def\kk{\vec{\mathbf{k}}}

As seguintes definições são padrão:

$$\ii=\V(1,0,0) \qquad \jj=\V(0,1,0) \qquad \kk=\V(0,0,1)$$

Exercício: calcule

a) $[\ii,\jj,\kk]$

b) $[\ii,\kk,\jj]$

c) $[\jj,\ii,\kk]$

d) $[\jj,\kk,\ii]$

e) $[\kk,\ii,\jj]$

f) $[\kk,\jj,\ii]$

g) $[\ii,\jj,\ii]$

g) $[2\ii,3\jj,4\kk]$

h) $[a\ii,b\jj,c\kk]$

i) $[a\ii+b\jj+c\kk,d\jj+e\kk,f\kk]$

j) $[a\ii, b\ii+c\jj, d\ii+e\jj+f\kk]$

% (find-angg ".emacs" "gaq161")
% (gaq161 58 "20160704" "Visualizar R^3")



\newpage

%  ____       _                          ____ /\ _____    ________  
% |  _ \  ___| |_ ___    ___ _ __ ___   |  _ \/\|___ /   / /___ \ \ 
% | | | |/ _ \ __/ __|  / _ \ '_ ` _ \  | |_) |   |_ \  | |  __) | |
% | |_| |  __/ |_\__ \ |  __/ | | | | | |  _ <   ___) | | | / __/| |
% |____/ \___|\__|___/  \___|_| |_| |_| |_| \_\ |____/  | ||_____| |
%                                                        \_\    /_/ 
% «determinantes-em-R3-2» (to ".determinantes-em-R3-2")
% (gar181p 5 "determinantes-em-R3-2")
% (gar181    "determinantes-em-R3-2")
% (gar181p 32 "determinantes-em-R3-2")
% (gam172p 32 "determinantes-em-R3-2")

\mypsection {determinantes-em-R3-2} {Determinantes em $R^3$ (2)}

\ssk

Lembre que o determinante em $\R^2$ mede áreas, que são ``base vezes altura'',

e que a gente pode deslizar um lado ($\vv$) do paralelogramo gerado por $\uu$ e $\vv$

``numa direço paralela a $\uu$'', sem alterar nem a ``base'' nem a ``altura''...

Algebricamente: $[\uu,\vv] = [\uu,\vv+a\uu]$.

E deslizando o $\uu$, temos $[\uu,\vv] = [\uu+a\vv,\vv]$.

\msk

Em $\R^3$ podemos pensar que o determinante $[\uu,\vv,\ww]$ mede

a área da base --- a área do paralelogramo gerado por $\uu$ e $\vv$ ---

vezes a altura.

Se $\uu$, $\vv$ e $\ww$ são ortogonais entre si então

a ``área da base'' é $||\uu||·||\vv||$, e a ``altura'' é $||\ww||$.

\ssk

(Obs: em $\R^3$, $\V(a,b,c)·\V(d,e,f) = ad+be+cf$, $||\vv|| = \sqrt{\uu·\vv}$,

$\uu⊥\vv = (\uu·\vv=0)$, $\Pr_{\uu}\vv = \frac{\uu·\vv}{\uu·\uu}\uu$.)

\msk

Propriedades mais importantes dos determinantes em $\R^3$:

$[a\uu,b\vv,c\ww] = abc[\uu,\vv,\ww]$

$[\uu,\vv,\ww] = [\uu,\vv,\ww+a\uu+b\vv]$

$[\uu,\vv,\ww] = [\uu,\vv+a\uu+b\ww,\ww]$

$[\uu,\vv,\ww] = [\uu+a\vv+b\ww,\vv,\ww]$

\msk

Quase todas as idéias sobre determinantes em $\R^3$ que a gente vai
ver agora ficam mais fáceis de entender se a gente as entende em três
etapas: 1) com $\uu$, $\vv$, $\ww$ ortogonais entre si, e todos com
comprimento 1; 2) usando vetores $\uu'=a\uu$, $\vv'=b\vv$, $\ww'=c\ww$
construídos a partir dos anteriores; estes $\uu'$, $\vv'$ e $\ww'$ são
ortogonais entre si, mas podem ter qualquer comprimento, 3) usando
vetores $\uu''=\uu'$, $\vv''=\vv'+d\uu'$ e $\ww'=\ww'+e\uu'+f\vv'$.

\msk

{\bf Exercício importantíssimo} (encontrar coeficientes):

a) Encontre $a,b,c$ tais que $\V(a,b,c)·\V(x,y,z) = 2x+3y+4z$

b) Encontre $a,b,c,d$ tais que $\V(a,b,c)·\V(x,y,z)+d = 2x+3y+4z+5$

c) Encontre $a,b,c$ tais que $\vsm{1&2&3 \\ 4&5&6 \\ x&y&z \\} = \V(a,b,c)·\V(x,y,z)$

d) Encontre $a,b,c$ tais que $\vsm{u_1&u_2&u_3 \\ v_1&v_2&v_3 \\ x&y&z \\} = \V(a,b,c)·\V(x,y,z)$

e) Encontre $a,b,c$ tais que $\vsm{u_1&u_2&u_3 \\ v_1&v_2&v_3 \\ w_1&w_2&w_3 \\}
   = \V(a,b,c)·\V(w_1,w_2,w_3)$

% (find-fline "/tmp/33.jpg")



\newpage

%                                                   _ 
%   ___ _ __ ___  ___ ___       _ __  _ __ ___   __| |
%  / __| '__/ _ \/ __/ __|_____| '_ \| '__/ _ \ / _` |
% | (__| | | (_) \__ \__ \_____| |_) | | | (_) | (_| |
%  \___|_|  \___/|___/___/     | .__/|_|  \___/ \__,_|
%                              |_|                    
%
% «cross-prod» (to ".cross-prod")
% (gar181p 6 "cross-prod")
% (gar181    "cross-prod")
% (gar181p 33 "cross-prod")
% (gam172p 33 "cross-prod")

\mypsection {cross-prod} {O produto cruzado ($×$) em $R^3$}

\ssk

\def\area{\textsf{área}}

O ``produto cruzado'' (ou ``produto vetorial'') $\uu×\vv$ é definido como

se ele fosse ``uma parte da conta do determinante'': $(\uu×\vv)·\ww = [\uu,\vv,\ww]$.

Exercício: verifique que no item (e) acima temos

$\uu×\vv = \V(\uu_2\vv_3-\uu_3\vv_2, \uu_3\vv_1-\uu_1\vv_3, \uu_1\vv_2-\uu_2\vv_1)$.

\msk

{\sl Idéia importantíssima:}

1) para quaisquer $\uu$ e $\vv$, se $\ww$ é ortogonal a $\uu$ e $\vv$
e $||\ww||=1$, então o volume $[\uu,\vv,\ww]$ é exatamente a área do
paralelogramo gerado por $\uu$ e $\vv$ (exceto talvez pelo sinal);

2) para quaisquer $\uu$ e $\vv$, se $\ww$ é ortogonal a $\uu$ e $\vv$
e $||\ww||=1$, então o volume $[\uu,\vv,\ww+a\uu+b\vv]$ é exatamente a
área do paralelogramo gerado por $\uu$ e $\vv$ (exceto talvez pelo
sinal);

3) para quaisquer $\uu$ e $\vv$, se $\ww$ é ortogonal a $\uu$ e $\vv$
e $||\ww||=1$, então o volume $[\uu,\vv,a\uu+b\vv+c\ww]$ é
$c·\area(\uu,\vv)$ (exceto talvez pelo sinal);

4) para quaisquer $\uu$ e $\vv$, se $\ww$ é ortogonal a $\uu$ e $\vv$
e $||\ww||=1$, então $(\uu×\vv)·(a\uu+b\vv+c\ww)$ é $c·\area(\uu,\vv)$
(exceto talvez pelo sinal);

5) para quaisquer $\uu$ e $\vv$, se $\ww$ é ortogonal a $\uu$ e $\vv$
e $||\ww||=1$, então $\uu×\vv = \area(\uu,\vv)·\ww$ (exceto talvez
pelo sinal).

\msk

{\bf Exercício:}

Use o (5) acima para tentar descobrir quais são as duas respostas
possíveis para $\uu×\vv$ nos casos a e b abaixo, e depois compare as
suas respostas com resposta ``algébrica'' dada pela fórmula lá no alto
da página.

a) $\uu=\V(3,0,0)$, $\vv=\V(0,4,0)$, $\ww=\V(0,0,1)$

b) $\uu=\V(0,3,0)$, $\vv=\V(0,3,3)$, $\ww=\V(1,0,0)$



\newpage

% (find-fline "/tmp/34.jpg")


%   ___                    
%  / _ \ _ __  ___   __  __
% | | | | '_ \/ __|  \ \/ /
% | |_| | |_) \__ \   >  < 
%  \___/| .__/|___/  /_/\_\
%       |_|                
%
% «alguns-usos-do-x» (to ".alguns-usos-do-x")
% (gar181p 7 "alguns-usos-do-x")
% (gar181    "alguns-usos-do-x")
% (gam172p 34 "alguns-usos-do-x")
% (gam172     "alguns-usos-do-x")
% (gaq 31)

\mypsection {alguns-usos-do-x} { Alguns usos do `$×$'}

\ssk

1) $||\uu×\vv|| = \area(\uu,\vv)$

2) $\uu×\vv$ sempre dá um vetor ortogonal a $\uu$ e $\vv$

3) $\uu×\vv=\V(0,0,0)$ se e só se $\area(\uu,\vv)=0$, ou seja, se
$\uu$ e $\vv$ são colineares

(i.e., paralelos).

4) Digamos que

$r  = \setofst{A+t \uu}{t \in\R}$,

$r' = \setofst{B+t'\vv}{t'\in\R}$,

$B = A+\ww$.

Então $r$ e $r'$ são reversas se e só se $[\uu,\vv,\ww] \neq 0$.

(Se $[\uu,\vv,\ww]=0$ então $r$ e $r'$ são ou paralelas, ou coincidentes, ou se cortam).

5) Pra testar se quatro pontos $A,B,C,D∈\R^3$ são coplanares,

encontre $\uu,\vv,\ww$ tais que $A+\uu=B$, $A+\vv=C$, $A+\ww=D$;

temos $[\uu,\vv,\ww]=0$ se e só se $A,B,C,D$ forem coplanares.

6) (Difícil!) Sejam

$r  = \setofst{A+t \uu}{t \in\R}$,

$r' = \setofst{B+t'\vv}{t'\in\R}$,

$B = A+\ww$.

\def\ut#1#2{\underbrace{#1}_{\text{#2}}}

Então: $d(r,r') = \ut{\ut{|[\uu,\vv,\ww]|}{volume} / \ut{\area(\uu,\vv)}{área da base}}{altura}$.

7) (Difícil!) Sejam

$r  = \setofst{A+t \uu}{t \in\R}$,

$r' = \setofst{B+t'\vv}{t'\in\R}$,

$B = A+\ww$.

Como a gente encontra uma reta $s$ que corte $r$ e $r'$ e seja ortogonal a ambas?

Sejam $C_t = A+t \uu$ e $D_{t'} = B+t' \vv$.

Queremos que $\Vec{C_tD_{t'}}$ seja ortogonal a $\uu$ e $\vv$,

ou seja, que $\Vec{C_tD_{t'}}$ seja paralelo a $\uu×\vv$,

ou seja, que $\Vec{C_tD_{t'}}×(\uu×\vv)=\V(0,0,0)$,

ou seja, que $(D_{t'}-C_t)×(\uu×\vv)=\V(0,0,0)$,

ou seja, que $((B+t'\vv)-(A+t \uu))×(\uu×\vv)=\V(0,0,0)$,

ou seja, que $(t'\vv - t\uu + \Vec{AB})×(\uu×\vv)=\V(0,0,0)$,

o que dá um sistema que nos permite encontrar $t$ e $t'$ com poucas contas...

Sabendo $t$ e $t'$ sabemos $C_t$ e $D_{t'}$, e a reta $s$ passa por $C_t$ e $D_{t'}$.

\bsk

{\sl Agora você deve ser capaz de resolver os exercícios 1 a 20 da lista 9 da}

{\sl Ana Isabel! Yaaaaay!} $=)$ $=)$ $=)$




\newpage


%  ___           _ _          
% |_ _|_ __   __| (_) ___ ___ 
%  | || '_ \ / _` | |/ __/ _ \
%  | || | | | (_| | | (_|  __/
% |___|_| |_|\__,_|_|\___\___|
%                             
% «indice» (to ".indice")
% (mpgp 43 "indice")
% (mpg     "indice")

\directlua{PPV(psections)}
\mypsectionstex

\end{document}




%   ____ _ _   
%  / ___(_) |_ 
% | |  _| | __|
% | |_| | | |_ 
%  \____|_|\__|
%              
% «git»  (to ".git")
% (find-es "math" "material-para-GA-git")





% Local Variables:
% coding: utf-8-unix
% modes: (latex-mode sh-mode)
% ee-tla: "mpg"
% End:
